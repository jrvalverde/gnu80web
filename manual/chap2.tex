\chapter{\sf Standard Methods}
\label{chap2}
\section{\sf Preview}
The mnemonics available to access the facilities provided
by Standard Methods provide a kind of terse command-shell to
drive the whole of {\tt gnu80}.
The command themselves are placed on a 
single ``Command Record'' 
which is identified to {\tt gnu80} by a hash (\#)
in its first column. Items on the Command Record 
are separated by comma (,)
slash (/) minus (-) or space ( ).
The whole Command Record, typically, specifies the quantum mechanical
model, the choice of Gaussian basis and the use to which
this combined approximation is to be put. 
There are standard 
defaults for unspecified commands.
Before giving the definition of all the possible entries on
the Command Record, some specific examples with a brief
explanation will provide the flavour of this
command shell.
\\
\vspace{0.5cm}
\\
{\bf \# HF/STO-3G}
\\
\vspace{0.5cm}
\\
This record requests the performance of a Hartree-Fock (single
determinant) calculation using a standard {\bf STO-3G} Gaussian basis for
all the atoms of the molecule. 
By default, {\bf RHF} is chosen if the system is closed-shell or
{\bf UHF} if the system is open-shell).

\begin{center}
\fbox{
\parbox{4.2in}{\bf
Throughout this manual the terms ``Hartree-Fock'' and
``Self-Consistent Field (SCF)'' are used interchangeably
to mean the use of the Linear Expansion technique (LCAO) within
the single-determinant model. See Appendix \ref{bananas} for 
possible pitfalls of this assumption.
}
}
\end{center}
\ \\
\vspace{0.5cm}
\\
{\bf \# MP2/3-21G OPT}
\\
\vspace{0.5cm}
\\
Here, a second-order Moller-Plesset calculation of the
total energy is requested (necessarily preceded by a {\bf UHF} or {\bf RHF}
calculation, as appropriate) using a split-valence standard
basis for each atom in the molecule. The {\bf OPT} command
requests that the calculation be repeated for automatically-generated
changes in the molecular geometry until a minimum is obtained at
the OPTimum geometry.

This latter example illustrates another choice which must be
made when using a particular quantum mechanical model; in many
cases, for a combination of theoretical and practical reasons, one
must choose a particular numerical algorithm and/or some
limitations on the way the algorithm or the underlying quantum
mechanical method is implemented. 

In the particular case cited above,
the optimisation of molecular geometry, it is sometimes necessary to
specify the particular technique used to "drive" the quantum mechanical
energy calculation method; either to be sure of generating scientifically
meaningful results or to avoid excessive demands on computer resources.
Thus in the case considered here, it is a matter of computational
experience that the use of electron configurations in which the
inner-shell electrons are excited in the perturbation expansion has
minimal effect on the optimised geometrical parameters; as one
might reasonably expect. 

In point of fact, {\tt gnu80} has only
{\em one} optimisation algorithm which can be used with the {\bf MP2}
method (the Fletcher-Powell method, which does not require 
explicit derivatives of the energy with respect to the geometrical
parameters: bond distances, bond angles etc.) but in the case of the
{\bf HF} model there are three algorithms to choose from.

Choices of this kind {\em within} a given model/numerical decision
are called {\bf OPTIONS} in {\tt gnu80}. In addition to the ``major''
options of the type described above, there are many ``minor''
options available to govern the amount of output generated etc. etc.
As a general rule the major options for a given choice of
model/basis/use may be given in mnemonic form on the 
Command Record while many of the minor options must be specified
by using a specific command {\bf NONDEF} 
(for NON-DEFault options) on the Command Record followed by
a list of such options. 

The vast majority of cases of the use
of Standard Methods do not require the use of {\bf NONDEF}.
Indeed, there are {\em default} options (chosen on the basis
of much experience) for most of the types of calculation which
have to be done and, in most cases of the use of Standard Methods
the user need not even be aware of the existence of options.

In the case of {\bf MP2} calculations {\tt gnu80} automatically chooses
the Fletcher-Powell optimisation algorithm and the ``Frozen
Core''  option which excludes the use of core excitations in 
the perturbation expansion. However to illustrate how these options
might be explicitly specified, the original Command Record
is completely equivalent to the record
\\
\vspace{0.5cm}
\\
{\bf \# MP2(FC)/3-21G OPT=FP}
\\
\vspace{0.5cm}
\\
Where {\bf (FC)} indicates the Frozen-Core option {\em within}
the {\bf MP2} method and {\bf OPT=FP} explicitly choose the Fletcher-Powell
optimisation technique. If the {\bf MP2} calculation has to include
{\em all} possible terms in the perturbation expansion then  {\bf (FC)}
is replaced by {\bf (FULL)} on the Command Record.

The full range of options accessed by this mnenonic method on the
Command Record is given in the section \ref{commands}.

Of course, the Command Record has no data describing the
particular molecular system under investigation, this is 
provided by a separate section of the whole input data to {\tt gnu80}.

The other pieces of data required to complete the description of
the molecular electronic structure are the number of electrons
and the spin multiplicity of the system.
The number of electrons is determined from the atomic numbers
of the atoms of the molecule which themselves are determined from
the atomic chemical symbols which are read in as part of the molecular
geometry.

If the molecule is charged this information must obviously be
supplied and the spin multiplicity must also be supplied.

Because the Command Record specifies the sequence
of calculations to be carried out and the order in
which they must be performed i.e. the set of modules
of {\tt gnu80} to be executed, this sequence can be thought of as
a {\em path} or {\em route} through {\tt gnu80}.
This ROUTE, once generated from the Command Record, 
is stored and used as control information during a {\tt gnu80}
run.

The principle difference between Standard Methods and General Use
is the way in which this route is generated. Standard Methods
involve the automatic generation of the route from the
Command Record while General Use enables the route to be
specified explicitly by the user.
The {\bf NONDEF} command provides a ``intermediate''
method of modifying a given route through {\tt gnu80}.

This section has given a general overview of the command
record and its use; later sections give detailed
specifications of the possible entries on the Command Record.
\section{\sf Input Organisation}
\label{inorg}
Input to {\tt gnu80} consists of records of alphanumeric information
usually free of (FORTRAN) format constraints. In the case of alphabetic
information upper and lower case letters are completely interchangeable
and may be used at the user's discretion to enhance the legibility of
the input. Thus, the cobalt atom may be identified as CO, co, Co or even
cO. The same applies to the commands on the Command Record; the example in
the last section could well have been given as:
\\
\vspace{0.5cm}
\\
{\bf \# hf/STO-3g}
\\
\vspace{0.5cm}
\\
for example. The first thing that {\tt gnu80} does with an input record
is to convert any alphabetics to upper case and it works internally entirely
in upper case.

\begin{center}
\fbox{
\parbox{4.2in}{\bf
The input of a {\tt gnu80} run consists of several separate
{\em sections} separated by blank records. The Input Sections
themselves are not identified except by their {\em position}
in the input file and the blank records separating them from
their predessors. For example, the ``Title'' Input Section
is not preceeded by the word TITLE, it is simply those records
following the blank record after the Command Record.
Similarly, the ``Variables'' Input Section is just those records
(assuming them to be in the correct form) which follow the blank
record after the Z-matrix; the Section is not announced by the word
VARIABLES, for example. 
}
}
\end{center}
 
Each section consists of one or more records (or lines) and is
terminated by a blank record (or line). We shall use the 
word ``record''
from now on to mean record or line. The list of input sections, in the
sequence in which they {\em must} appear if they appear at all, is:
\begin{center}
\fbox{
\parbox{4.2in}{
\bf
\begin{enumerate}
\item File Control Records (if any)
\item Command Record  (and Non-Default options, if any)
\item  Title
\item Molecule Specification
\item Variables Specification              (if any)
\item Constants Specification              (if any)
\item  General Basis Specification                 (if any)
\item Alteration of Configuration $(\alpha)$ (if any)
\item Alteration of configuration $(\beta)$  (if any)
\end{enumerate}
}
}
\end{center}
\ \\
Each Section is described  separately  below.
Only the  sections 2 --- 4 are {\bf required} in every Job. If a
particular section is not supplied, its terminating blank record should
{\em not} be included in the input file.

The first section of this input structure, consisting of up to
four
{\em File Control} commands (identified by a percent sign, \% ,
in column 1) can be used to preserve the
details of the job for a re-start. These records are not essential
for the preparation of successful jobs and a discussion is deferred
until the {\bf \#RESTART} command is introduced.
\section{\sf The Command Record}
\label{commands}
This single record requires the most description since it
defines the {\em nature} of the calculation to be performed; the
quantum mechanical model, the numerical accuracy (Basis set)
and the particular application. The commands on the Command Record
are very terse but require non-terse specifications.

The Command Record itself is identified by a hash (\#) in the
first column. On initiation, {\tt gnu80} reads records 
with no interpretation until
it finds a Command Record. Immediately following the \# on the
Command Record may be one of N P or F which set certain ``global''
printing options for the whole run:
\begin{description}
\item[N] Suppresses as much output as possible consistent with
providing a useful summary of the job. Use {\bf N} when there are
no problems are anticipated with the job.
\item[P] Generates more verbose output from various links (particularly
{\bf HF}) and prints a summary of the passage through the Links. Useful
for general information occasionally and if convergence difficulties are
anticipated.
\item[F] Generates messages from the File I/O routines; only
useful for de-bugging.
\end{description}
The \# (or \# N, \# P or \#F) {\em must} be followed by a space
before the substance of the Command Record is given.

Using Standard Methods, the Command Record specifies three general
pieces of information about the whole calculation:
\begin{itemize}
\item The Model of the molecular electronic structure to be used
\item The accuracy of the expansion of the molecular orbitals (the
particular basis used)
\item The physical problem to be attacked by the use of this particular
combination of model and accuracy
\end{itemize}
Obviously, some thought must be given to the first two of these
items in order that the results are useful and applicable to the third.
Some of the grosser errors are indicated in Appendix \ref{bananas}.

In addition to these three main commands the Command Record
may be used to choose some commonly-required modifications of
the way in which the calculation is to be performed; there are
some Command Record mnemonics for certain common OPTIONS within the
model/basis/use choice.
\newpage
\subsection{\sf Models of Electronic Structure}
One of the following commands must always be given (either explicitly
or by default) for a {\tt gnu80} run; each one specifies the overall approximation
to be made of the molecular electronic structure of the system under
investigation. 

There are three classes of model: 
\begin{enumerate}
\item Variants of the 
Molecular Orbital or Hartree-Fock model, 
\item The (non-variational) Moller-
Plesset perturbation method which includes some electron correlation,
\item The (variational) Configuration Interaction method using
double substitutions which also includes electron correlation.
\end{enumerate}
Factors influencing the choice of model are the obvious ones
of accuracy against consumption of resources. 

If the job involves the optimisation
of molecular geometry, the Hartree-Fock methods are often the only
practical choice, since they are inherently less expensive of
computer resources particularly since it is only possible to
use analytical gradient methods at the Hartree-Fock level in {\tt gnu80}.
In any case, geometry optimisations will usually {\em begin} by using the
Hartree-Fock model and  possibly be refined using the correlated
methods.

The possible commands to choose a model of electronic structure, together
with their major (command-record) options are given below: \\

\begin{description}
\item[HF   (or RHF, UHF)] These commands request Hartree-Fock calculations. 
If {\bf HF} is given, the default
based on the (supplied) multiplicity is applied. 
{\bf RHF} (Closed-Shell Restricted Hartree-Fock) is used for singlets and
{\bf UHF} (Unrestricted Hartree-Fock, actually Different Orbitals
for Different Spins: DODS)
for higher multiplicities. In the latter case, separate
$\alpha$ and $\beta$ orbitals will be computed.  

Similar
operations are implied by specification of {\bf U} or {\bf R} with other
procedures ({\bf MP2} vs {\bf RMP2} vs. {\bf UMP2}, for instance); see below.
\item[ROHF]   This command requests a Restricted Open-shell 
Hartree-Fock calculation
for those non-singlet states which can be represented by a single
determinant.  

The resulting molecular orbitals will be either
doubly or singly occupied and be spatially orthogonal to each other unlike
the UHF spatial orbitals for different spins.

The single determinant function (and the component Molecular Orbitals
generated by this procedure do not have many of the familiar
formal properties of Hartree-Fock (Self Consistent) functions. Koopman's
theorem is not valid and the Brillouin theorem is not obeyed, for example.
These differences all arise from the fact that the ROHF wave function
is a {\em constrained} single determinant.
\item[SCFDM] This command requests the use of a Direct Minimisation
method to minimise the Hartree-Fock Energy Functional instead of
the repeated-diagonalisations method used by {\bf HF, UHF} and 
{\bf ROHF}. It is more time-consuming than the standard procedure
but its convergence properties are better. It should be used
when {\bf HF} or {\bf UHF} do not converge (it is not available
as a replacement for {\bf ROHF}).
\item[MP2 (or RMP2, UMP2)]  Requests second order 
Moller-Plesset perturbation calculation (restricted or
unrestricted, depending upon {\bf R} or {\bf U}). Both {\bf MP2} and
{\bf MP3} also generate a Hartree-Fock calculation as a necessary
precursor to the perturbation calculation which uses the Self-Consistent
Molecular Orbitals.
\item[MP3 (or RMP3, UMP3)]  Requests second and third order 
Moller-Plesset perturbation calculation (restricted or
unrestricted, depending upon {\bf R} or {\bf U}).
\item[CID]    Requests a Configuration Interaction calculation with
Double excitations. Again, this command requests a Hartree-Fock
calculation (of the appropriate type: {\bf RHF} or {\bf UHF})
to generate the Molecular Orbitals for the CI.
\end{description}
\subsubsection{\sf Commands with Options Associated with the Model Commands}
There are a few options available via mnemonics on the Command
Record for the Hartree-Fock commands:
\begin{description}
\item[VSHIFT] Does nothing unless given an option; ({\bf VSHIFT=nnnn}).
The occupied/virtual energy gap is increased by $nnnn/1000$. This is just
an implementation of the ``Level-Shifter" method to improve SCF
convergence.
\item[SCFCYC]  This is used to specify the maximum number of SCF cycles
allowed. it is meaningless by itself, and is used by
{\bf SCFCYC=number}. The default for ` number' is 
20 cycles for {\bf SP, HF} jobs,
and 32 cycles for {\bf OPT} or post-SCF jobs. The use of the
``Level Shifter" Command {\bf VSHIFT} to improve convergence
resets the default value of cycles to 50; this can be over-ridden
as usual by the {\bf SCFCYC} command.
\item[COMPLEX] This option allows the orbitals of the single
determinant wave function to be complex. The option is only available
for {\bf RHF} i.e. not for {\bf UHF}, {\bf ROHF} or any post-SCF
calculation.
\end{description}
The non-Hartree-Fock commands all have a similar set of
OPTIONS which can be given on the Command Record; typically in
parenthesis immediately following the command. The options
describe the {\em extent} to which the perturbation or configuration 
interaction calculation will be carried out:
\begin{description}
\item[FC]  This is the default in the absence of any
explicit option being given for {\bf MP2, MP3, CID} commands.
Its action is to limit the number of terms in the perturbation or CI
expansion by excluding those arising from
the excitation of the inner shell electrons.

{\bf FC} is a mnemonic for Frozen Core.
\item[FULL] If the {\bf FULL} option is given
all possible terms in the perturbation or CI expansion are used even those
involving the excitation of the atomic inner shells or cores.
\end{description}
Note that ONLY ONE of {\bf UMP2, MP2, UMP3, MP3,
CID} may be used on a given Command Record. Typical Command Records
using the correlated models of electronic structure are:
\begin{description}
\item[\# UMP2(FULL)/3-21G]
\item[\# CID(FC)/6-31G**]
\item[\#P MP3(FC)/STO-3G]
\end{description}
where the command following the slash (/) requests a type of 
Gaussian basis detailed in the following section.
\newpage
\subsection{\sf Basis Sets Internal to {\tt gnu80}}
\label{basis}
When a {\em model} of electronic structure has been chosen, the most
important {\em numerical} approximation within that model is the nature
and length of the expansion used to approximate the Molecular Orbitals
within that model. That is, one must make a {\em choice of (Gaussian)
basis}.

In the same mnemonic spirit introduced for the quantum
mechanical models there are a number of stored Gaussian basis sets
available in {\tt gnu80} which may be used via the following
standard (descriptive) mnemonics. 
Thes mnemonics are also commands to {\tt gnu80} and use of {\em one} of these
on the Command Record
will generate a basis set for the system described in the
Molecule Input Section.
This command chooses an overall molecular basis
which has the same {\em type} of
atomic Gaussian basis on every atom in the molecule.
{\bf 
\begin{itemize}
\item STO-NG, STO-NG* (for N = 1,6) e.g STO-3G*, STO-4G
\item  3-21G, 3-21G*, 3-21G**
\item  4-21G, 4-21G*, 4-21G**
\item  6-21G, 6-21G*, 6-21G**
\item  4-31G, 4-31G*, 4-31G**
\item  6-31G, 6-31G*, 6-31G**
\item  6-311G, 6-311G*, 6-311G**
\item  LP-31G, LP-31G*, LP-31G**
\item  LP-41G, LP-41G*, LP-41G**
\item  LANL1MB, LANL1DZ (LP-31G is a synonym for LANL1DZ)
\end{itemize}
}
The meanings of these mnemonics are assumed to be familiar.

By default, {\bf STO-3G} is selected. The only  option available is the
specification of the type of {\em d} or {\em f} orbitals used
(see below). 

The ``Local
Potential'' Basis Sets  (e.g. {\bf LP-31G}) will lead to 
calculations involving valence
electrons only; that is, selection of any of these bases
will {\bf automatically} use an associated Local Potential
on each atom in the molecule (except hydrogen and helium).
The particular choice of Local Potential used is explained elsewhere.
(see the description of the {\bf PSEUDO} command.

Note that the use of the {\bf PSEUDO} command simply uses the
Local Potentials on the atoms; {\bf it does not involve any
choice of basis} so that a basis must be specified. Usually, this
means the use of a user-supplied basis (see the {\bf GEN} command)
since the other bases are full bases including core orbitals and
the use of the Local Potential basers also chooses an approriate
basis so that {\bf PSEUDO} is redundant there.
%
\subsection{\sf ECP Basis Set Types in gnu80 and g94}
The Basis Set commands in {\tt gnu80} have been changed to make
them compatible with {\tt g94}. They now are:
\begin{itemize}
\item
{\bf LANL1MB}; this is the same basis as g94. \\
It uses the Minimal Los Alamos Bases for atoms from Na - Bi and
an STO-3G set for atoms H-Ne. \\
The Los Alamos Potentials are used for atoms Na-Bi and NO potential
is used for first-row atoms; all electrons are treated explicitly.
That is, for example, the Carbon atom has a full (5) basis
(1s, 2s,2p) and atomic charge 6.
\item
{\bf LANL1DZ}; again the same as g94. \\
The Los Alamos Double Zeta Basis is used for atoms Na-Bi
and a Dunning D95V ("Double Zeta for valence") for atoms
H-Ne. \\
The Los Alamos Potentials are used for atoms Na-Bi and NO potential
is used for first-row atoms; all electrons are treated explicitly.
That is, for example, the Carbon atom has a full (9) basis
(1s, 2s, 2s',2p,2p') and atomic charge 6.
\item
{\bf LAL1STV}; this is not available in g94. \\
It uses the Minimal Los Alamos Bases for atoms from Na - Bi and
an STO-3G set for atoms H-Ne. \\
The Los Alamos Potentials are used for atoms Na-Bi and a CHF potential
is used for first-row atoms.
That is, for example, the Carbon atom has a valence (4) basis
( 2s,2p) and atomic charge 4 plus the potential.
\item
{\bf LAL1LP3}; this is not available in g94. \\
The Los Alamos Double Zeta Basis is used for atoms Na-Bi
and a LP-31G ("Double Zeta for valence") for atoms
H-Ne. \\
The Los Alamos Potentials are used for atoms Na-Bi and a CHF potential
is used for first-row atoms.
That is, for example, the Carbon atom has a split-valence (8) basis
LP-31G
(2s, 2s',2p,2p') and atomic charge 4 plus the potential.
\end{itemize}
\begin{quote}
{\bf Notice that this means that previous {\tt gnu80} input
files, which used LANL1MB for what is now LA1STV and
LANL1DZ for what is now LA1LP3, must be changed if the same
basis is required.}
\end{quote}
This is awkward but the non-compatibility with g94 is also
awkward.
%
\subsection{\sf GEN: User-supplied Bases}
Obviously, the user may wish to supply {\tt gnu80}
with a basis set of his own choice. This is possible
by use of the command {\bf GEN} on
the Command Record in place of a basis set mnemonic.
The details of the input of a basis which is not internal to {\tt gnu80}
are given below in Section \ref{genbasis}.
\subsection{\sf nD, nF; Cartesians or Cubic Harmonics?}
The only other basis-related command  available once an
overall  basis set command
has been given is in the choice of Cartesian or Cubic
Harmonic functions for {\em d} and {\em f} functions.

The ``Cartesian'' Gaussian bases generate a set
of 6 {\em ``d'' } functions which are equivalent to the
usual set of 5 (Cubic Harmonic real) d functions plus
an additional ``s'' -type Gaussian.
Likewise, the Cartesian set of {\em ``f'' } 10 functions
is equivalent to a set of 7 (Cubic Harmonic real) {\em f } functions
plus a set of 3 {\em p } functions.

For some purposes the presence of these additional {\em s}-type 
or {\em p}-type functions produces
an ``unbalanced'' basis and may lead to unforeseen results. It is
possible to specify which case is required by the use of
{\bf 5D} or {\bf 6D} and {\bf 7F} or {\bf 10F}
as appropriate; thus the above Command Record
could be replaced by:
\ \\
\vspace{0.25cm}
{\bf \# MP2/6-31G* 5D or \# MP2/6-31G* 6D } 
\ \\
depending on individual requirements. The defaults are {\bf 5D},
{\bf 7F}.

The relative merits are obvious; {\bf 6D} introduces an additional
degree of variational freedom for each {\em d}-set at the risk
of unbalancing the {\em s}-type functions in the basis.
\subsection{\sf Model/Basis Summary}
The quantum mechanical procedure and the basis-set commands may be
conveniently combined as a specification of the theoretical model/numerical
approximation scheme.
Thus the Command Record:
\ \\
\vspace{0.25cm}
{\bf \# MP2/6-31G*}
\ \\
\vspace{0.25cm}
Requests an {\bf MP2} (Frozen Core by default) 
calculation using the {\bf 6-31G*} basis set.
The established convention is to use a slash (/) to separate
the model command (in this case {\bf MP2}) from the basis-set
command ({\bf 6-31G*} here) to provide a convenient mnemonic for
the ``quality'' of the calculation which is easily
recognizable; although any of the Command Record command separators
would be acceptable.
\newpage
\subsection{\sf Uses of the Quantum Model}
\begin{description}
\item[SP] Requests a ``single point'' calculation; that is
an energy calculation at the specified nuclear geometry.
No optimisation of geometrical parameters is performed. 
This is the default type of job
and so is not often explicitly specified.
\item[OPT]  Requests that a geometry OPTimization is to be performed. 
The (geometrical)
variables listed in program input Section 4 (see below) will be
adjusted until a stationary point on the potential surface is
found. By default the best available procedure is selected.
thus, you can keep up with current developments by simply
supplying {\bf OPT} with no options. However, if conditions demand
a particular type of optimization, it may be chosen by
supplying the appropriate option.
\item[FORCE]  This requests a single analytical calculation of the forces on
the nuclei and that the energy derivatives with respect to the
geometry parameters be evaluated. This is currently available
only for HF energies.
\item[NONDEF]  This command indicates that additional non-default options will
be read in from the next recordset (see below).
\item[TEST]    This command indicates that this is a test job. the route will
be generated as normal. The job is aborted before the calculation of
any integrals.
\item[NONSTD]    This command indicates that this is not a standard job;
the desired route will be explicitly supplied in the form of
overlay numbers, options, and segment numbers. This is, in fact,
the way in which the method of General Use is indicated to {\tt gnu80}.
\end{description}
\subsubsection{\sf OPTIONS for the OPT Command; OPT=xxxx}
The {\bf OPT} command is, of course, the most widely used command
specifying the use of a quantum mechanical model and there are a number
of OPTIONS which may be specified by mnemonics on the Command Record:
\begin{description}
\item[GRAD]   Requests use of analytical gradients and is only
available for HF (RHF or UHF) calculations.
\item[MS] Use Murtaugh-Sargent optimization [00]. It does use
analytical gradients and is only available for {\bf HF} ({\bf RHF} and
{\bf UHF}) calculations. {\bf OPT=MS} is generally slower but more
reliable then {\bf OPT=GRAD}.
\item[FP] Fletcher-Powell optimization [00]. This option does not
require analytical derivates and may be used in conjunction
with any procedure.
\item[READFC] For the first point in a gradient optimisation, read the force
constants from the guess file.  These must have been produced
in a previous run, for example by the {\bf STARONLY} option.
\item[TS] Optimize to a transition state - a turning point on the
energy surface for which the Hessian has one and only one negative eigenvalue
pictorially a saddle point on the surface.
\item[STARONLY] For {\bf GRAD}  optimisations, 
this option requests that the force constants associated with
the movements in the degrees of freedom specified in the
{\bf Variables} input section will be estimated numerically.
These force constants may be used as input to the {\bf MS} optimisation
program as starting points for the second derivative matrix.
The second
derivatives of the energy will be computed as directed on the variable
definition records (see below).
\item[Defaults:]
The default options for the command {\bf OPT}  are {\bf GRAD} for
Hartree-Fock ({\bf RHF} or {\bf UHF}) calculations and {\bf FP} for
post-Hartree-Fock.
\end{description}
\newpage
\subsection{\sf Miscellaneous Commands}
These command and associated options provide
a variety of utilities in {\tt gnu80}
\begin{description}
\item[SAVE]    saves selected data on the {\bf GUESS} file. by default, this
command saves both the Basis Set and the Molecular Orbital
coefficients. (see also {\bf\#RESTART} and the File Control Commands)
\item[Options] for the {\bf SAVE} command ({\bf SAVE=xxxx}):
\begin{description}
\item[BASIS] saves the basis set.
\item[MO] saves the Molecular Orbital coefficients.
\item[FC] saves the Force Constants, if these are
calculated. Note the unfortunate collision with the Frozen Core
mnemonic available as an option for another command; there should
be no danger of confusion.
\end{description}
\item[GUESS]   specifies initial guess options.  This does nothing unless one
of the options is actually given. In the absence of the command, a default
projected Huckel guess is used. 
\item[Options] for the {\bf GUESS} command ({\bf GUESS=xxxx}):
\begin{description}
\item[READ] Read initial guess. This guess must have been 
created as output from a
previous job and will be associated with a File Control
Command to specify the file to read the guess from. (see also {\bf RESTART}).
\item[CORE] Diagonalize core Hamiltonian.
\item[ALTER] Indicates that the orbitals selected for occupation in
the HF wave function should not be those of lowest energy. If
this option is used, information about the alteration of
configuration will be expected in Sections 7 and 8 of \ref{inorg}.
{\bf ALTER} may
be used itself as a command, being equivalent to {\bf GUESS=ALTER}.
\item[PRINT] Causes the initial guess to be printed.
\item[ONLY] Results in a route which goes only as far as 
L401 (the module which generates the guessed coefficients), and
the guess is automatically printed when this option is
selected. This is useful in preliminary runs to check if
configuration alternation is necessary.
\end{description}
\item[NOEXTRAP] Issuing this command will cause all extrapolation in the
SCF to be suppressed.
\item[COORD]  Indicates that the geometry will be supplied in the form of
Cartesian coordinates (not a Z-matrix). No optimization is possible in this
case.
\item[NORAFF]  This command demands that the ``regular'' 
integral format  be used.
\item[OPTCYC]  This sets the maximum number of optimization cycles. The
format is {\bf OPTCYC = number}, for instance.
The default for ` number' is 10 plus one for each degree of freedom,
with a maximum of 20 cycles. If this number is too small the chances
are that there is something seriously wrong withe the model or the
basis!
\item[SYMM]  Specifies how symmetry is to be used in the calculation. By
default, symmetry is used whenever possible, so this command
means nothing unless an option is provided.
\item[Options] for the {\bf SYMM} command:
\begin{description}
\item[INT] Use symmetry in two-electron integral evaluation.
\item[NOINT] Do not use symmetry for these integrals.
\item[GRAD] Use symmetry in gradient evaluation.
\item[NOGRAD] Do not use symmetry in gradient evaluation.
\end{description}
\item[NOSYMM] This command supresses any use of symmetry 
in the calculation.  By
default, the symmetry of the molecule may be used to avoid
calculation of some integrals either because they are zero by symmetry
or because they are equivalent to other (calculated) integrals.
\item[NOPOP]   Eliminates wavefunction and full population analysis.
\item[MINPOP] Causes a ``minimal'' population analysis. The orbital
symmetries are printed, along with the 
``condensed to atoms'' summary.
\item[CNOE] ``Causes a Complete Neglect of Everything'' 
calculation to be performed. This command does no calculation at all;
it simply reads the data and generates the ROUTE. It is useful to
generate ROUTEs for later modification.
\item[ALTER] Requests that the initial guess MO coefficients be rearranged.
The program will read pairs of integers indicating orbitals
which are to be interchanged in the initial guess. This input
forms one input section, or two input sections if the
calculation is {\bf UHF} (one for $\alpha,$ one for $\beta$).
\item[PSEUDO] Requests that a model potential be substituted for the core
electrons. This is automatically selected if one of the {\bf LP-}
bases or the {\bf LANL1MB or LANL1DZ} 
bases have been specified on the Command Record.
\item[UNITS]  Definition of the units used in the Z-matrix
i.e. in the Molecule Input Section of the input data. Does nothing
without an option:
\item[Options] for the {\bf UNITS} command ({\bf UNITS=xxxx}):
\begin{description}
\item[ANG] Bond lengths are in Angstroms.
\item[AU] Bond lengths are in Bohrs.
\item[DEG] Angles are in degrees.
\item[RAD] Angles are in radians.
\end{description}
The default values are {\bf (ANG, DEG)}.
\end{description}
\newpage
\subsection{\sf Commands not yet Implemented}
Since {\tt gnu80} is, in one sense, a ``snapshot'' in the development
of the GAUSSIAN series of programs, some parts of the code are behind
others in the implementation. There are a number of commands which are
recognised by the parser but the code to carry through the requested
calculation is not present or not complete. In these cases a
message is printed to that effect and the run is stopped. Clearly, the
code to implement these commands are prime examples of additions to
{\tt gnu80} which could be carried through simply at the
level of the numerical calculation with no changes necessary to
the parse tables.

The unimplemented commands are:
\begin{description}
\item[GRAD] for post-SCF models
\item[MP4] fourth-order Moller-Plesset perturbation calculation
of the correlation energy.
\item[FREQ] calculation of the vibration frequencies and
normal co-ordinates (requires second derivative integrals)
\item[CISD] Configuration interaction with {\em single and} double
excitations.
\item[STABIL] test the stability of (HF)SCF wavefunctions for
stability with respect to certain classes of change.
\item[FORCE] force calculations for post-SCF models
\end{description}
\subsection{\sf System Defaults}
By default the following commands and options are selected:
\begin{itemize}
\item {\bf SP}
\item {\bf HF}  (i.e. {\bf RHF} for closed shells, {\bf UHF} for open shells.)
\item {\bf STO-3G}
\item {\bf OPT = GRAD} for SCF optimisations, {\bf OPT=FP} for
post-SCF optimisations ({\bf MP2, MP3, CID})
\item {\bf FC} Frozen Core for all multi-determinant calculations
({\bf MP2, MP3, CID}).
\end{itemize}
Note, however, that if an {\bf HF/STO-3G} calculation is required
(all defaults), a blank
Command Record is {\em not} permitted.
\subsection{\sf Examples of Valid Command Records}
Some examples of correct Command Records which
generate valid routes may help make all this
more clear:
\begin{itemize}
\item \#  RHF/STO-3G, SAVE=MO
\item \#  HF/3-21G, VSHIFT=500
\item \#  MP3(FULL)/6-311G** OPT
\item \#  UHF-CI=FC/6-31G* OPT=FP
\item \#  RHF/6-31G*, GUESS=READ
\item \#  HF/STO-3G, SCFDM, ALTER, SAVE
\item \#  RHF/STO-3G OPT NONDEF \\
(blank record) \\
3/34=1,35=4;3(2)/34=1,35=4;
\end{itemize}
This last Command Record is used to select a standard route, 
but the options 34
and 35 in the first and the second occurence of overlay 3 are set to 1
and 4 respectively. The meaning of the last example will be 
clear after the section on General Use has considered explicit
routes and options.

Note that the possibility of separating commands and options by
spaces, slashes or commas may be used to make the Command Record
more intelligible to according to individual choice.

This completes the description of the commands and options
which may be used on the Command Record input to {\tt gnu80}. The Other
sections of data usually constitute more data but require considerably less
discussion.
\section{\sf Title Input Section}
This section is required in the input but is not interpreted in any
way by the {\tt gnu80} system. It appears in the output for purposes
of identification and description. Typically, this might contain
compound name, the symmetry, the electronic state and other relevant
information.
The title section cannot exceed five records. However, a single record
is usually adequate. 
Remember that the title section must have a terminating blank record.
\section{\sf Molecule Input Section}
\label{molinput}
This section is, of course, always required. It specifies the nuclear positions
and the number of electrons of $\alpha$ and $\beta$ spin. The input is
free-field; the several items on each record may be separated by either
blanks or commas.

The first record of the section specifies the net electric charge
(signed integer) and the spin multiplicity (positive integer). Thus,
for a neutral molecule in a singlet state, the entry ` 0 1' is
appropriate, while for a single charged anion radical, ` -1 2' 
would be used.

The remaining records are used to specify the relative positions of
the nuclei. Most of these will be real nuclei, used later in the
molecular orbital computation. However, it is frequently useful to
introduce ``dummy nuclei'' which help specify the geometry but are
ignored subsequently; their use will become clear in examples given
below.

Since the Molecule Input Section is the most demanding to prepare,
it is worth a brief overview of the method used to specify the
molecular geometry in {\tt gnu80}.
\subsection{\sf The Z-Matrix, an Overview}
Before giving the method used by {\tt gnu80} to read the geometrical
information necessary for a job, it is useful to have an
overview of the so-called Z matrix method. Some of the
information in this outline is, of course, duplicated in the
reference section of the manual below.

There are, at least, three obvious methods to supply the
data specifying the relative positions of the atoms in a
molecule to a program:
\begin{enumerate}
\item Use a ``laboratory'' Cartesian co-ordinate
system and give all the atomic positions as absolute
Cartesian co-ordinates.
\item Still using a laboratory co-ordinate system, give the
molecular symmetry (say the point group symbol) and the absolute
Cartesian co-ordinates of the symmetry-distinct atoms in the molecule;
allowing the program to apply the molecular symmetry to generate the
full Cartesian co-ordinates for the entire molecule.
\item Use an ``internal'' co-ordinate system and specify
only the {\em relative} positions of the atoms and leave the program
to provide an origin and orientation of the laboratory system and do the
conversion to a Cartesian set.
\end{enumerate}
Obviously, in cases 1 and 3 the symmetry group of the molecule
may be {\em inferred} from the supplied co-ordinates.

The advantages and disadvantages of each approach are clear:
\begin{enumerate}
\item This choice is only really practical for small molecules or
for very symmetrical molecules.
\item This method is most convenient for highly symmetrical systems;
in most cases of low symmetry, it will degenerate into 1.
\item This is the most flexible approach, particularly since most
experimental molecular data are presented as bond distances, bond
angles etc. referred to a ``natural'' non-redundant
system of internal co-ordinates.
\end{enumerate}
However, when the question of the {\em automatic} generation
of changes in molecular geometry during a structure optimisation arises,
it is obvious that the use of the natural molecular parameters
of choice 3 above is the only viable option. The variation of
what are essentially arbitrary Cartesian co-ordinates is bound to be
inefficient and redundant at best. At worst, the use of three independent
co-ordinates for each atom in a molecule makes such automatic 
optimisations impossible.

That is, a description of molecular geometry in terms of the
essential molecular geometrical parameters (bond distances and angles)
is the only sensible way to both input data to a program and to
organise such data internally to allow automatic structure variations
which are likely to be chemically meaningful and numerically
efficient.

The method used by {\tt gnu80} (and by all the GAUSSIAN series and
many other {\em ab initio} programs) is the so-called
{\bf Z-matrix} method.

The technique is to ``walk '' through the molecule,
specifying the position of each atom as it is encountered by
its distance from and orientation to the other atoms already
encountered. In this way the natural geometric parameters used by
chemists and spectroscopists, can be used directly to generate 
an overall molecular structure which is unique but not (as in
the case of Cartesian co-ordinates) over-determined. The
overall orientation of the molecule with respect to the
laboratory co-ordinate system is not specified. This latter
freedom is used by {\tt gnu80} to position and orient the molecule
in a way which co-incides with the standard group-theoretical
choice of Cartesian axes if this is useful.

To completely specify an atom by this method, we need its atomic
number and enough geometrical information to place it uniquely
with respect to the other atoms in the system. 
\begin{enumerate}
\item As a matter of convience,
the atomic number is determined by the chemical symbol (H, LI, MN, etc.)
and so this is the way the atomic number is supplied.
\item The ``first'' atom in the molecule needs no geometrical
information, it is specified completely by its chemical symbol and all
other atoms' positions are referred (either directly or indirectly)
to it. As a convenience an atom' s label may have other characters
following the chemical symbol in order to make the data more legible to
humans; so, for example, an atom may be labelled C1 or LI4 etc.
Let us assume that we have labelled our first atom C1. {\tt gnu80}
internally places this atom at the origin of the laboratory Cartesian
co-ordinate system. \\
The form of the input data (the line of the Z-matrix) is simply
\begin{verbatim}
      C1
\end{verbatim}
\item The next atom, which typically will be an atom formally bonded
to the first (i.e. a nearest neighbour) although, logically, it need not
be, has its nature and position completely specified by its chemical
symbol and its distance from the first atom; no angles are involved.
Let us assume that it is called C2. \\
The Z-matrix line of input consists of the label of the second atom,
the label of the atom to which its distance is to be given and that distance:
\begin{verbatim}
      C2 C1 RC2C1
\end{verbatim}
where {\tt RC2C1} is the distance between C1 and C2 and may be supplied
(in Angstroms) as a number like 1.4 or may be left as an alphanumeric
string whose numerical value is supplied later. Clearly, it is more
convenient to call it {\tt RC2C1} than {\tt LUCIFER}, for example,
but both are allowed.
The {\em direction} from the first to the second atoms is arbitrary and
is chosen by {\tt gnu80} as the Cartesian z axis.
\item The third atom (C3 say) needs at least one distance and an angle
to complete its specification. Since any triatomic is planar, {\tt gnu80}
places the third atom in the xz plane of the Cartesian system, which
means that the first three atoms all lie in the xz plane.
As before, C3 is specified by the three items {\tt C3 C2 RC3C2}
if the distance between C3 and C2 is the most convenient to supply.
But this time the {\em angle} between the C3-C2 and C2-C1 bonds must be
given. This information is supplied by giving the label of the
third atom in the chain and the value of the angle (in degrees):
\begin{verbatim}
      C3 C2 RC3C2 C1 A
\end{verbatim}
where {\tt A} is the C3C2C1 angle.
\item The fourth and (all subsequent atoms) requires more information
to specify it uniquely since it may well be out of the plane defined
by the first three atoms. 

The most convenient way to do this (with one important exception, which
will be treated separately later) is to use a distance and a bond angle
as in the previous case and to specify the {\em ``out-of-plane
angle''} formed by the new bond. Traditionally, chemists have used
the Newman projection to define this {\em dihedral angle}.

If the four atoms in the chain are C1-C2-C3-C4, then, looking along
the C3-C2 bond (i.e. from C3 towards C2), the C3-C4 bond and the C2-C1
bond do not eclipse each other (in general). The {\em clockwise}
angle by which the C3-C4 bond must be rotated to bring it directly over the
C2-C1 bond is the required dihedral angle (in degrees). 
The specification of the fourth (and subsequent) atoms is then given
by:
\begin{verbatim}
      C4 C3 RC4C3 C2 A2 C1 D
\end{verbatim}
where {\tt RC4C3} is the bond distance from {\tt C4} to {\tt C3},
{\tt A2} is the angle {\tt C4C3C2} and {\tt D}
is the dihedral angle.

The simplest case is
perhaps the hydrogen peroxide molecule, $H_2 O_2$. The dihedral
angle is the angle between the two O-H bonds when viewed along the O-O
bond. In this simplest case the dihedral angle may be taken as
$\alpha$ or $360 - \alpha$ since there are no other atoms involved.
In general, however, the {\em sense} of the angle must be preserved to
ensure a correct orientation of subsequent atoms (either on the chain
or branched)
\end{enumerate}
This scheme is sufficient to specify completely the geometry
of molecules of arbitrary complexity with the following provisos:
\begin{itemize}
\item When a molecule is not basically a chain but is more
``compact'' i.e. a central branched system like $CH_3F$,
for example, the ``dihedral angles'' can still be defined by
the above procedure and are acceptable to {\tt gnu80} but they are not the
sort of angle which the chemist normally calls dihedral angles. This
is a small price to pay for a coherent system.
\item Linear Molecules have dihedral angles zero. This is not a problem
for actual linear molecules but it {\em is} a problem for larger
molecules with {\em linear sections}. In these cases, passing along 
a linear chain causes the system to ``lose its memory''
of the original orientation of earlier non-linear pieces.
\end{itemize}
The second of these provisos can be cured (as can many other
{\em practical} difficulties associated with the Z-matrix method)
by the introduction of the concept of the {\bf Dummy Centre}.

A dummy centre is just that; a centre introduced in order to
make the Z-matrix specification of a particular molecule either: 
\begin{itemize}
\item Possible, because it contains a linear sub-section or 
\item Easier, because some of the ``dihedral angles'' are awkward
and artificial. 
\end{itemize}
The dummy centre is specified in exactly the same way as ``genuine''
atomic centres and is recognised by {\tt gnu80} by its own ``chemical''
symbol X. It is ignored in the calculation once the co-ordinate system
has been set up.

The most characteristic use for a dummy centre is to resolve the
linear sub-sections problem. Simple by putting a dummy centre off the
linear axis, the dihedral angle between this dummy and subsequent off-axis
atoms can be used.  It is often found convenient to use dummy centres
to generate Z-matrices for particular purposes, for example to define
an angle in a molecule which is a ``natural'' parameter in
an optimisation, but which is not defined by particular atoms; angles
between the natural directions of functional groups or aromatic
planes.

The  the central concepts of the Z matrix
are made more useful by the addition of a few technical
aids.
\begin{itemize}
\item When giving the specification of an atom, the earlier atoms in
the Z matrix may be referenced {\em by number} as well as by label.
This facility is provided for compatibility with the earlier GAUSSIAN
series and its use is not reccommended; the insertion of an atom in a chain
will upset the numbering system but not reference by label.
\item A dummy atom may be labelled by a {\bf -} (minus or dash, not
underscore) as well as by {\bf X}.
\item Any of the distances and angles in the Z matrix may be immediately
preceeded by a minus with the obvious meaning. Clearly if the distance
or angle is given explicitly as a number (like -1.4) this is not news;
it is included in the definition of ``number''. But the use
of the minus is possible {\em even when the distance or angle is
given symbolically} ( like {\tt RC2C1}).

This is very often useful to preserve the {\em symmetry} of a molecular
arrangement during optimisations, or simply as an aid to getting
the geometry expressed in an easily-recognisable form. Examples 2 and 3
in the compilation of input examples (Appendix \ref{inex}) illustrates
this point.
\item It was remarked in the description of the dihedral angle
that, in some cases, the actual physical interpretation of this angle
is not that of a dihedral angle even when the geometrical recipe
still works. It is often more sensible to define the position of
an atomic centre by means of a distance and {\em two} bond angles.
This facility is provided in {\tt gnu80} and is described in the reference
section below.
\end{itemize}
Obviously, at some point the symbols used to define distances and angles
in the Z matrix must be given numerical values to enable {\tt gnu80}
to carry through an actual calculation. Either fixed numerical values
must be supplied or (in the case of optimisations) initial values
for {\tt gnu80} to use as a starting point for automatically generated
values.

This is done {\em after} the Z matrix is complete in the two input
sections immediately following the blank record which terminates
the Z matrix input section. The first of these sections is called the
{\bf Variables} section and is generally used to supply {\em initial
value} assignments to the symbols during an optimisation. Obviously,
if the {\bf OPT} command has not been given on the command record, no
optimisation will be performed and the assignments in the {\bf Variables}
section simply provide fixed values for the distances and angles.

The form of the {\bf Variables} input section is simply a set of
records of symbols followed by the initial numerical values of these
symbols, separated by either a space or, more pictorially, by an equals
symbol ({\bf =}). In the example given above, the {\bf Variables} section 
might include the records:
\begin{verbatim}
      RC2C1=1.34
      A 109.5
\end{verbatim}
among other records.

If it required to have all the Z matrix data in the same general form ---
symbolic names for the distances and angles --- and yet it is required
to keep some of these geometrical parameters {\em fixed} (i.e. not
take part in the optimisation) the the {\bf Variables} input section
is followed (after its terminating blank record) by the
{\bf Constants} input section which is of precisely the same form but
{\tt gnu80} does not attempt to change these numerical values, keeping them
fixed at the values given throughout the calculation.
\subsection{\sf The Z-matrix, Reference}
Each nucleus (including dummies) is numbered sequentially and
specified on a single record. 
This collection of data, determining the
molecular geometry is referred to as the Z-matrix.
Thus, the nature and location of the N'th nucleus is specified on the
(N+1)'th record in the section in terms of the positions of the previously
defined nuclei 1,2,\ldots (N-1).

The information about the position of the N'th 
nucleus is contained in up to eight
separate items on the record:
\begin{enumerate}
\item ELEMENT
\item N1
\item LENGTH
\item N2
\item ANGLE
\item N3
\item TWIST
\item J
\end{enumerate}
each of these items is now discussed.

\begin{enumerate}
\item {\bf ELEMENT} specifies the chemical nature of the nucleus. It may
consist of just the chemical symbol such as {\bf H} or {\bf C} 
for hydrogen or
carbon.  Alternatively, it may be an alphanumeric string beginning with
the chemical symbol, followed immediately by a secondary identifying
integer. Thus, {\bf C5} can be used to specify a carbon nucleus, identified
as the fifth carbon in the molecule, for example. 
This is sometimes convenient in
following conventional chemical numbering. 

Dummy nuclei are denoted by
the symbols {\bf X} or {\bf -}. The item {\bf ELEMENT} 
is required for every
nucleus. For the first nucleus specified (N=1), it is the only item on
the record.
\item {\bf N1} specifies the (previously defined) nucleus for which the
internuclear distance R(N,N1) will be given. This item may be either an
integer (the value of N1 < N) or an alphanumeric string. In the latter
case, the string must (of course!) match 
the {\bf ELEMENT} field of a previous Z-matrix
record.
\item {\bf LENGTH} is the internuclear distance R(N,N1). 
This may be either a
positive floating point number giving the length in Angstroms (unless
modified by the {\bf UNIT} command) or an alphanumeric string (maximum 8
characters). In the latter case, the length is represented by a
{\bf VARIABLE} for which a value will be specified in Input Section 4. 

Use
of variables in the Z-matrix is essential if optimization is to be
carried out. However, they can also be used in single-point runs. The
items {\bf N1} and {\bf LENGTH} are required for all nuclei after the first.
For the second nucleus, only {\bf ELEMENT}, {\bf N1}, and {\bf LENGTH} are
required.
\item {\bf N2} specifies the nucleus for which the internuclear angle
$\theta (N,N1,N2)$ will be given. Again this may be an integer (the value
of N2 < N) or an alphanumeric string which matches a previous {\bf ELEMENT}
entry. Note that {\bf N1} and {\bf N2} must represent {\em different} nuclei.
\item {\bf ANGLE} is the internuclear angle $\theta (N,N1,Nn2).$ 
This may be a
floating point number giving the angle in degrees (unless modified by
the {\bf UNIT} command) or an alphanumeric string representing a variable.
{\bf N2} and {\bf ANGLE} are required for all nuclei after the second. 
For the
third nucleus, only {\bf ELEMENT}, {\bf N1}, {\bf LENGTH},
{\bf N2}, and {\bf ANGLE} are
required.
\item {\bf N3} The significance of {\bf N3} and {\bf TWIST} 
depends on the value of
the last item {\bf J}. If J=0, or is omitted, {\bf N3} specifies the nucleus
for which the internuclear {\em dihedral} angle, $\phi (N,N1,N2,N3)$ will be
given; as with {\bf N1} and {\bf N2}, 
this may be either an  integer (N3<N)  or
an alphanumeric string matching a previous {\bf ELEMENT} entry.
\item {\bf TWIST} (if J=0) is the internuclear dihedral angle
$\phi (N,N1,N2,N3)$. Again, this may be a floating point number giving the
angle in degrees (unless modified by {\bf UNIT}) or an alphanumeric string
representing a variable (or a  variable  preceeded  by a negative
sign). The dihedral angle is defined as the angle 
$(-180.0 < \phi <=180.0)$ 
between the planes (N,N1,N2) and (N1,N2,N3). The sign is
positive if the movement of  the directed vector $R(N1 \rightarrow N)$
towards the
directed  vector $R(N2 \rightarrow N3)$ involves a righthanded screw motion.
\item {\bf J} The above descriptions of 
{\bf N3} and {\bf TWIST} apply if the  item
{\bf J} is zero or absent. 

Although it is usually possible to specify the
nucleus N by a bond length, a bond angle, and a dihedral angle,  it is
sometimes simpler to replace the dihedral angle by a second bond angle.
This possibility is called for by using J = +1 or -1.
\item Values of parameters for non-zero J:
\begin{itemize}
\item {\bf N3} If {\bf J} is +1 or -1, {\bf N3} specifies 
the nucleus for which the
{\em second} internuclear angle 
$\chi (N,N1,N3)$ will be given. As usual, this
may be either an integer (N3 < N) or an alphanumeric string
representing a previously defined {\bf ELEMENT}.
\item {\bf TWIST} If {\bf J } is +1 or -1, 
then this item gives the value for the
{\em second} internuclear angle 
$\chi (N,N1,N3)$. 

As before, this may be either a
floating point number (value in degrees, unless modified by {\bf UNITS} ) or
an alphanumeric string representing a variable.
\item {\bf J} In the event of specification by two internuclear angles
$\theta , \chi ,$ there will be two possible positions for the nucleus N.
This is fixed by the sign of {\bf J}. Thus J=+1 if the triple vector
product:
\[
R(N1 \rightarrow N) \cdot (R(N1 \rightarrow N2) \times R(N1 \rightarrow N3))
\]
is positive,  and J=-1 if the product is negative.
Note that J=+1 corresponds to a clockwise rotation angle when
looking from N1 to N.
\end{itemize}
\end{enumerate}
The Z-matrix is terminated by the blank record which indicates the
end of the molecule specification section. 

If no variables have been
introduced and if the command {\bf ALTER} has not been invoked in the
job-type section, the input is complete and {\tt gnu80} will perform
the requested computation.
\section{\sf Variable and Constant Input Sections}
These sections contain values of parameters introduced into the
Z-matrix as alphanumeric strings in the preceding molecule
specification section.  If there were no parameters (all lengths and
angles having been specified by floating-point numbers), these sections
are not required.  If parameters are used, they are divided into two
sets. 

The first set contains the {\bf VARIABLES} which are the parameters
which are varied and optimized in the optimization run. 
Input Section 4 contains the
{\em initial values} of the variables. If optimization is not called for,
these values will be inserted for a single run as if they had been
specified numerically in Input Section 3. 

The second set of parameters
contains {\bf CONSTANTS} which are parameters which are not varied in
optimization run. The values of these should be given in Input Section
5. The {\bf CONSTANT} section is provided only for occasional convenience; it
can be avoided completely if the values of the constants are entered
numerically in program Input Section 3.

Each parameter in Input Sections 4 and 5 is given a value on a
separate record, the value following the parameter name in free field.
Blanks, commas or equal signs may be used as separators with
perhaps equals signs being the most intuitively appealing. Each section is
terminated by a blank record. Thus, if the variables (R1,R2,A) are given
values (1.08, 1.36, 105.4), the full {\bf VARIABLE} section (input section 4)
consists of the four records:
\begin{center}
R1 = 1.08 \\
R2 = 1.36 \\
A  = 105.4 \\
(blank record) \\
\end{center}
If the {\bf CONSTANT} section is empty (i.e. if all of the parameters have
been listed in the variable section), a second blank record is not
needed.

If the job is a single point run (i.e. if the {\bf OPT} command has not
been issued in the Command Record), the program will substitute the
values from Input Sections 4 and 5 into the Z-matrix and proceed. 

If, on
the other hand, optimization {\bf (OPT)} has been requested, the program will
adjust the {\bf VARIABLES} (Input Section 4) but not the {\bf CONSTANTS} 
(Input Section 5) until a stationary point has been located.
\section{\sf GEN Basis Set Input}
This section is for specification of a basis set if a standard type
is not used. If the basis has already been given in 
the Command Record (or the {\bf GEN} command omitted)
this section is not needed and no blank record is required. For fuller
details, see Chapter \ref{chap3}.
\section{\sf Alteration of Configuration}
If the command {\bf GUESS=ALTER} or {\bf ALTER} has been issued in the
Command Record, information is expected in the
alteration-of-configuration program input sections. 

There will be one
such section if a singlet state is involved but two for higher
multiplicities since $\alpha-$ and $\beta-$ orbitals then have to be treated
separately.  Normally, the occupied orbitals are selected as those with
lowest eigenvalues for the one-electron Hamiltonian used in the initial
guess.  These sections consist of a set of substitutions, indicating
that one of these occupied orbitals is to be replaced by one of the
other (virtual) orbitals. Each such substitution is on a separate record
and has two integers N1, N2 (free field) indicating that orbital N1 is
replaced by obital N2 and that orbital N2 is replaced by N1. The list
of substitutions is terminated by the blank record at the end of the
section.

Note that both sections are required for non-singlet states. Thus,
if only $\alpha-$ substitutions are needed, the $\beta$ section is expected
even though it is empty and vice versa. The second blank record to
indicate this omission has to be included.
\section{\sf Program Limitations}
\label{limit}
This section outlines the various limitations
that exist within {\tt gnu80}.  These limitations occur throughout the
system in the form of fixed dimension statements, algorithm design
limitations, etc., and their overall effect is 
to restrict the ``size'' of
calculation that can be performed in the sense of number of
basis functions used or number of atoms in the molecule.

The limitations described here can be divided into two broad categories:
\begin{itemize}
\item Limitations associated with the Z-matrix specification 
\item Basis
set restrictions.  
\end{itemize}
This section is concluded by
``quick-reference''  Table which summarises the
pertinent information.
\subsection{\sf Z-matrix Limitations}
This class of restrictions involves: the size of the Z-matrix, the number
of variables in a geometry optimization and the maximum number of atoms.
Recall that a Z-matrix may have ``dummy'' atoms and 
therefore the number of
atoms used in the calculation may be fewer than the number of records in the
Z-matrix.  

The total number of records in the Z-matrix may not exceed 50.
This includes both dummies and genuine atoms.  

Furthermore, the maximum
number of variables which can be specified in an optimization must not exceed
50 (30 for a Fletcher-Powell ({\bf FP}) optimization).  
These two restrictions are imposed
by explicit DIMENSIONing which appears in the Links that process the Z-matrix
and its variables (links 101, 102, 103, 105, 202 and 716).

The internal restrictions on the total number of atoms are not so strict.
Within the interior of the program, the arrays which hold
Cartesian co-ordinantes and atomic number information are DIMENSIONed to
accommodate 100 atoms.  This DIMENSIONing far exceeds the maximum size
of the Z-matrix and certain integral evaluation limitations (described below).
These expanded arrays were installed in the program the last time it was
necessary to alter the COMMON-blocks containing them, and are intended for
future expansion.
\subsection{\sf Basis Set Limitations}
\label{basis_lims}
Throughout the Gaussian system, basis set limitations manifest themselves
in two ways.  Firstly, restrictions are imposed within the integral evaluation
programs; limiting the number of primitive gaussian functions and how they
are combined into atomic orbital basis functions.  Secondly, DIMENSIONing
requirements limit the total number of basis functions that can be used
in each of the energy evaluation procedures (SCF, MP2, etc.)

\subsubsection{\sf General input data limitations}
See \ref{genbasis} for a fuller discussion of non-default bases and
\ref{limit} for a compendium of the various limitations
on {\tt gnu80} and the way that basis set limitations fit into the
general scheme of limitations. Meanwhile, here is a summary
of the limitations on basis sets which is most useful for
the default bases; currently {\tt MAXSHL} is set to 100.

{\tt MAXSHL} shells;  see definition of a shell below.\\
Types: s, p, d, f, sp, spd  \\
Maximum of 6 Gaussian functions per shell  
but
maximum of {\tt 3*MAXSHL} Gaussians totally. 

Definitions of the concepts of {\em primitive} shells and {\em contracted} 
(or {\em full}) shells:
\begin{itemize}
\item A {\em primitive shell} is defined to be a set of basis functions up to
and including functions of some specified effective 
maximum angular quantum number ($\ell$) which
share a common orbital exponent.  Thus, an $\ell =1$ shell 
would consist of the
functions (s,px,py,pz) all with the same Gaussian exponent.  Similarly,
an $\ell =2$ shell would contain (s,px,py,pz,xx,yy,zz,xy,xz,yz) where xx,
etc.  denote the normalized second-order Gaussian functions.
\item A {\em full, or contracted} shell results from contracting the functions
of several primitive shells together.  In typical calculations, one
normally uses contracted shells.
\end{itemize}
As an example, consider the carbon atom in the 6-31G basis.  In
this basis, the carbon will have 10 primitive shells (6+3+1). The first
6 primitive shells are s-shells ($\ell =0$), and are contracted together to 
make a single s-type basis function. The next three shells are
sp-shells ($\ell =1$), each consisting of (s,px,py,pz) functions. These
primitive shells are contracted together to make four atomic orbital
basis functions: s, px, py and pz. The outermost primitive shell is
also of sp type, and makes yet another 4 atomic orbital basis
functions.

If the program were limited to this definition of shells, one would
have to use a set of sp-functions whenever a set of d-functions was
desired. Since it is frequently desired to have just a set of d or f
type functions, some way must be devised to handle this.
thus, we introduce the idea of a ` shell constraint' .  The shell
constraint specifies which functions within a shell are actually
employed. By appropriately setting the shell constraint, one can get
just the d portion of an $\ell =2$ shell.
\subsection{\sf Integral Evaluation Limitations}
In order to fully understand the limitations in the 
integral programs, one
must have some understanding of the concepts presented in \ref{genbasis}
(input of non-standard bases).  In the terminology introduced in
\ref{genbasis}, the limitations are as follows: 
\begin{itemize}
\item The maximum number of
primitive s- and p-type shells is {\tt 3*MAXSHL},
\item The maximum number of primitive
d-shells is {\tt MAXSHL}, 
\item The maximum number of contracted shells is {\tt MAXSHL}, and the
maximum degree of contraction is 8.  
\end{itemize}
Note:
\begin{itemize}
\item That the {\tt MAXSHL} contracted shell
limit implicitly declares an {\tt MAXSHL} atom limit if each atom has only a single
shell of basis functions.  
\item The maximum degree of contraction limit of
8 is actually an input limitation imposed by routine GBASIS in L301;
the integral evaluation programs are DIMENSIONed to handle expansions of
up to 10 gaussians.
\end{itemize}
An important limitation in the integral programs concerns the types of shells
which can be used.  The present version of  {\tt gnu80} can handle s,
sp, p and d shells.  Re-DIMENSIONing of the array TQ in L314 to 40000
will enable the program to handle spd-shells at the SCF level.  Implementation
of this feature in the gradient program is not straightforward.  Even though
much  of the code is present, f shells are not implementated in this version.

The last major restriction which appears in the integral programs is in the
manner in which integral labels are packed.  When the {\em standard} integral
storage procedure is selected (in contrast to the {\em Raffenetti} 
storage modes),
the suffixes I, J, K  and L of the two-electron integral (IJ,KL) are
packed into a 32 bit computer word as 
limited length quantities.  This in effect
limits the number of basis functions to 100.  When the Rafenetti modes
are selected, the two linearized suffixes IJ and KL (where e.g. 
IJ = (I*(I-1))/2 + J are packed into a standard integer.  
This imposes a theoretical
limit of 361 basis functions which is well outside any real possibility.
\newpage
\begin{center}
Basis function limitations in SCF
and post-SCF Links \\
Currently {\tt MAXBAS} is set to 150
\  \\
\begin{tabular}{|l|r|} \hline
Link & Max. Basis \\ \hline
L401 & {\tt MAXBAS} \\
\hline
L501 & {\tt MAXBAS} \\
L502 & 80 \\
L503 & 70 \\
L505 & 70  \\ 
\hline
L601 & {\tt MAXBAS} \\
\hline
L701 & {\tt MAXBAS} \\
L702 & {\tt MAXBAS} \\
L703 & {\tt MAXBAS} \\
L716 & {\tt MAXBAS} \\
\hline 
L801 & 70 \\
L802 & 70 \\
L803 & 70 \\
\hline
L901 & 70 \\
L909 & 60 \\
L910 & 60 \\
L911 & 60 \\
L912 & 60 \\
L913 & 60 \\ \hline
\end{tabular}
\end{center}
\newpage
\subsection{\sf SCF and POST-SCF Limitations}
The limitations present in the energy evaluation procedures are mainly
dependent on DIMENSIONing requirements.  These are summarized in the
Table.  These programs have only rarely been used at or near their
respective dimension limits, and the user should exercise some caution
when large calculations are attempted.
\subsection{\sf Basis Set Ranges}
The atoms for which the standard basis sets apply are: \\
{\bf 
\begin{center}
Basis Function Ranges \\
\  \\
\begin{tabular}{|l|r|} \hline
STO-3G &    H - Xe \\
3-21G  &    H - Ar \\
6-21G  &    H - Ar \\
4-31G  &   H - Ne \\
6-31G  &   H - Ne \\
6-311G &  H - Ne \\
LP-31G &    H - Cl \\
LANL1MB & Na - U \\
LANL1DZ & Na - U \\
\hline
\end{tabular}
\end{center}
} % end of bold
Note:
\begin{itemize}
\item
All these basis ranges {\bf exclude} the Lanthanides for which
there are no bases or potentials (yet).
\item
{\bf LANL1DZ} and {\bf LP-31G} are synonyms; use of either command
uses the {\bf LP-31G} basis for the first-row atoms (and related potentials)
and the {\bf LANL1DZ} basis and potentials for all other atoms.
\item
Use of {\bf LANL1MB} results in the use of the {\bf STO-3G} valence set
(and related potentials) for atoms of the first row and the actual
{\bf LANL1MB} basis for all other atoms.
\end{itemize}
These ranges are not affected by the addition of either p or d polarization
functions.
\begin{center}
\fbox{
\parbox{3.5in}{
But note that the addition of polarisation functions to an STO-3G
basis for atoms of the first row of the periodic table (H ---Ne)
is forbidden because it is an ``unbalanced'' basis which
may give poor or meaningless results. Thus specifying
STO-3G* for a calculation on the water molecule will result in
a warning message and an STO-3G calculation being performed.
}
}
\end{center}
\subsection{\sf Concluding Remarks}
The  sections above briefly describe the most limitations built
into the {\tt gnu80} system. The restrictions which are, by far, the
most likely to be encountered
are those associated with the size of the Z-matrix, number of geometrical
variables or the total number of basis functions.  In practice, the integral
program limitations almost never cause problems.

It is important to point out that underlying all of the above restrictions
is the question of practicality.  The amount of computation in the SCF step
increases very rapidly with the number of basis functions.  In the correlation
procedures, MP2, MP3 and CID, the computation is also strongly dependent
on the number of electrons.
