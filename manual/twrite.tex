\newpage
\setcounter{page}{1}
\markboth{\bf TWRITE}{\bf TWRITE}
\begin{description}
\item[NAME] TWRITE \\
LCAOMOSCF calculations
\item[SYNOPSIS] \ \\
{\tt
   subroutine TWRITE(FILE,X,M,N,MM,NN,K) \\
   double precision X(1) \\
   integer FILE, M, N, MM, NN, K \\
}
\item[DESCRIPTION] \ \\
TWRITE writes data to the gnu80 internal file number {\tt FILE}
from the double precision array {\tt X}. In the calling segment
X must be {\tt DIMENSION}ed {\tt MM} by {\tt NN}. The data written
is taken from this array as far as {\tt M} by {\tt N} consistent
with FORTRAN matrix storage rules. {\tt K} indicates whether or
not the matrix has been stored in a compressed mode (for symmetric
matrices) i.e. contains only M(M+1)/2 elements, not M**2.
The routine is called to write actual matrices to files {\em and}
to write COMMON blocks in which case usually {\tt N=NN=1}
and {\tt M=MM=} length of file (in units of double precision reals,
padded out if necessary)
\item[ARGUMENTS:] \ \\
\begin{description}
\item[FILE]  The gnu80 internal file number
\item[X] Array containing the data to be written.
\item[M] Actual number of rows in the matrix X.
\item[N] Actual number of columns in the matrix X.
\item[MM] Number of rows in the DIMENSION statement of calling program.
\item[NN] Number of columns in the DIMENSION statement of calling program.
\item[K] {\tt K = 0} means that all the matrix is write, {\tt K = 1}
means only ``half'' was write.
\end{description}
\item[SEE ALSO] \\ \\\\
TREAD, TQUERY, NTRAN, FILEIO
\item[DIAGNOSTICS]
None; but NTRAN tracks errors
\end{description}
