\documentstyle{book}
\begin{document}
\newpage
\setcounter{page}{1}
\markboth{\bf NTRAN}{\bf NTRAN}
\begin{description}
\item[NAME] NTRAN \\
Lowest level I/O routine.
\item[SYNOPSIS] \ \\
{\tt
   subroutine NTRAN(UNIT, OP, NWRDS, X, L) \\
   double precision X(1) \\
   integer UNIT, OP, NWRDS, L \\
}
\item[DESCRIPTION] \ \\
This routine is developed to handle the I/O requests of {\bf FILEIO} and
to perform all asynchronous I/O. (two electron storage).  It provides
gnu80 with a {\em word addressable} disc I/O capability.  It is called
by {\bf FILEIO} and the routines {\bf IREAD, IWRITE, IWIND} etc. 
which are ENTRY points to {\bf INTEGI}

\begin{center}
\fbox{
\parbox{3.5in}{
\bf NTRAN can nominally handle synchronous and asynchronous I/O requests.
In the gnu80 implementation all I/O is portable FORTRAN 77 and so
there is no asynchronous I/O but the code has been left in (as
comments) in case this facility is ever supported by future FORTRANs.
}
}
\end{center}

Synchronous I/O is random access (on 4 byte word level),
{\bf NTRAN} is designed to allow the FORTRAN programmer to store and
fetch arbitrary data from disc.  Arbitrary means that any 
number of 4 byte words may be transferred to or from any location
on a disc file.
Asynchronous I/O is basically sequential, and more restricted
than synchronous I/O, in the sense that only an {\em integer} number
of records can be written or read. This implies that 
the same number of words must be read back as were written out on that
record previously.
All operations or options are performed only on the specified
unit.
The synchronous I/O is performed with the use of a direct access unit.
At the moment this is fixed (through a {\tt DEFINE FILE}
statement). Because of this {\bf NTRAN} can only handle one direct access
unit ({\tt DEFINE}d to FORTRAN unit  18), but the code could be extended to
handle other direct access without much effort.  Currently gnu80
only uses one asynchronous unit (3) but {\bf NTRAN} is coded to handle 2
more (just define them through an appropriate {\bf NTRAN} call).

Usage notes:
\begin{enumerate}
\item  {\tt UNIT} must always be equal to one of the 
logical unit numbers stored
in the array {\tt UNITS}, even for calls which are not device specific.
\item  Don't rewind before a reposition. If necessary, 
{\bf NTRAN} will do this
automatically. To reposition {\bf NTRAN} will either: \\
Read records until the specified position is reached, and if
necessary write records to extend the file. \\
Backspace the unit until the specified location is reached \\
Rewind the unit and skip records to the specified location.
\item Only use {\tt OP}=23 (wait) if you want to use the  data read or
redefine the array used in a write operation. All other
waits are performed automatically if needed. Use the negative
form of the op parameter if you want to simulate synchro-
nous I/O on an otherwise asynchronous unit.
\item The correct sequence of operation is (synchronous) \\
a) define unit (26) \\
b) reopen unit if it is an old file (21) \\
c) reposition unit (6) \\
d) read/write (2,1) \\
e) go to c \\
asynchronous: \\
a) define unit (27) \\
b) rewind unit (10) \\
c) read or write \\
d) a rewind (10)  should be issued
between read and write operations
\end{enumerate}
\item[ARGUMENTS:] \ \\
\begin{description}
\item[UNIT]  the fortran logical unit number of the file to be
accessed.  {\tt UNIT} maps onto a local unit number
via the array {\tt UNITS}.  All calls to {\bf NTRAN}
require a valid unit.  Note that the value of {\tt IUNIT}
is used as a local event flag number to signal
I/O completion of asynchronous I/O.
\item[OP] specifies the operation to carry out. A {\em negative}
value indicates that {\bf NTRAN} is not to return control
to the calling program until the requested I/O
operation is complete. (only for 1 and 2)
\begin{description}
\item[+/- 1] write
\item[+/- 2] read
\item[6] reposition
\item[10] rewind
\item[21] reopen old unit, {\tt NWRDS} = number of records in old file.
\item[22] close unit (only for synchronous I/O)
\item[23] wait
\item[26] define synchronous  unit (always done first)
\item[27] define asynchronous unit (always done first)
\item[29] print switch, {\tt NWRDS}=0 =>off, {\tt NWRDS}$\not=$ 0 => on
\end{description}
\item[NWRDS] the number of words involved in the operation.
A radix (number of 8 bit bytes per user datum)
of 4 is assumed, thus nwrds specifies the number of
longwords to transfer.
\item[X] data to be transferred.
\item[L] status word: \\
-1 indicates operation in progress \\
-3 indicates end of file was encountered during 
synchronous read \\
-2 indicates error \\
$>0$ indicates successful completion of the request,
{\tt L} containing the number of words transferred. \\
Note that as currently coded {\bf NTRAN} will abort the job
if an error occurs, therefore L=-3 should not occur.
\end{description}
\item[SEE ALSO] \ \\
{\bf FILEIO}
\item[DIAGNOSTICS] \ \\
Provides error trapping {\em before} the FORTRAN diagnostics.
\end{description}
\end{document}
