\documentstyle{book}
\begin{document}
\subsubsection{Z matrix Concepts}
Before giving the method used by gnu80 to read the geometrical
information necessary for a job, it is useful to have an
overview of the so-called Z matrix method. Some of the
information in this outline is, of course, duplicated in the
reference section of the manual below.

There are, at least, three obvious methods to supply the
data specifying the relative positions of the atoms in a
molecule to a program:
\begin{enumerate}
\item Use a ``laboratory'' Cartesian co-ordinate
system and give all the atomic positions as absolute
Cartesian co-ordinates.
\item Still using a laboratory co-ordinate system, give the
molecular symmetry (say the point group symbol) and the absolute
Cartesian co-ordinates of the symmetry-distinct atoms in the molecule;
allowing the program to apply the molecular symmetry to generate the
full Cartesian co-ordinates for the entire molecule.
\item Use an ``internal'' co-ordinate system and specify
only the {\em relative} positions of the atoms and leave the program
to provide an origin and orientation of the laboratory system and do the
conversion to a Cartesian set.
\end{enumerate}
Obviously, in cases 1 and 3 the symmetry group of the molecule
may be {\em inferred} from the supplied co-ordinates.

The advantages and disadvantages of each approach are clear:
\begin{enumerate}
\item This choice is only really practical for small molecules or
for very symmetrical molecules.
\item This method is most convenient for highly symmetrical systems;
in most cases of low symmetry, it will degenerate into 1.
\item This is the most flexible approach, particularly since most
experimental molecular data are presented as bond distances, bond
angles etc. referred to a ``natural'' non-redundant
system of internal co-ordinates.
\end{enumerate}
However, when the question of the {\em automatic} generation
of changes in molecular geometry during a structure optimisation arises,
it is obvious that the use of the natural molecular parameters
of choice 3 above is the only viable option. The variation of
what are essentially arbitrary Cartesian co-ordinates is bound to be
inefficient and redundant at best. At worst, the use of three independent
co-ordinates for each atom in a molecule makes such automatic 
optimisations impossible.

That is, a description of molecular geometry in terms of the
essential molecular geometrical parameters (bond distances and angles)
is the only sensible way to both input data to a program and to
organise such data internally to allow automatic structure variations
which are likely to be chemically meaningful and numerically
efficient.

The method used by gnu80 (and by all the GAUSSIAN series and
many other {\em ab initio} programs) is the so-called
{\bf Z-matrix} method.

The technique is to ``walk '' through the molecule,
specifying the position of each atom as it is encountered by
its distance from and orientation to the other atoms already
encountered. In this way the natural geometric parameters used by
chemists and spectroscopists, can be used directly to generate 
an overall molecular structure which is unique but not (as in
the case of Cartesian co-ordinates) over-determined. The
overall orientation of the molecule with respect to the
laboratory co-ordinate system is not specified. This latter
freedom is used by gnu80 to position and orient the molecule
in a way which co-incides with the standard group-theoretical
choice of Cartesian axes if this is useful.

To completely specify an atom by this method, we need its atomic
number and enough geometrical information to place it uniquely
with respect to the other atoms in the system. 
\begin{enumerate}
\item As a matter of convience,
the atomic number is determined by the chemical symbol (H, LI, MN, etc.)
and so this is the way the atomic number is supplied.
\item The ``first'' atom in the molecule needs no geometrical
information, it is specified completely by its chemical symbol and all
other atoms' positions are referred (either directly or indirectly)
to it. As a convenience an atom' s label may have other characters
following the chemical symbol in order to make the data more legible to
humans; so, for example, an atom may be labelled C1 or LI4 etc.
Let us assume that we have labelled our first atom C1. gnu80
internally places this atom at the origin of the laboratory Cartesian
co-ordinate system. \\
The form of the input data (the line of the Z-matrix) is simply
\begin{verbatim}
      C1
\end{verbatim}
\item The next atom, which typically will be an atom formally bonded
to the first (i.e. a nearest neighbour) although, logically, it need not
be, has its nature and position completely specified by its chemical
symbol and its distance from the first atom; no angles are involved.
Let us assume that it is called C2. \\
The Z-matrix line of input consists of the label of the second atom,
the label of the atom to which its distance is to be given and that distance:
\begin{verbatim}
      C2 C1 RC2C1
\end{verbatim}
where {\tt RC2C1} is the distance between C1 and C2 and may be supplied
(in Angstroms) as a number like 1.4 or may be left as an alphanumeric
string whose numerical value is supplied later. Clearly, it is more
convenient to call it {\tt RC2C1} than {\tt LUCIFER}, for example,
but both are allowed.
The {\em direction} from the first to the second atoms is arbitrary and
is chosen by gnu80 as the Cartesian z axis.
\item The third atom (C3 say) needs at least one distance and an angle
to complete its specification. Since any triatomic is planar, gnu80
places the third atom in the xz plane of the Cartesian system, which
means that the first three atoms all lie in the xz plane.
As before, C3 is specified by the three items {\tt C3 C2 RC3C2}
if the distance between C3 and C2 is the most convenient to supply.
But this time the {\em angle} between the C3-C2 and C2-C1 bonds must be
given. This information is supplied by giving the label of the
third atom in the chain and the value of the angle (in degrees):
\begin{verbatim}
      C3 C2 RC3C2 C1 A
\end{verbatim}
where {\tt A} is the C3C2C1 angle.
\item The fourth and (all subsequent atoms) requires more information
to specify it uniquely since it may well be out of the plane defined
by the first three atoms. 

The most convenient way to do this (with one important exception, which
will be treated separately later) is to use a distance and a bond angle
as in the previous case and to specify the {\em ``out-of-plane
angle''} formed by the new bond. Traditionally, chemists have used
the Newman projection to define this {\em dihedral angle}.

If the four atoms in the chain are C1-C2-C3-C4, then, looking along
the C3-C2 bond (i.e. from C3 towards C2), the C3-C4 bond and the C2-C1
bond do not eclipse each other (in general). The {\em clockwise}
angle by which the C3-C4 bond must be rotated to bring it directly over the
C2-C1 bond is the required dihedral angle (in degrees). 
The specification of the fourth (and subsequent) atoms is then given
by:
\begin{verbatim}
      C4 C3 RC4C3 C2 A2 C1 D
\end{verbatim}
where {\tt RC4C3} is the bond distance from {\tt C4} to {\tt C3},
{\tt A2} is the angle {\tt C4C3C2} and {\tt D}
is the dihedral angle.

The simplest case is
perhaps the hydrogen peroxide molecule, $H_2 O_2$. The dihedral
angle is the angle between the two O-H bonds when viewed along the O-O
bond. In this simplest case the dihedral angle may be taken as
$\alpha$ or $360 - \alpha$ since there are no other atoms involved.
In general, however, the {\em sense} of the angle must be preserved to
ensure a correct orientation of subsequent atoms (either on the chain
or branched)
\end{enumerate}
This scheme is sufficient to specify completely the geometry
of molecules of arbitrary complexity with the following provisos:
\begin{itemize}
\item When a molecule is not basically a chain but is more
``compact'' i.e. a central branched system like $CH_3F$,
for example, the ``dihedral angles'' can still be defined by
the above procedure and are acceptable to gnu80 but they are not the
sort of angle which the chemist normally calls dihedral angles. This
is a small price to pay for a coherent system.
\item Linear Molecules have dihedral angles zero. This is not a problem
for actual linear molecules but it {\em is} a problem for larger
molecules with {\em linear sections}. In these cases, passing along 
a linear chain causes the system to ``lose its memory''
of the original orientation of earlier non-linear pieces.
\end{itemize}
The second of these provisos can be cured (as can many other
{\em practical} difficulties associated with the Z-matrix method)
by the introduction of the concept of the {\bf Dummy Centre}.

A dummy centre is just that; a centre introduced in order to
make the Z-matrix specification of a particular molecule either: 
\begin{itemize}
\item Possible, because it contains a linear sub-section or 
\item Easier, because some of the ``dihedral angles'' are awkward
and artificial. 
\end{itemize}
The dummy centre is specified in exactly the same way as ``genuine''
atomic centres and is recognised by gnu80 by its own ``chemical''
symbol X. It is ignored in the calculation once the co-ordinate system
has been set up.

The most characteristic use for a dummy centre is to resolve the
linear sub-sections problem. Simple by putting a dummy centre off the
linear axis, the dihedral angle between this dummy and subsequent off-axis
atoms can be used.  It is often found convenient to use dummy centres
to generate Z-matrices for particular purposes, for example to define
an angle in a molecule which is a ``natural'' parameter in
an optimisation, but which is not defined by particular atoms; angles
between the natural directions of functional groups or aromatic
planes.
\subsubsection{Z matrix Facilities}
The  the central concepts of the Z matrix
are made more useful by the addition of a few technical
aids.
\begin{itemize}
\item When giving the specification of an atom, the earlier atoms in
the Z matrix may be referenced {\em by number} as well as by label.
This facility is provided for compatibility with the earlier GAUSSIAN
series and its use is not reccommended; the insertion of an atom in a chain
will upset the numbering system but not reference by label.
\item A dummy atom may be labelled by a {\bf -} (minus or dash, not
underscore) as well as by {\bf X}.
\item Any of the distances and angles in the Z matrix may be immediately
preceeded by a minus with the obvious meaning. Clearly if the distance
or angle is given explicitly as a number (like -1.4) this is not news;
it is included in the definition of ``number''. But the use
of the minus is possible {\em even when the distance or angle is
given symbolically} ( like {\tt RC2C1}).

This is very often useful to preserve the {\em symmetry} of a molecular
arrangement during optimisations, or simply as an aid to getting
the geometry expressed in an easily-recognisable form. Examples 2 and 3
in the compilation of input examples (Appendix \ref{inex}) illustrates
this point.
\item It was remarked in the description of the dihedral angle
that, in some cases, the actual physical interpretation of this angle
is not that of a dihedral angle even when the geometrical recipe
still works. It is often more sensible to define the position of
an atomic centre by means of a distance and {\em two} bond angles.
This facility is provided in gnu80 and is described in the reference
section below.
\end{itemize}
Obviously, at some point the symbols used to define distances and angles
in the Z matrix must be given numerical values to enable gnu80
to carry through an actual calculation. Either fixed numerical values
must be supplied or (in the case of optimisations) initial values
for gnu80 to use as a starting point for automatically generated
values.

This is done {\em after} the Z matrix is complete in the two input
sections immediately following the blank record which terminates
the Z matrix input section. The first of these sections is called the
{\bf Variables} section and is generally used to supply {\em initial
value} assignments to the symbols during an optimisation. Obviously,
if the {\bf OPT} command has not been given on the command record, no
optimisation will be performed and the assignments in the {\bf Variables}
section simply provide fixed values for the distances and angles.

The form of the {\bf Variables} input section is simply a set of
records of symbols followed by the initial numerical values of these
symbols, separated by either a space or, more pictorially, by an equals
symbol ({\bf =}). In the example given above, the {\bf Variables} section 
might include the records:
\begin{verbatim}
      RC2C1=1.34
      A 109.5
\end{verbatim}
among other records.

If it required to have all the Z matrix data in the same general form ---
symbolic names for the distances and angles --- and yet it is required
to keep some of these geometrical parameters {\em fixed} (i.e. not
take part in the optimisation) the the {\bf Variables} input section
is followed (after its terminating blank record) by the
{\bf Constants} input section which is of precisely the same form but
gnu80 does not attempt to change these numerical values, keeping them
fixed at the values given throughout the calculation.
\end{document}
