\documentstyle{book}
\begin{document}
\section{Parsing the Command Record}
In order to be able to make significant changes to gnu80 (adding new
links and the possibility of new Routes) it is necessary to
understand how the Route is generated from the command record.
The command record must be {\em parsed} for the meaning of the
symbols it contains, checked for errors and meaningless symbols
and finally the correct command translated into a valid Route.

These tasks are performed by the {\tt INTEGER FUNCTION QPARSE} and its
associated data structure a {\em Parse Table}. The information
in the following sections, together with a study of the use of
{\tt QPARSE} in gnu80, should enable the user to master the
parsing of the command record.
\subsection{{\tt QPARSE} use}
{\tt QPARSE} is a table-driven line parser. It is called with a 
{\em string of characters} to
be parsed, a parse table, and a {\em Result Vector.} 
The input string is just
a string of characters which, in line with the design
of gnu80, is contained in an INTEGER array.  
The parse table tells the FUNCTION which sub-strings are meaningful
in the input string (a number, a certain keyword,
etc.), and what to do if a specified meaningful sub-string 
is found in the input string.

The {\tt RESULT} Vector contains the output from {\tt QPARSE}. 
The specification of {\tt QPARSE} is:
\begin{verbatim}
      INTEGER FUNCTION QPARSE(RESULT,TABLE,LINE,LENGTH)
      INTEGER RESULT(1), TABLE(1), LINE(1), LENGTH
\end{verbatim}
\begin{description}
\item[{\tt LINE}] contains the (ASCII) string to be parsed
stored four characters to a (default) integer.
\item[{\tt TABLE}] is a DATA structure containing the information necessary to
parse the string in {\tt LINE}
This structure is discussed in detail below.
\item[{\tt RESULT}] is an INTEGER array in which the results of the parse
are stored. The description of the {\tt TABLE} structure
gives the meaning of the {\tt RESULT} array.
\item[{\tt LENGTH}] The number of {\em characters} in the string stored
in {\tt LINE}.
\item[{\tt QPARSE}] The {\tt INTEGER} value returned by
{\tt QPARSE} gives the status of the
parse. The possibilities are:
\begin{description}
\item[{\tt 0}] Success; a transition has occured to {\tt EXI}
\item[{\tt -1}]  Failure; a transition has occured to {\tt FAI}. This
indicates some syntax error in the input string.
\item[{\tt -2}]    Failure: an ambigous keyword was detected in the
input string.
\item[{\tt -3}]  Error; an error was detected in the parse table.
\item[{\tt 1}]      Return; the parse is not yet completed, but
control has returned to the caller because of a
parse-table request.
\end{description}
\end{description}
Before any calls to {\tt QPARSE} are made,
the routine must be initialized by a call to
{\tt QPINIT} which sets certain options to {\tt QPARSE}.
\begin{verbatim}
      SUBROUTINE QPINIT(TABLE,BLNKS,CAPS,ABVRS)
      INTEGER TABLE(1), BLNKS, CAPS, ABVRS
\end{verbatim}
\begin{description}
\item[{\tt TABLE}] the parse table used as input to {\tt QPARSE}. The
remaining arguments set options in the parser. 
\item[{\tt BLNKS}]   By default, 
any number of (contiguous) blanks and tabs in the input string are treated
as invisible delimiters.
A non-zero value for {\tt BLNKS} means that
blanks will be seen by {\tt QPARSE}.
\item[{\tt CAPS}]     By default, the difference between upper and lower case
letters is ignored. A non-zero value of {\tt CAPS} indicates that
only exact matches are acceptable. 
\item[{\tt ABRVS}] By default, any unambiguous abbreviation is allowed for
a keyword. A non-zero value of {\tt ABRVS} makes abbreviations
non-acceptable.
\end{description}
The following material describes how to construct a
parse table with some examples.
\section{The {\tt QPARSE} Table}
The {\tt QPARSE} table, which descibes how the input record is to be
parsed, can be built with FORTRAN {\tt DATA} statements. 

The table consists
of one or more {\em states}, and the general structure of the table is
illustrated below:
\begin{center}
% \fbox{
% \parbox{3in}{
\begin{verbatim}
<name-of-state>             (1 default integer)
<transition definition>     (variable length)
<transition definition>         "       "
          "
          "
          "
<end-of-state>              (1 default integer)
<name-of-state>             (1 word)
<transition definition>
          "
          "
          "
<end-of-state>
<end-of-table>              (1 default integer)
\end{verbatim}
% }
% }
\end{center}
Thus, the table consists of one or more states, and each state
consists of one or more {\em transition definitions}. Generally, the states
provide the overall control structure for the parse, and the transition
definitions provide the details of what to look for in the input record,
and what to do if it's found.

The $<$name-of-state$>$ field is just a string of up to four
characters which labels the state. The parser has three pre-defined
state names which should not be used as $<$name-of-state$>$ fields in the
table. 
These are {\tt RET}, {\tt EXI} and {\tt FAI}. The use and importance of
these special state names are descibed below in the detailed
description of the transition definition. The $<$name-of-state$>$ field may
be left zero in some cases, but one default integer must always be
allocated in the table.

The $<$transition definition$>$ field is of variable length and is
discussed in detail below. Generally, a transition definition tells the
parser what kind of object to look for in the input record (keyword,
character, integer, etc.), and what to do if it's found. If the object
specified by the transition definition matches the next character(s) in
the input record, then the character(s) from the input record are accepted,
and a state transition occurs. This transition may be to a new state,
or just back to the top of the transition definition. Other subfields
of the transition definition can specify one of a set of simple
operations which can be performed during the transition:
\begin{itemize}
\item Store the object accepted from the input record into a
specified storage location.
\item Add a given value into a specified storage location.
\end{itemize}
The $<$end-of-state$>$ mark is just aa integer with a particular value
({\tt EOS}) which terminates the state. If none of the transition
definitions in the current state succeeds, then this mark will be
encountered and a transistion will occur to the special state {\tt FAI} and
the parse fails. In this case, a failure status will be returned by
{\tt QPARSE}.

The $<$end-of-table$>$ mark is just a $<$name-of-state$>$ field containing
the value {\t END}.
\subsection{Transition Definitions}
A {\em transition definition} consists of a number of subfields, one of
which is of variable length:

$<$alphabet token$>$, $<$destination$>$,$<$index$>$,$<$value$>$

Each of these subfields is discussed individually below.
\begin{description}
\item[$<$token$>$] This  field describes what to look for in the input record. 
If
this token matches the next data in the input record, then the data in
the record is {\em accepted}, and the transition succeeds. 
For instance, one 
can ask:
\begin{quote}
``if the next characters in the record form a decimal integer,
then accept them and make the transition.'' 
\end{quote}
Characters which are
accepted are removed from the front of the input record, and subsequent
transition definitions will try to match whatever characters come after
these. If the transition succeeds, then the rest of the transition
definition is used to determine the details of what actions are to be
taken. If there is no match, then the transition fails, and the next
transition definition in the current state is checked against the same
characters. If no transition definitions remain, then a transition to
the state {\tt FAI} occurs, and the parse fails. The various tokens, some
of which are longer than one integer, are detailed below in the
section on the {\tt QPARSE} alphabet.
\item[$<$destination$>$] This field is a string of characters which names the
state to which the transition is to be made. If this field is zero,
then a transition is made to the state immediately following the
current state. Otherwise, this string should match either one of the
specified state names, or one of the $<$name-of-state$>$ fields in the
table. The special states, and the effect of a transition to each is
detailed below:
\begin{description}
\item[{\tt EXI}]        Causes the parser to halt, and to return an 
zero
status (return value of {\tt QPARSE}). 
A transition to {\tt EXI} indicates succesfull
completion of the parse.
\item[{\tt FAI}]        causes the parser to halt, and return a failure
status. A transition to {\tt FAI} indicates a syntax
error in the input record.
\item[{\tt RET}]        causes a transition to the state immediately
following the current one in the table.
However, control is returned to the calling routine
before the first transition definition in this new
state is examined. This is useful when the calling
routine needs to perform some other action before
the parse can continue.
\end{description}
\item[ $<$index$>$ and $<$value$>$] These fields are used to request some simple
operations which can be performed during the transition. 
If both of
these values are non-zero, then  $<$value$>$  is added into
{\tt RESULT($<$index$>$)}. 
{\tt RESULT} is one of the calling arguments to the parser.
If only the $<$index$>$ field is non-zero, then the object accepted from
the input record (character, integer, keyword, etc.) is stored into
{\tt RESULT($<$index$>$)}. If the $<$index$>$ field is zero, then nothing is done
during the transition.
\end{description}
\subsection{{\tt QPARSE} Alphabet}
\begin{enumerate}
\item  Tokens which match single characters. \\
If it is requested that
this token be stored into {\tt RESULT} (a zero value for $<$value$>$), then this
character is justified in the indicated word as {\tt A1}.
\begin{description}
\item[{\tt NUM}]    matches any numeric character.
\item[{\tt ALP}]    matches any alphabetic character.
\item[{\tt ALN}]    matches any alphanumeric character.
\item[{\tt CHR}]    matches any character.
\end{description}
\item Tokens which match numbers. If it is requested that one of these be
stored in {\tt RESULT}, the corresponding numeric value is stored there.
Note the difference, then, between the tokens {\tt NUM} and {\tt D10} 
(below).
Each of these matches one numeric character in the input record, but
there is a difference if the object accepted is to be stored into
{\tt RESULT}. {\tt NUM} causes the {\em character} to be stored there, while 
{\tt D10}
causes the corresponding integer value to be stored. The following
tokens are recognised:
\begin{description}
\item[{\tt D10}]    matches a base 10 digit.
\item[{\tt D16}]    matches a hexadecimal digit.
\item[{\tt D8}]     matches an octal digit.
\item[{\tt D2}]     matches a binary digit.
\item[{\tt I10}]    matches a decimal integer. this must be terminated
in the input record by a non-alphanumeric character (or
by the end of the input record).
\item[{\tt I16}]    matches a hexadecimal integer.
\item[{\tt D8}]     matches an octal integer.
\item[{\tt I2}]     matches a binary integer.
\item[{\tt FP}]     matches a floating point number.
\item[{\tt DP}]     matches a double precision floating point number.
the only difference between this and {\tt FP} is that this
takes two locations in {\tt RESULT}.
(starting with {\tt RESULT($<$index$>$)}.
\end{description}
\item  Tokens which match strings. If one of these is to be stored
into {\tt RESULT}, then a variable number of words in {\tt RESULT} 
will be used,
depending unpon the length of the string. The integer actually indicated
by $<$index$>$ will contain a count of characters in the string, and
subsequent words will contain the ASCII string. The following tokens
are recognised:
\begin{description}
\item[{\tt N,$<$keyword$>$}] This token matches a keyword. 
{\tt N} is a positive
integer which indicates the number of characters in
the keyword which follows. For instance, the token
describing the keyword ``HELLO''  takes three integers;
\begin{center}
5, {\tt 'HELL','O'}
\end{center}
By default, any abbreviation is accepted for a
keyword, as long as it is unambiguous. A keyword is
defined as an alphanumeric string terminated by a
non-alphanumeric character. Thus, if the characters
'HELLOAB' were at the front of the input record,  the
above token ('HELLO') would not match, since the
string 'HELLO' in the input record is not terminated by
a {\em non-alphanumeric} character.
\item[{\tt -N,$<$string$>$}]    This variable-length token matches a specific
string of characters.  {\tt -N} means that the
character(s) to be matched follow in the next
{\em integer(s)}. For instance, the token:
\begin{center}
-5,'HELL','O'
\end{center}
matches the string 'HELLO'. note the difference
between the token for the string 'HELLO' and that for
the keyword 'HELLO'. If the input record contains the
characters 'HELLOAB', the string 'HELLO'
(-5,'HELL','O') will match. The keyword 'HELLO'
(5,'HELL','O') will not match, however, since it is
not terminated by a non-alphanumeric character in the
input record.
\item[{\tt WRD}]  This token matches any alphanumeric string
(terminated by a non-alphanumeric character).
\item[{\tt STR,N,$<$delims$>$}] Matches a user-defined string. This is similar
to {\tt WRD}, except that the characters which delimit
the desired string are supplied in the token. {\tt N} is
the number of such delimiters, and the delimiting
characters are supplied in the subsequent words.
\end{description}
\item Tokens which remove no characters from the input string.
The following are recognised:
\begin{description}
\item[{\tt NUL}]     Always matches. This results in no characters being
removed from the front of the input record, but
the supplied transition occurs. It is useful as a
``go to'' -type command.
\item[{\tt EOL}]         Matches end-of-line. 
If there are no more characters
in the input record, then this token matches.
\end{description}
\end{enumerate}
\end{document}
