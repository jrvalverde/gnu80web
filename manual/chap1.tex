\chapter{\sf Introduction} 
\label{chap1}
{\tt gnu80} is a set of programs to perform the most commonly-required
tasks of quantum chemistry.
The numerical work involved in performing these tasks falls naturally into
modules and the programs of {\tt gnu80} reflect this 
underlying modularity.

The basic assumption involved in all the programs is the
``Expansion Method'' or ``Algebraic Approximation'' ;
an approximate wavefunction for a molecular system (molecule, ion,
radical, group of molecules) may be built up from
(anti-symmetrised) products of Linear Combinations of Basis Functions.
The basis functions themselves are linear combinations of Gaussian
Functions. 
The approximate Molecular Hamiltonian embodies the non-relativistic
Born-Oppenheimer model; the so-called elecetrostatic model, the
only forces acting between the particles are those due to
Coulomb's law.

With these two basic assumptions all calculations of molecular 
electronic structure fall into three stages:
\begin{enumerate}
\item The calculation of the energy integrals involving the
one- and two-particle operators in the Hamiltonian and the basis
functions; kinetic energy and nuclear attraction (``one-electron'' 
integrals) and the far more numerous electron-electron repulsion
integrals (``two-electron'' integrals).
\item Use of these energy integrals and matrix techniques to 
solve the algebraic representation of the equations of
the quantum mechanical model used.
\item Analysis of the results of the calculation
\end{enumerate}
{\tt gnu80} provides implementations of these procedures with the 
provision of several different quantum mechanical models
of the molecular electronic structure. 
Control programs are also provided for the automatic optimisation of
molecular geometries by seeking turning points in the total energy
of the molecule with respect to changes in those geometries.

There are two main areas of research in which {\tt gnu80} may be useful:
\begin{enumerate}
\item {\em Chemical research using quantum chemistry tools;}
the calculation of the structure and properties of transient or
otherwise experimentally inaccessible species to help in the
resolution of chemical problems, elucidation of reaction
paths etc.
\item {\em Quantum chemistry research;} calculations which are aimed at
an understanding of the electronic structure of molecules and
exploring the utility and limitations of the 
quantum mechanical models themselves.
\end{enumerate}
{\tt gnu80} provides two distinct but related methods of working
which, broadly speaking, are suitable for the two main uses of the system:
\begin{enumerate}
\item Use of Standard Methods
\item General Use
\end{enumerate}
Both of these approaches must, of course, use the same
capabilities of {\tt gnu80} and share the same ultimate limitations;
types of quantum mechanical model and classes of Gaussian function available.
Within these restrictions outlined earlier (electrostatic
Hamiltonian and Gaussian Algebraic Approximation) the quantum
mechanical models available are:
\begin{itemize}
\item ``Hartree-Fock'' , this term has come to mean a
whole class of models related by the general idea of a one-term
(single-determinant) approximation; the best single determinant (UHF),
the best single determinant of doubly-occupied orbitals (RHF),
the best single determinant of doubly- and singly-occupied orbitals
for open-shell species (ROHF). 
\item Variational Methods including electron correlation; 
Configuration Interaction  of double excitations from a Hartree-Fock
determinant.
\item Perturbation Methods of including electron correlation;
Moller-Plesset perturbation theory to second and third order using
the Hartree-Fock determinant as zero-order function.
\end{itemize}
When a quantum mechanical model has been chosen, considerations
of the accuracy required and the available computational resources
must be made in order to make a choice of the {\em numerical}
limiations of the calculation.
This choice dictates the number and type of Gaussian functions to be used.
{\tt gnu80} has libraries of standard Gaussian basis functions
(particularly useful for Standard Methods) and the capability
of accepting basis functions of the user's choice.

Thus the combination of the quantum mechanical {\em model}
and the {\em numerical} choice of basis are primary items of
data to {\tt gnu80} which are independent of the particular 
molecular system under investigation. This information must
be given in both Standard Methods and General Use; Chapter
\ref{chap2} describes the use of Standard Methods and
Chapter \ref{chap3} outlines General Use.
