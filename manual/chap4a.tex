\chapter{\sf {\tt gnu80} COMMON BLOCKS}
\section{\sf COMMON /MOL/}
{\small
\begin{verbatim}
      INTEGER NATOMS, ICHARG, MULTIP, NAE, NBE, NE, 
     & NBASIS, IAN
      DOUBLE PRECISION ATMCHG. C
      COMMON/mol/ NATOMS,ICHARG,MULTIP,NAE,NBE,NE,NBASIS,
     & IAN(101),ATMCHG(100),C(300)
\end{verbatim}
}
Here is the basic information concerning the molecule for which
the energy calculation is being carried out.
\begin{description}
\item[NATOMS]  the number of atoms in the molecule.
\item[ICHARG]  the total electric charge on the molecule,
for neutral, 1 for cations, etc.
\item[MULTIP]  the molecule's spin multiplicity (1 for
singlets, 2 for doublets, etc.)
\item[NAE]  the number of electrons of $\alpha$ spin.
\item[NBE]  the number of electrons of $\beta$  spin.
\item[NE]  the total number of electrons (= NAE + NBE).
\item[NBASIS]  the number of basis functions.
\item[IAN(I)]   the atomic number of atom I.  This array is
dimensioned to 101 so that the number
of double-words in the COMMON block is even. \\
Note that this is strictly the Atomic Number, not the nuclear
charge which may be different due to the use of Effective
Potentials for the inner shells.
\item[ATMCHG(I)] the nuclear charge of atom I. In non-effective
potential runs this is the same as the atomic number IAN(I).
\item[C(I)]    the x, y, and z, coordinates of the atoms in
atomic units.  The coordinates are stored in
the order x1, y1, z1, x2, y2, z2, x3, etc.
\end{description}
\newpage
\section{\sf COMMONs {\tt /ZMAT/ and /ZSUBST/}}
\subsubsection{\sf {\tt /ZMAT/}}
\begin{verbatim}
     INTEGER IANZ, IZ, LBL, LALPHA, LBETA, NZ, NVAR
     DOUBLE PRECISION BL, ALPHA, BETA 
     COMMON /ZMAT/   IANZ(50),IZ(50,4),BL(50),ALPHA(50),BETA(50),
     $                LBL(50),LALPHA(50),LBETA(50),NZ,NVAR
\end{verbatim}
{\tt /ZMAT/} consists of eight arrays dimensioned to hold the
data from up to 50 Z-matrix records and two
variables containing counters.
\begin{description}
\item[{\tt IANZ}]   atomic numbers
\item[{\tt IZ}]     connectivity data
\item[{\tt BL}]     bond lengths
\item[{\tt ALPHA}]  valence angles
\item[{\tt BETA}]   dihedral or second valence angle
\item[{\tt LBL}]    an array used to map from the array of
variable values in {\tt /ZSUBST/} to the array
{\tt BL} in /ZMAT/.   
\begin{itemize}
\item For {\tt LBL(N)}=0, no
substitution is required to get {\tt BL(N)}:
it already contains the proper value.
\item For {\tt LBL(N) = I}; {\tt BL(N)} = {\tt VALUES(I)}.
\item  For
{\tt LBL(N) = -I}; {\tt BL(N) = -VALUES(I)}.
\end{itemize}
\item[{\tt LALPHA}]  analagous to {\tt LBL} for {\tt ALPHA}.
\item[{\tt LBETA}]   analagous to {\tt LBL} for {\tt BETA}.
\item[{\tt NZ}]      the number of records in the Z-matrix.
\item[{\tt NVAR}]    the number of variables, equal to the
number of symbols defined in the
VARIABLES section of the input.
\end{description}
\subsubsection{\sf {\tt /ZSUBST/}}
\begin{verbatim}
      INTEGER  INTVEC
      DOUBLE PRECISION VALUES, FPVEC, ANAMES
      COMMON /ZSUBST/ ANAMES(50),VALUES(50),INTVEC(50),FPVEC(50)
\end{verbatim}
{\tt /ZSUBST/} contains the data in the VARIABLES section of
the input.
\begin{description}
\item[{\tt ANAMES}]  alphanumeric names of the variables.
(limited to 8 characters in the current
version).
\item[{\tt VALUES}]  the corresponding numeric values.  These
will be altered in the course of
geometry optimizations and potential
surface scans.
\item[{\tt INTVEC}]  an array of the integer values following
the symbol and value on a line in the
VARIABLES section.
\item[{\tt FPVEC}]   a corresponding array containing the
floating point number.  The use of these
two arrays depends upon the route.
\end{description}
\newpage
\section{\sf COMMON /B/}
/B/ is a common block which contains the necessary arrays
to define and describe the basis set.  
{\small
\begin{verbatim}
      INTEGER SHELLA,SHELLN,SHELLT,SHELLC,SHLADF,AOS,AON
      DIMENSION SHLADF(80), C4(80)
      COMMON/B/EXX(240),C1(240),C2(240),C3(240),X(80),Y(80),Z(80),
     $         JAN(80),SHELLA(80),SHELLN(80),SHELLT(80),SHELLC(80),
     $         AOS(80),AON(80),NSHELL,MAXTYP
      EQUIVALENCE(SHLADF(1),C3(161)),(C4(1),C3(81))
\end{verbatim}
}
This common block is organized in such a manner so as to
facilitate the calculation of integrals over Gaussian
functions.

Before  describing the various arrays,
it will be useful to define the concepts of primitive shells
and contracted (or full) shells.

A {\em primitive} shell is defined to be a set of basis functions
up to and including functions of some maximum angular
quantum number which share a common exponent.  thus,
an $\ell = 1 $ shell would consist of the functions (s,px,py,pz)
all with the same Gaussian exponent.  Similarly, an $\ell = 2$
shell would contain (s,px,py,pz,xx,yy,zz,xy,xz,yz) where
xx, etc. denote the normalized second-order Gaussian
functions.

A {\em full, or contracted} shell results from contracting
the functions of several primitive shells together.
in typical calculations, one normally uses contracted
shells.

As an example, consider the carbon atom in the 6-31G basis.
in this basis, the carbon will have 10 primitive shells
(6+3+1).  The first 6 primitive shells are s-shells $(\ell = 1)$,
and are contracted together to make a single s-type basis
function.  The next three shells are sp-shells $(\ell = 1)$,
each consisting of (s,px,py,pz) functions.  These primitive
shells are contracted together to make four atomic orbital
basis functions: s, px, py and pz.  The outer-most primitive
shell is also of sp type, and makes yet another 4 atomic
orbital basis functions.

If the program were limited to this definition of shells,
one would have to use a set of sp-functions whenever
a set of d-functions was desired.  Since it is frequently
desired to have just a set of d or f type functions,
some way must be devised to handle this.

Thus,  the idea of a ` shell constraint' is introduced.
the shell constraint specifies which functions within
a shell are actually employed.  By appropriately setting
the shell constraint, one can use only the d portion
of an $\ell = 2$ shell.

This general information enables the
arrays which appear in {\tt /B/} to be described.
\newpage
\begin{description}
\item[EXX] contains the Gaussian exponents for all the
 the primitive shells.  The array {\tt SHELLA}
 contains pointers into {\tt EXX} for the various
 primitive shells.
\item[C1] contains the s coefficients for all the primitive
 shells;  indexed by {\tt SHELLA}.
\item[C2] contains the p coefficients for the primitive
 shells;  indexed by {\tt SHELLA}.
\item[C3] this array really consists of 3 sub-arrays
 (each 80  long) which contain the
 d and f coefficients plus the d and f
 pointer table for d and f type shells.
 these arrays are {\tt C3, C4} and {\tt SHLADF}, respectively.
\item[X] the x-Cartesian coordinate for each of the
 primitive shells.
\item[Y] the y-Cartesian coordinate for each of the
 primitive shells.
\item[Z] the z-Cartesian coordinate for each of the
 primitive shells.
\item[JAN] **reserved for future expansion**
\item[SHELLA] {\tt SHELLA(I)} contains the starting location
 within {\tt (EXX,C1,C2)} of the data
 (exponents, s-coefficients, p-coefficients)
 for the I'th primitive shell.
\item[SHELLN] {\tt SHELLN(I)} contains the number of primitive
 Gaussians in the I'th primitive shell.
\item[SHELLT] {\tt SHELLT(I)} contains the maximum angular quantum number
 of the I'th shell.
\item[SHELLC] {\tt SHELLC(I)} contains the shell constraint for the I'th
 shell (see table below).
\item[AOS] {\tt AOS(I)} gives the starting atomic orbital basis
 function number (ie number within the list
 of atomic orbital basis functions) of the
 I'th shell.  note that {\tt AOS} is always filled
 as though the shell contained all possible
 lower angular momentum functions. \\  
See {\tt SUBROUTINE GBASIS} for details on how {\tt AOS} is filled.
\item[AON] **reserved for future expansion**
\item[NSHELL] the number of primitive shells.
\item[MAXTYP] the highest angular quantum number present
 in the basis.
\end{description}
The following table summarizes the relationship between
SHELLT AND SHELLC.
\begin{verbatim}
=========================================
TYPE   FUNCTIONS          SHELLT   SHELLC
=========================================
S    :  S                 :   0   :    0
SP   :  S PX PY PZ        :   1   :    0
SPD  :  S PX PY PZ        :   2   :    0
     :  XX YY ZZ XY XZ YZ :       :
P    :  PX PY PZ          :   1   :    1
D    :  XX YY ZZ XY XZ YZ :   2   :    2
F    :  XXX YYY ZZZ XYY   :   3   :    2
     :  XXY XXZ XZZ YZZ   :       :
     :  YYZ XYZ           :       :
=========================================
\end{verbatim}
\newpage
\section{\sf COMMON /INFO/}
{\small
\begin{verbatim}
      INTEGER INFO
      COMMON /INFO/ INFO(10)
\end{verbatim}
}
This is an area reserved for the communication of 
miscellaneous information
between links. It is little used at the moment.
\begin{description}
\item[ INFO(1)..(3)] not used at present
\item[ INFO(4)] summary word for the calculation (see {\tt SUBROUTINE
ARCSET} in link 1)
\item[ INFO(5)..(10)] not used at present
\end{description}
\newpage
\section{\sf COMMON /PHYCON/}
{\small
\begin{verbatim}
      DOUBLE PRECISION PHYCON
      COMMON /PHYCON/ PHYCON(30)
\end{verbatim}
}
This common block holds the values of the physical constants
which are used in the program.  It is initialized by {\tt SUBROUTINE
PHYFIL} in link1 which also documents the sources for the values used.
\begin{description}
\item[PHYCON( 1)] {\tt TOANG},  angstroms per bohr
\item[PHYCON( 2)] {\tt TOKG},   kilograms per amu
\item[PHYCON( 3)] {\tt TOE},    esu per electron charge
\item[PHYCON( 4)] {\tt PLANCK},  Planck's constant
\item[PHYCON( 5)] {\tt AVOG},   Avogadro number
\item[PHYCON( 6)] {\tt JPCAL},  joules per calorie
\item[PHYCON( 7)] {\tt TOMET},  metres per bohr
\item[PHYCON( 8)] {\tt HARTREE}, joules per Hartree
\item[PHYCON( 9)] {\tt SLIGHT}, speed of light
\item[PHYCON(10)] {\tt BOLTZ},  Boltzman constant
\item[PHYCON(11)..PHYCON(30)] not used at present
\end{description}
\newpage
\section{\sf COMMON /MUNIT/}
{\small
\begin{verbatim}
      INTEGER IUNIT
      COMMON/munit/ iunit(20)
\end{verbatim}
}
Defines the unit numbers for all external data files needed
by {\tt gnu80}.  {\tt IUNIT} is initialized by {\tt SUBROUTINE DEFUNT} in
link 1.  The purpose of this COMMON is to centralize the definitions
of the FORTRAN logical units required in the Gaussian system.  Thus,
if it is ever necessary to change to different logical units on
another machine, one should be able to merely change the definitions
in {\tt DEFUNT}.

Note that in  {\tt gnu80}, unit number 18 (munit 7) is not
initialized by {\tt DEFUNT}.  This unit is "hard wired" 
by a specific {\tt OPEN} satatement in the {\tt MAIN} segment.
\begin{description}
\item[IUNIT(1)]  not used.
\item[IUNIT(2)]  primary input (usually channel 5).
\item[IUNIT(3)]  primary printed output (usually channel 6).
\item[IUNIT(4)]  primary punched output (usually channel 7 but now obselete).
\item[IUNIT(5)]  {\tt BINWT} unit (obselete).
\item[IUNIT(6)]  {\tt BINRD} unit (obselete).
\item[IUNIT(7)]  Direct Access files and buckets (usually channel 18).
\item[IUNIT(12)] Repulsion integrals (usually channel 3).
\end{description}
\newpage
\section{\sf COMMON /IOP/}
{\small
\begin{verbatim}
      INTEGER IOP
      COMMON/IOP/ IOP(50)
\end{verbatim}
}
The system {\bf OPTIONS} are stored in this COMMON. The definitions
of the 50 options differ from overlay to overlay. This COMMON
is updated in {\tt FUNCTION NEXTOV}. See \ref{app3b} for a full description
of the contents of this COMMON block.
\newpage
\section{\sf COMMON /ILSW1/}
{\small
\begin{verbatim}
      INTEGER ILSW
      COMMON/ILSW1/ ILSW1(2)
\end{verbatim}
}
This is the so-called the Inter-Link Status ` Word' .
The inter-link status word is an area on read/write file 998 which is
used as a storage place for information required by {\em several} overlays.
The information is packed into the area bit-wise, and  {\tt SUBROUTINE
ILSW} is designed to facilitate the reading and updating of this
information.  The use of each bit (or group of bits) is described below.
\begin{verbatim}
      SUBROUTINE ILSW(IOPER,WHERE,WHAT) 
      INTEGER IOPER, WHERE, WHAT
\end{verbatim}
Each of the arguments is described below, first {\tt IOPER}:
\begin{description}
\item[IOPER = 1] update the bits indicated by {\tt WHERE} with the
value of {\tt WHAT}.
\item[IOPER = 2]. determine the status of the bits indicated by
{\tt WHERE} and store the result in {\tt WHAT}.
\end{description}

{\tt WHERE}     is the sequence number (in {\tt COMMON/CWD/}) of the
information which is to be read or updated.

{\tt WHAT}  this argument is used to update the indicated
bits if {\tt IOPER}=1, or is returned with the
status of these bits if {\tt IOPER}=2.
The current possibilities for {\tt WHERE} are:
{\small
\begin{verbatim}
1     CONST    0-1    SCF constraints.
                      0 ... real RHF.
                      1 ... real UHF.
                      2 ... complex RHF.
                      3 ... complex UHF.

2     IPURD    2      number of d functions.
                      0 ... five d.
                      1 ... six d.

3     BASIS    3-5    type of basis set.
                      0 ... minimal.
                      1 ... extended.
                      2 ... general.

4     POLAR    6-8    polarization functions (p, d).
                      bit 6=0 ... no p functions on hydrogen.
                      =1 ... p functions on hydrogen.
                      bit 7=0 ... no d functions on first row.
                      =1 ... d on first row.
                      bit 8=0 ... no d on second row.
                      =1 ... d on second row.

                      note that these bits can be addressed
                      individually (see below).

5     CONVER   9      SCF convergence.
                      0 ... the SCF has not gone max cycles.
                      1 ... the SCF has gone max cycles.
                      6     stabil   10     2nd order stability of wave function.
                      0 ... not tested or not stable.
                      1 ... tested and found stable.

7     SYM      11     symmetry of univariate search in second.
                      0 ... search is symmetric for tau pos/neg
                      1 ... search is not symmetric.

8     PUESS    12     type of guess for {\tt TAU} (see {\tt SCFDM}).
                      0 ... tau not available.
                      1 ... tau available.

9     BNREAD   14     whether binrd has tried and failed to
                      read a file of records.
                      0 ... no.
                      1 ... yes.

10    IFPONH   6      polarization on hydrogen (see above).

11    IFDON1   7      polarization on first row (see above).

12    IFDON2   8      polarization on second row (see above).

13    IFFON1   15     f functions on first row.
                      0 ... no.
                      1 ... yes.

14    IFFON2   16     f functions on second row.
                      0 ... no.
                      1 ... yes.

15    IFFON3   17     f functions on third row.
                      0 ... no.
                      1 ... yes.

16    F7F10    18     number of f functions.
                      0 ... seven f.
                      1 ... ten f.

17    IFPOL    15-17  used to address these bits collectively.
                      (see description above.)

18    IFAU     13     if coordinates in blank COMMON are in
                      atomic units.
                      0 ... no, angstroms instead.
                      1 ... yes.

19    IFFP     19     if this is a Fletcher-Powell
                      optimization run.
                      0 ... no.
                      1 ... yes.

20    PRTOFF   20     control print in nextov and chain
                      0 ... print turned on
                      1 ... print turned off

21    PSAVE   21     paper save master switch.
                      0 ... (off) paper used as normal.
                      1 ... (on) paper usage drastically
                      reduced.

22    OPCLO   22     closed/open selector.
                      0 ... closed shell.
                      1 ... open shell.
                      note: this bit is used for RHF open
                      shell.

23    IFGRD   23    if this is a gradient
                      optimization run
                      0 ... no.
                      1 ... yes.

24    IFRC    24    if energy derivatives are to be
                      calculated
                      0 ... yes.
                      1 ... no.

25    IFARCH  25   if a summary of the job is to be placed
                      into the archive file.
                      0 ... no.
                      1 ... yes.

26    NOSYM   26-27 should the point group symmetry of the
                      molecule be used to eliminate
                      redundant computation
                      0 ... yes.
                      1 ... no, the point group will be
                      computed and printed only.
                      2 ... no, a change in the point group
                      has been detected during an
                      optimization.
\end{verbatim}
}
\newpage
\section{\sf COMMON /GEN/}
{\small
\begin{verbatim}
      DOUBLE PRECISION DGEN
      COMMON/GEN/DGEN(47)
\end{verbatim}
}
This is a COMMON area where many of the ` single number'
results of the current calculation are stored. The value of
{\tt DGEN(I)} for the various values of I are given below:
\begin{description}
\item[1]  electronic configuration of SCF wave-function.
\item[2-20]    not used.
\item[21] root-mean-squared force of optimized parameters.
\item[22] dipole moment.
\item[23] rms error in density matrix.
\item[24-30]   not used.
\item[31] {\tt TAU} from {\tt SCFDM}
\item[32] SCF energy.
\item[33] UMP2 energy.
\item[34] UMP3 energy.
\item[35] UMP4(sdtq) energy.   (reserved for future use)
\item[36] CID energy, or {\tt E(VAR1)}. (reserved for future use)
\item[37] CISD energy.     (reserved for future use) 
\item[38]      MP4DQ energy.    (reserved for future use) 
\item[39] MP4SDQ energy    (reserved for future use)      
\item[40] CCD energy.   (reserved for future use)         
\item[41] nuclear repulsion energy.
\item[42] T (length of correction of reference determinant).
\item[43] updated energy for Fletcher-Powell optimizations.
\item[44] $<\hat{S}^2>$ of SCF wave function.
\item[45] $<\hat{S}^2>$ corrected to first order (after {\tt DOUBAR}).
\item[46] $<\hat{S}^2>$ corrected for doubles (not implemented).
\item[47] a0.
\end{description}
