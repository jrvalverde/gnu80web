\appendix
\chapter{\sf List of links in {\tt gnu80}}
\label{app3a}
The following is a list of the {\bf Links} of {\tt gnu80}
and a brief summary of their function.
\begin{description}
\item[MAIN] initialization, controls overlaying.
\item[L001]  reads route, initializes disc files, fills error-function
interpolation table.
\item[L101]     Reads:
\begin{enumerate}
\item Title
\item Z-matrix
\item Variables (if any)
\item Constants (if any)
\end{enumerate}
\item[L102]     controls ``Fletcher-Powell'' optimization.
\item[L103]     controls ``Berny'' optimization.
\item[L105]     controls ``Murtaugh-Sargent'' optimization.
\item[L202]     calculates coordinates from Z-matrix and determines:
\begin{enumerate}
\item Stoichiometry
\item Framework group
\item Symmetry Information.
\item Rotates molecule to standard (centre of charge) orientation.
\end{enumerate}
\item[L301]     Generates basis set information.
\item[L302]     Computes overlap, kinetic, and potential integrals.
\item[L303]     Computes x-, y- and z-dipole integrals.
\item[L305]     Setup for pseudo-potential integrals; obselete, it is now a
dummy and its removal is overdue.
\item[L306]     Computes pseudo-potential integrals.
\item[L310]     Primitive two-electron integral program (spdf). for testing
purposes only.
\item[L311]     sp two-electron integral program.
\item[L314]     (sp)d two-electron integral program.
\item[L401]     Generates initial guess at density matrix.
\item[L501]     RHF closed shell SCF.
\item[L502]     UHF open shell SCF.
\item[L503]     Direct Minimization SCF (does RHF/UHF, real/complex).
\item[L505]     Restricted open-shell SCF program.
\item[L601]     Mulliken population analysis; Fermi contact analysis for open
shell systems; computes dipole moment.
\item[L602]     Provides output for interfacing with the {\bf RPAC}
suite of programs for Random-Phase Approximation Calculations of molecular
properties.
\item[L701]     Calculates one-electron integral first derivatives.
\item[L702]     Calculates two-electron integral first derivatives for sp
functions.
\item[L703]     Calculates two-electron integral first derivatives for spd
functions.
\item[L705]     Calculates pseudo-potential first derivative integrals for sp
bases.
\item[L716]     Converts forces to internal coordinate forces and communicates
with optimization control programs.
\item[L801]     Setup program for transformation of two-electron integrals;
Generates molecular orbital coefficient matrix and eigenvalues
removing the orbitals which are not used in the correlation
study.
\item[L802]     RHF closed-shell transformation of two-electron integrals.
\item[L803]     UHF open-shell transformation of two-electron integrals.
\item[L901]     Computes anti-symmetrized two-electron integrals; computes MP2
energy and Moller-Plesset first-order wave-function.
\item[L909]     Initialization for CID and higher-order energy perturbation
calculations.
\item[L910-L912] Carry out higher-order perturbation calculations, or one CID
iteration.
\item[L913]     Calculates various energies. in the case of a CID calculation,
L913 tests for convergence, and if necessary returns to L910
for the start of the next iteration.
\item[L9999]    Terminates the run normally.
\end{description}
\chapter{\sf Link Descriptions and Options}
\label{app3b}
\section{\sf OVERLAY 1} 
Gaussian system input and optimization control
In the {\tt gnu80} system, OVERLAY 1 contains those programs which
read in geometry and optimization input and those which control
optimization calculations. Currently, the following Links are
implemented:
\begin{description}
\item[Link 101] Basic input. this Link reads in the title, Z-matrix,
variables, and constants sections of the input.
\item[Link 102] Fletcher-Powell optimization program. this Link implements
the algorithm of Fletcher and Powell as modified by
Binkley and Pople.  It, along with the interface code
present in other Links, is capable of driving geometry
optimizations.  Derivative information is gained by
numerical differentiation of the energy with respect to
the geometrical variables.
\item[Link 103] Gradient optimization program. This program is used in
conjunction with those Links which produce analytical
energy derivatives to perform geometry optimizations.
\item[Link 105] Murtaugh-Sargent optimization program.  This program also
uses analytically determined first derivatives to optimize
geometry with respect to energy; in general, it
is not as efficient as Link 103, it has a more stable
algorithm and is less prone to aimless wandering on the
potential surface.
Note that most of the options for Link 103 are only examined in the
first call to the Link in any given run.
\end{description}
\subsubsection{\sf OPTIONS for OVERLAY 1}
\begin{description}
\item[IOP(5)] L103  mode of optimization  \\
0  find local minimum \\
1  find a saddle point \\ 
N find a stationary point on the energy surface with N negative
eigenvalues of the 2nd derivative matrix
\item[IOP(6)] L103 and L105  maximum number of steps {\bf (OPTCYC=N)} \\
0  {\tt NSTEP = MIN(20,NVAR+10)} \\
N {\tt NSTEP=N}
\item[IOP(7)] L103 and L105  convergence on the first derivative
and estimated displacement for the optimization:  \\
(RMS first derivative) $<$ {\tt CONFV}, \\
(RMS estimated displacement) $<$ {\tt CONVX=4*CONFV} \\
0  {\tt CONFV = 0.0003} Hartree/bohr or radian \\
N {\tt CONFV = 0.001 / N}
\item[IOP(8)] L103  maximum step size allowed during optimization \\
0  {\tt DXMAXT = 0.2} bohr or radian \\
N {\tt DMAXT = 0.01 * N}
\item[IOP(10)] L103  input of initial second derivative matrix
all values must be in atomic units (Hartree, bohr, and radians). \\
0  NO \\
1 {\tt READ ((FC(I,J),J=1,I),I=1,NVAR)}  {\tt (8F10.6)} \\
2  {\tt READ I,J, FC(I,J)}      {\tt (5I3,F20.0)} end with a blank record. \\
3  read from guess file \\
\item[IOP(12)] L103  optimization control parameters \\
0  use default values \\
1  read in new values for all parameters (see {\tt INITBS}) \\
\item[IOP(14)] L103  minimum RMS force for which a linear search
will be attempted  \\
0  {\tt FSWTCH = 0.01} Hartrees/bohr or /radian \\
N {\tt FSWTCH = 0.001 * N}
\item[IOP(15)] L103  abort if derivatives too large \\
This has been disabled due to the fact taht large
detivatives do genuinely appear in molecules of high symmetry
(when one variable may change much geometry) and molecules containing
heavy atoms. The Original usage is given below: \\
0  {\tt FMAXT = 1.0} Hartree / bohr or radian \\
N {\tt FMAXT = 0.1 * N}
\item[IOP(16)] L103  maximum allowable magnitude of the eigenvalues of the
second derivative matrix. If the limit is exceeded, the size of
the eigenvalue is reduce to the maximum, and processing
continues. \\
0  {\tt EIGMAX = 25.0} Hartree / bohr**2 or radian**2 \\
N {\tt EIGMAX = 0.1 * N}
\item[IOP(17)] L103 minimum allowable magnitude of the eigenvalues
of the second derivative matrix. similar to IOP(16)  \\
0  {\tt EIGMIN = 0.0001} \\
N {\tt EIGMIN = 1.0 / N}
\item[IOP(18)] L103  star only option {\bf (OPT=STARONLY)}  \\
0  proceed normally \\
1  second derivatives will be computed as directed on the variable \\
definition records. NO optimization will occur. 
\item[IOP(19)] L103  skip linear search.  \\
0  NO \\
1  YES. \\
\item[IOP(20)] L101  input units {\bf (UNITS=)}  \\
0  angstroms degrees \\
1  bohrs  degrees \\
2  angstroms radians \\
3  bohrs  radians \\
\item[IOP(29)] L101  Specification of nuclear centers  \\
Note that this option is usually set in conjuntion with IOP(29)
in Link 202.  \\
0  by Z-matrix \\
1  by direct coordinate input \\
\item[IOP(30)] L101  Nuclear Charges \\
0  nuclear charge equals atomic number \\
1  read in center name or number {\tt I}, read in charge {\tt CHG} (floating 
point) nuclear charge for {\tt I}-th nucleus is replaced by {\tt CHG}.
recordset must be ended by blank record.
\item[IOP(32)] L103  writing of second derivatives to the punch unit (GUESS
file) at the conclusion of the optimization.  \\
0  NO \\
1  YES \\
\item[IOP(33)] L101 L102 L103  debug print  \\
0  off \\
1  on \\
\item[IOP(34)] L101 L102 L103  debug + dump print  \\
0  off \\
1  on \\
\end{description}
\section{\sf OVERLAY 2 (Link 202 only)}
Procedure to determine the coordinates, given the Z-matrix, and to
analyze the molecular symmetry, if requested.
This Link receives the Z-matrix or coordinates from the RW-files, and
determines the framework group of the molecule and produces the
standard (3 by 3) transformation matrices. When input consists of a
Z-matrix, routine {\tt ZTOC} is called upon to obtain the coordinates.
\subsection{\sf OPTIONS for OVERLAY 2}
\begin{description}
\item[IOP(9)]  printing of distance matrix. \\
0  print distance matrix (only if there are more than 
two atoms in the molecule). \\
1  do not print the distance matrix. \\
\item[IOP(10)]  Tetrahedral angle fixing.  \\
0  angles within 0.001 degree of 109.471 will be set to
{\tt acos(-1/3).} \\
1  do not test for such angles. \\
\item[IOP(11)]  printing of Z-matrix and resultant coordinates.  \\
0  print. \\
1  do not print. \\
\item[IOP(12)]  Crowding abort control  \\
0  if two atoms are less than 0.5 angstroms apart, abort the run. \\
1  do not abort the run for small interatomic distances. \\
\item[IOP(15)]  Symmetry control.  \\
0  leave symmetry in whatever state it is presently in. \\
1  unconditionally turn symmetry off. Note that {\tt SYMM} is still \\
called, and will determine the framework group.  However, the
molecule is not oriented.
\item[IOP(29)]  direct coordinate input
Note that this option is usually set in conjunction with
IOP(29) in L101.  \\
0  coordinates were input via the Z-matrix. \\
1  coordinates were input directly. \\
\end{description}
\section{\sf OVERLAY 3}
Gaussian system integral package.  \\
Overlay 3 consists of the necessary programs to evaluate the one-
and two-electron integrals required for an SCF calculation. \\
This
package consists of the following Links:
\begin{description}
\item[L301]   Constructs basis set, either through internally stored data or
through input and performs other initialization chores for the
integral programs.
\item[L302]   Calculates the Overlap {\bf (S)}, 
Kinetic Energy  {\bf (T)} and Core-Hamiltonian {\bf (H)}
one-electron integrals.
\item[L303]   Calculates the x-, y- and z-dipole integrals (one-electron).
\item[L305]   Formula generator for the pseudo-potential program. OBSELETE
\item[L306]   Evaluates one-electron pseudo-potential integrals.
\item[L310]   Evaluates s-, p-, d- and f-type two-electron integrals by use of
general formula; used only for testing.
\item[L311]   Evaluates two-electron integrals for those shell combinations
that contain s- and p-functions.
\item[L314]   Evaluates s-, p-, d- and f-type two-electron integrals using the
method of Rys polynomials.
\end{description}
Note that there is some overlap in functionality between the
two-electron integral programs listed above. In standard calculations,
one would normally use Links 301, 302, 303, 311 and 314 to obtain all
the necessary integrals for basis functions up (and including)
d-functions. If d-functions are not present in the route, L314 can be
omitted.

\subsection{\sf OPTIONS for OVERLAY 3}
\begin{description}
\item[IOP(5)]  Type of basis set  \\
0  Minimal STO-2G to STO-6G  \\
1  Extended 4-31G,5-31G,6-31G \\
2  Minimal STO-NG (valence functions only) \\
3  Extended LP-N1G (valence basis for coreless Hartree-Fock 
pseudopotentials)  \\
\ \ \ The Los Alamos Split Valence Basis is equivalent to
LP-31G.
4  Extended 6-311G (UMP2 frozen core optimized) basis. \\
* and ** effected by IOP(21) as usual.  \\
use IOP(8) to select 5d/6d.  \\
5  Split Valence N-21G (or NN-21G) basis for first or 
second row atoms. (various implementations may omit
second row atoms.) see IOP(6) for determination of the
number of Gaussians in the inner shell.  \\
7  GENERAL;  see routine {\tt GBASIS} for input instructions. \\
\item[IOP(6)]  Number of Gaussian functions \\
N STO-NG, N-31G, LP-N1G, STO-NG-valence, N-21G.  \\
Note: if IOP(5)=3 and IOP(6)=8; LP-31G for Li, Be, B, Na, Mg, Al
LP-41G for other row one and two
atoms.  \\
Default Options:  \\
  IOP(6)=0  \\
if IOP(5)=0 N=3 STO-3G  \\
if IOP(5)=1 N=4 4-31G  \\
if IOP(5)=2 N=3 STO-3G (valence)  \\
if IOP(5)=3 N=3  \\
if IOP(5)=5 N=3  \\
When IOP(5)=7 (general bases), this option is used to control where the
basis is taken from:  \\
0  read general basis from the input stream. \\
1  read the general basis from the rw-files and merge with 
the coordinates in blank common to produce the current
basis.  \\
This option is useful when doing general basis geometry optimizations.
\item[IOP(7)]  Polarization type.  \\
0  none. \\
1  Add a set of second-order Gaussians {\bf (6D)} to first row n-31G 
atoms (N-31G* basis),
or, \\
if IOP(5)=0, add a set of d-functions {\bf (5D)} to second
row STO-NG atoms (STO-NG* basis).
or, \\
if IOP(5)=3, add a set of 5d to rows 1 and 2  \\
2  Does the same as IOP(7)=1, but additionally adds a set 
of p-functions to N-31G hydrogen atoms (N-31G** basis). \\
\item[IOP(8)]  Selection of second-order Gaussians/true d-functions. \\
0  selection determined by the basis: \\
N-31G*      6d  \\
N-31G**     6d  \\
N-21G*      5d  \\
STO-NG*     5d  \\
LP-N1G*     5d  \\
LP-N1G**    5d  \\
General Basis  5d.  \\
1  use 5d throughout.  ({\bf 5d}) \\
2  use 6d throughout.  ({\bf 6d}) \\
\item[IOP(9)]  selection of third-order/true f-functions. \\
0  reserved for when f-functions are part of standard bases. \\
1  use 7f throughout. \\
2  use 10f throughout. \\
Note f-functions only partly implemented.
\item[IOP(10)]  Modification of internally stored bases.  \\
0  none. \\
1  Read in replacement scale-factors. Standard scale-factors
are listed below.  \\
2  Read in replacement polarization exponents for N-31G*, 
N-31G** and STO-NG* bases. standard values are listed
after the standard scale-factors.  \\
3  combination of 1 and 2 above (ie. read in both scale-factors 
and polarization exponents).  \\
For general basis runs, this option has the following
definitions:  \\
0  Not a scale-factor run. \\
1  Continuation entry for scale-factor optimization. \\
2  Initialization entry for a scale-factor optimization run. \\
See routine {\tt GBASIS} for further details on scale-factor
optimizations.
\item[IOP(11)]  Control of two-electron integral storage format.  \\
0  Standard integral format is used. \\
1  Raffenetti 1 integral format is used. Can only be used with the 
closed shell SCF.  \\
2  Raffenetti 2 integral format. Suitable for use with the open 
shell (UHF) SCF.  \\
3  Raffenetti 3 integral format. Suitable for use with open shell 
RHF SCF and the post-SCF procedures.  \\
\item[IOP(16)]  Check for pseudopotential run. See IOP(17) through
IOP(19) for more details. \\
0  NO \\
1  YES \\
NOTE IOP(17)-IOP(19) apply only if IOP(16)=1
\item[IOP(17)]  Specification of pseudopotentials  \\
0  Use internally stored 'Coreless Hartree-Fock' \\
7  Read in from input stream (see {\tt PINPUT} for details) \\
\item[IOP(18)]  Printing of pseudopotentials  \\
0  Print only when these are read \\
1  Print \\
2  Do not print \\
\item[IOP(19)]  Specification of substitution potential type  \\
0  Do not use any substitution potentials \\
N Replace the standard potential of this run (eg.CHF),
with a substitution potential of type N wherever such
a substitution potential exists.  \\
\item[IOP(23)]  Definition of two-electron integral scale factor.
(for a discussion of how two-electron integrals are
stored, see the program documentation).  \\
0  default, 10**8. \\
N (10**8)*(10**N).
\item[IOP(24)]  Printing of Gaussian function table.  \\
0  Table is printed only if non-standard features are used. \\
1  Print table. \\
2  Do not print table. \\
10 Print out the basis in a form suitable for GEN input \\
\item[IOP(25)]  Number of last two-electron integral Link.
\item[IOP(26)]  Test option {\bf (TEST)}  \\
0  proceed as normal \\
1  print Gaussian function table and abort job after L301 \\
\item[IOP(27)]  Handling of small two-electron integrals.  \\
0  Discard integrals with magnitude less than 10**-6. \\
N Discard integrals with magnitude less than 10**-N.  \\
Beware of underflow when N is made large.
\item[IOP(28)]  Polarization option  \\
0  NO special features invoked. \\
10 Compute all two-electron integrals in L310 \\
Note: L310 should be included in the route by the use of
the {\bf NONSTD} command.  \\
Note: option 25 should be set to 10.  \\
14 Compute all two-electron integrals in L314. \\
\item[IOP(29)]  Rotation of coordinates.  \\
0  Coordinates are not rotated. \\
1  Read 1 record in {\tt 3E20.10} format giving the three Euler 
angles $\phi,$ $\theta$ and $\chi$.  \\
\item[IOP(30)]  Control of two-electron integral symmetry.  \\
0  Two-electron integral symmetry is turned off. \\
1  Two-electron integral symmetry is turned on. Note, 
however, the {\tt SUBROUTINE SET2E} will interrogate {\tt ILSW} to
see if the symmetry rw-files exist. If they don't, symmetry
has been turned off elsewhere, and {\tt SET2E} will also
turn it off here.
\item[IOP(32)]  Punching of {\tt COMMON/B/} in compressed form.  \\
0  NO punching. \\
1  punch {\tt COMMON/B/} in compressed form. The data is written to 
the GUESS file for possible use by the initial guess
in a subsequent job.  \\
\item[IOP(33)]  Integral package printing.  \\
0  NO integrals are printed. \\
1  Print one-electron integrals. \\
3  Print two-electron integrals in standard format. \\
4  Print two-electron integrals in debug format. \\
5  Combination of 1 and 3. \\
6  Combination of 1 and 4. \\
\item[IOP(34)]  Dump option.  \\
0  NO dump. \\
1  Control words printed (as usual). \\
2  Additionally, {\tt COMMON/B/} is dumped at the beginning 
of each integral Link.  \\
3  Additionally, the integrals are printed (standard format). \\
\end{description}
\section{\sf OVERLAY 4 (Link 401 only)}
This is a program which produces an initial guess to the solution
of the SCF equations.  This guess is in the form of molecular orbital
coefficients and/or density matrices which are stored on the
appropriate read-write files.  The steepest descents procedure (Link
503) requires MO coefficients as an initial guess, while the classical 
SCF procedures (Links 501 and 502) require density matrices.  Since a
density matrix can be produced from the MO coefficients, but not vice
versa, the former is a more constraining requirement.
There are several ways in which this guess may be produced.  One
easy way is to diagonalize the core hamiltonian.  In general, this is
not a very good guess, but it is applicable to any basis set, and is
available as an option.

Another type of guess is called the H\"{u}ckel guess, which is modeled
after extended Huckel MO theory.  Essentially, the initial guess is
formed from internally stored constants (for more details see
{\tt SUBROUTINE  HUCKEL}).  These constants were determined from studies on
internal minimal and split valence basis sets (STO-3G, 4-31G, 6-31G),
so the use of this type of guess with bases other than these is not
recommended.

This Huckel guess can be applied to other bases in the following
way:  the guess MO coefficients are formed from internal data as if
there were an STO-3G basis set on the molecule.  The guess MO vectors
in the desired basis are then formed by choosing the vectors which give
the best least-squares fit to those described in the STO-3G basis.
Since this will usually produce fewer than N basis vectors,the MO
coefficient matrix must be completed with orthonormal vectors of the
proper symmetry if the full matrix is required (Link 503).  This
procedure is called a projected H\"{u}ckel guess, and is applicable to any
basis set.

A still better type of guess, usually, is to read the coefficient
or density matrix from the input stream.  If the matrix read in is for a basis
other than the one used in the current run, the matrix can be projected
(by a least-squares fit) into the desired basis.  Since the projected
MO vectors can be normalized and orthogonalized, and this is not
possible for a projected density matrix, projection of MO coefficients
usually produces a better guess than projection of the density matrix.
\subsection{\sf OPTIONS for OVERLAY 4}
\begin{description}
\item[IOP(5)]  Type of guess.  \\
0  Default.  This gives a Huckel guess for minimal bases, or 
a projected huckel guess otherwise.  \\
1  Read guess from GUESS file. \\
2  Guess from core Hamiltonian. \\
3  Blocked Huckel guess. \\
4  Projected Huckel guess. \\
5  Renormalize and orthogonalize the coefficients which are 
currently on the read-write files. \\
\item[IOP(6)]  Forced projection when guess is read in  \\
0  Do not force projection. \\
1  Force projected guess, even when bases are identical. \\
2  Suppress projection. \\
\item[IOP(7)]  SCF constraints.  \\
0  use {\tt ILSW} to determine. \\
1  real RHF. \\
2  real UHF. \\
3  complex RHF. \\
4  complex UHF. \\
5  complex, but use {\tt ILSW} to decide whether RHF/UHF. \\
\item[IOP(8)]  Alteration of configuration.  \\
0  Do not alter configuration. \\
1  Read in pairs of integers (2I3) indicating which pairs of MOs 
are to be interchanged.  Pairs are read until a blank record is
encountered.  \\
Note:  \\
If the configuration is altered on an open shell system, two sets
of data as described above will be expected first for $\alpha$,
second for $\beta$.
\item[IOP(10)]  Orbitals to mix with complex.  \\
0  Mix the HOMO with the LUMO. \\
1  Read from records (2I3) pairs of integers indicating which pairs of 
orbitals are to be mixed.  Reading is terminated by a blank record.  \\
Note:  \\
The same considerations for open shell systems which
applied in IOP(8) apply here, also.
\item[IOP(12)]  Off-diagonal scale factor for H\"{u}ckel guess.  \\
0  Default.  (K/2=.875) \\
N (K/2=N*.4375)
\item[IOP(16)]  Completion of coefficient matrix after projection.  \\
0  Complete the coefficient matrix after projection. \\
1  Do not complete. \\
\item[IOP(22)]  Type of basis set.  \\
0  Use {\tt ILSW} to determine. \\
1  Minimal basis. \\
2  Extended basis. \\
7  General basis. \\
\item[IOP(23)]  five/six d, seven/ten f.  \\
0  use {\tt ILSW} to determine. \\
1  five d, seven f. \\
2  six d, seven f. \\
3  five d, ten f. \\
4  six d, ten f. \\
\item[IOP(24)]  Polarization functions on hydrogen.  \\
0  Use {\tt ILSW} to determine. \\
1  NO polarization functions on hydrogen. \\
2  A set of p functions on each hydrogen. \\
\item[IOP(25)]  Polarization functions on first row atoms.  \\
0  Use {\tt ILSW} to determine. \\
1  NO polarization functions. \\
2  A set of d functions on each first row atom. \\
\item[IOP(26)]  Polarization functions on second row atoms.  \\
0  Use {\tt ILSW} to determine. \\
1  NO polarization functions. \\
2  A set of d functions on second row atoms. \\
Note that whenever {\tt ILSW} is over-ridden, it is also over-written.
\item[IOP(33)]  Printing of guess.  \\
0  NO printing. \\
1  Print the MO coefficients. \\
2  Print everything. \\
\item[IOP(34)]  Dump option.
0  NO dump. \\
1  Turn on all possible printing. \\
\end{description}
\section{\sf OVERLAY 5}
\begin{description}
\item[Link 501] Perform Roothaan SCF procedure using the method
of repeated diagonalizations.
\item[Link 502]  solution of the Pople-Nesbet equations by the method of
repeated diagonalizations.  J. Chem. Phys. 22, 571 (1954)
\item[Link 505]  Solves the Binkley, Pople and Dobosh equations for a
spin-restricted open shell system. The wavefunction
produced is not compatible with the {\tt gnu80} post-SCF
procedures (CI, MP2, MP3).
\end{description}
% UP TO HERE IN CLEAN-UP
\subsection{\sf OPTIONS for OVERLAY 5 (501,502,505 but not 503)}
\begin{description}
\item[IOP(5)] L501,L502  location of input density matrix.  \\
0  density matrix is taken from the rw-files. \\
1  the density matrix is read in via routine {\tt BINRD}. \\
\item[IOP(6)] L501,L502 requested convergence on the density matrix.  \\
0  iterations are performed until the RMS convergence on the density 
matrix is $<$ 10**(-5) or {\tt MAXCYC} is reached.  \\
N (1 $<$ N $<$ 8) requested convergence is 10**(-N).  \\
\item[IOP(6)] L505   convergence on the density matrix.  \\
0  5.0D-7 (Note that this is less than in the closed shell RHF 
program because here we are converging on three matrices.  \\
N 1 $<$ N $<$ 8 final convergence=10**(-2*N).  \\
\item[IOP(7)] L501,L502  maximum number of SCF iterations. 
{\bf (SCFCYC= )}  \\
0  up to 32 iterations will be performed. \\
N up to N iterations.  \\
\item[IOP(7)] L505  maximum number of SCF cycles.  \\
0  20 cycles. (Note that each cycle involves the formation of three 
Fock matrices;  appropriate time should be allowed.  \\
N 1 $<$ N $<$ 7 number of cycles is 2**(IOP(7)-1).  a value of
one will permit only a single cycle.  \\
7  {\tt MAXCYC}=64*IOP(22)+8*IOP(23)+IOP(24). \\
\item[IOP(8)] L501,L502  energy convergence.  \\
0  convergence on the density matrix. 
see option 6 for details.  \\
N (0 $<$ N $<$ 7)  the SCF is assumed to have converged when the
change in the energy is .le. 10**(-3-N).  \\
7  input desired value via {\tt INCRD}, see below. \\
Note that if this option is set, the density matrix criterion is not
used at all.
\item[IOP(11)] L501,L502  extrapolation control.  \\
0  both three-point and four-point extrapolation are performed when 
applicable.  \\
1  three-point extrapolation is inhibited, but the program will 
still perform four-point extrapolation when possible.  \\
2  both three-point and four-point extrapolation schemes are 'locked 
out' (ie. disabled).  \\
\item[IOP(12)] L501,L502  entry mode.  \\
0  normal entry mode, regular SCF is performed. \\
1  control is passed immediately to the punch/print code (IOP(32)). 
this is useful at the termination of an optimization run.  \\
\item[IOP(13)] L501,L502,L505  action on convergence failure.  \\
0  the run is terminated in error mode (via {\tt LNK1E}) if the SCF fails 
to converge.  \\
1  the run is allowed to continue, but the convergence failure bit 
in {\tt ILSW} is set.  \\
\item[IOP(14)] L501  UHF test option.  \\
0  no. \\
1  YES, turn the current run into a UHF run at the end of this Link. 
\item[IOP(14)] L502 control of annihilation of spin contaminants.  \\
0  calculation is performed (provided of course that enough space 
exists in the rw-files).  \\
1  calculation is bypassed. \\
2  calculation is performed, contingent on space, and the system 
rw-files for the appropriate density matrices are updated (useful
if one wants a population analysis).  \\
\item[IOP(16)] L505  control of use of convergence routine.  \\
0  use convergence routine. \\
1  lock-out convergence routine. \\
\item[IOP(32)] L501,L505  punch (via {\tt BINWT}) option. {\bf Obselete}  \\
0  NO punching is performed. \\
1  the molecular orbital coefficients are written to the guess file 
at the end of the job.  These may provided as an initial guess to
a subsequent job (see Link 401).  \\
2  the MO coefficients and the density matrix are punched at the end 
of the SCF.  \\
3  the MO coefficients and the density matrix are punched at the end 
of each iteration of the SCF.  \\
\item[IOP(32)] L502 whether to save the MO coefficients or density matrices
on the guess file.  \\
0  don't save. \\
1  save final MO coefficients. \\
2  save final density matrices. \\
3  save both. \\
4  save both each cycle of the SCF. \\
\item[IOP(33)] L501,L502,L505 print option.  \\
0  only summary results are printed (with possible control from the 
'no-print' option).  \\
1  the eigenvalues and the MO coefficients are printed at the end of 
the SCF.  \\
2  same as IOP(33)=1, but additionally the density matrix is 
printed.  \\
3  same as IOP(33)=2, but at the end of each iteration. \\
4  same as IOP(33)=3, but all matrix transactions are printed \\
(BEWARE: much output even on small molecules.)
\item[IOP(34)] L501,L502,L505  dump option.  \\
Standard system defaults apply here.  \\
Input via routine incrd: (L501,L502)  \\
If IOP(8) = 7 or IOP(6)=8, program will read one record in
free-field format to obtain the user supplied values for density matrix
convergence and energy convergence.  \\
This one record has two fields:  \\
Field one: floating point  density matrix convergence criterion  \\
Field two: floating point  energy convergence criterion.  \\
The appropriate field is only used if the associated option
is set to 7.  \\
\end{description}
\section{\sf Link 503  SCFDM}
Solution of the Pople-Nesbet equations by means of a direct
minimization method involving a sequence of univariate searches.  
\subsection{\sf OPTIONS for Link 503}
\begin{description}
\item[IOP(6)]  convergence on density matrix  \\
0  5.*10**(-5) \\
N 10**(-N)  \\
\item[IOP(7)]  maximum number of univariate searches  \\
0  32 \\
N N cycles.  \\
\item[IOP(8)]  selection of the procedure of direct minimization  \\
0  steepest descent with search parameters default \\
1  steepest descent with search parameters read (see below) \\
2  classical SCF (Roothaan's method of repeated diagonalization \\
4  conjugate gradients with search parameters default \\
5  conjugate gradients with search parameters read (see below) 
the search parameters are: {\tt MAX}: number of search points (I1)
{\tt MIN}: number of search points (I1)
initial stepsize, {\tt TAU} (G18.5)
scaling factor for subsequent {\tt TAU} (G20.5) and
{\tt Q} (G20.5)  \\
\item[IOP(9)]  switch to classical SCF after density matrix has
achieved a certain convergency  \\
0  NO \\
1  YES, criterion default: 10(**-3) \\
2  YES, criterion read in (format G16.10) \\
\item[IOP(11)]  apply extrapolation procedures for classical SCF
0  four-point only \\
1  four-point only \\
2  none \\
\item[IOP(12)]  null entry for final saving of data.  \\
0  normal entry. \\
1  null entry (zero cycles). this is for saving 
final MO coefs in optimizations. see IOP(32).  \\
\item[IOP(14)]  reordering of the orbitals (maintaining continuity
of the wavefunction along the search path)  \\
0  on: Bessel criterion \\
1  on: stronger individual-overlap criterion \\
2  off \\
\item[IOP(15)]  controls the auto-adjustment of {\tt TAU} in {\tt INTOPN}  \\
0  done \\
1  {\tt TAU} is kept fixed \\
\item[IOP(16)]  inhibit performance of minimization of alternate
wavefunction provided by second order procedures  \\
0  NO \\
1  YES \\
\item[IOP(17)]  condition the off-diagonal terms of the MO-Fock
matrix:  \\
-set to zero if {\tt GABS(F(I,J)).LE.FUZZY}  \\
-delete coupling terms between almost degenerate  \\
{\tt (DELTA E .LE. DEGEN)} MO vectors  \\
0  {\tt FUZZY}=1.d-10, {\tt DEGEN}=2.d-5 \\
1  fuzzy and degen read in (2d20.14) \\
\item[IOP(18)]  cutoff criteria in symmetry determination of MOs.  \\
-symmetry is determined if largest off-diagonal MO  \\
Fock-matrix element {\tt GABS(F(I,J)).GE.STHRS}  \\
-elements {\tt GABS(F(I,J)).LE.L8AN} are considered to be zero  \\
0  {\tt STHRS}=1.d-4, {\tt SPAN}=5.d-7 \\
1  {\tt STHRS} and {\tt SPAN} read in (2d20.14) \\
\item[IOP(19)]  print {\tt F(1),T}. (read one record with {\tt START},
{\tt END} in 2i2)  \\
0  NO \\
1  YES \\
\item[IOP(20)]  max-time exit (in order to dump for restart. see
DOUBAR)  \\
0  NO \\
1  YES \\
to obtain a max-time dump, proceed as follows:
set this option to 1,
the next Link to be performed should be L901 ({\tt DOUBAR}),
set IOP(15) there to 6.  \\
Note: set IOP(14) on the integral route record (write integrals on
tape). of course, tape 'c' has to be assigned  \\
To restart from such a dump:
the Link preceeding SCF should be {\tt DOUBAR}.
set the following options there: IOP(15)=7, IOP(19)=1
after that blank common, the integrals and the rwf are loaded
\item[IOP(21)]  action if {\tt OTEST} detects problems:  \\
0  abort run via {\tt LNK1E}. \\
1  continue run. \\
\item[IOP(31)]  override print-save option  \\
0  NO (use ILSW) \\
1  force print-save on \\
2  force print-save off \\
\item[IOP(32)]  save the MO coefficients and/or the density
matrix on the guess file.  \\
0  none \\
1  MO coefficients only \\
2  density matrix only \\
3  both \\
\item[IOP(33)]  printing  \\
0  NO printing \\
1  print MO coefficients at end \\
2  print everything at end \\
3  print everything each cycle ... and at end \\
\end{description}
\section{\sf OVERLAY 6 (Link 601 only)}
This program performs a Mulliken population analysis for 
a computed wave
function.
\subsection{\sf OPTIONS for OVERLAY 6}
\begin{description}
\item[IOP(5)]  open or closed shell.  \\
0  use {\tt ILSW} to determine. \\
1  forced open shell. \\
2  forced closed shell. \\
The remaining options are print/no-print options. if the
value of the option is zero, the default value (given below) is
assumed. if the option is set to 1, the information is printed,
and if it is 2, the printing is suppressed.  \\
0  default. \\
1  print. \\
2  do not print. \\
\item[IOP(6)]  distance matrix. default: no-print.
\item[IOP(7)]  molecular orbital coefficients. default: print.
\item[IOP(8)]  density matrix. default: no-print.
\item[IOP(9)]  full population analysis. default: print.
\item[IOP(10)]  gross orbital charges. default: print.
\item[IOP(11)]  gross orbital type charges. default: no-print.
\item[IOP(12)]  condensed to atoms. default: print.
\end{description}
\section{\sf OVERLAY 7}
OVERLAY 7 is concerned with calculation of first and second
derivatives of the energy with respect to nuclear coordinates.
\begin{description}
\item[Link 701] calculates and uses one-electron integral derivatives
to get first energy derivatives.
\item[Link 702] calculates and uses two-electron integral (sp only)
derivatives to get first energy derivatives.
\item[Link 703] calculates and uses two-electron integral (spd)
derivatives to get first energy derivatives.
\item[Link 705] calculates and uses one-electron pseudopotential
integrals.
\item[Link 716] completes evaluation of energy derivatives and
transforms results to internal coordinates.
\end{description}
\subsection{\sf OPTIONS for Link 701}
\begin{description}
\item[IOP(33)]  print option.  \\
0  NO printing. \\
1  print atomic derivative contributions at end. \\
\item[IOP(34)]  dump option.  \\
0  NO dumping. \\
1  usual system stuff. \\
2  dump derivative contributions from within shell loops. \\
\end{description}
\subsection{\sf OPTIONS for Link 702}
\begin{description}
\item[IOP(18)]  establish critical cut-offs within shell loops.  \\
0  use standard values. \\
N {\tt VTOL}=10**(-IOP(18)-3) \\
Note: this is a 'use at own risk' option, and hence is not documented
fully. briefly, setting this option may speed things up, but it
can also sometimes give unpredictable results.  \\
\item[IOP(27)]  file initialization control.  \\
1  read in previous derivative contributions 
from file {\tt IRWFX} before computing anything.  \\
\item[IOP(28)]  skip option to defer integral evaluation  
to L703.  \\
0  compute as normal. \\
1  do all gradient integrals in L703. \\
\item[IOP(34)]  dump option.
0  NO dumping. \\
1  usual system stuff. \\
2  dump derivative contributions from within shell loops. \\
\end{description}
\subsection{\sf OPTIONS for Link 703}
\begin{description}
\item[IOP(27)]  file initialization control. \\
1  the contributions computed in dphnix are added 
to previous information contained in read/write file IRWFX.  \\
\item[IOP(28)]  integral evaluation option.  \\
0  compute as normal. (sp done in L702, spd done here.) \\
1  do all gradient integrals in L703. \\
\item[IOP(33)]  print option.  \\
0  NO printing. \\
1  print final contributions to {\tt FXYZ}. \\
\item[IOP(34)]  dump option.  \\
0  NO dumping. \\
1  usual system stuff. \\
2  dump derivative contributions from within shell loops. \\
options for Link 716
\item[IOP(7)] use of internal coordinates  \\
0  YES \\
1  NO \\
\item[IOP(27)]. does L702 read previous force information  \\
0  NO \\
1  YES \\
\item[IOP(30)]  use of symmetry in OVERLAY 7  \\
0  use (subject to availability). \\
1  don't use \\
\end{description}
\section{\sf OVERLAY 8}
The purpose of this OVERLAY is to do the transformation of the
two-electron integrals from the AO to MO basis. 
\begin{description}
\item[Link 801]  this Link prepares the MO coefficients and the one-electron
energies for the post-SCF programs as follows:  a selected set of
MOs (selected using IOP(10)) is written onto read/write file number
ISPECT.  Some
quantities commonly used in post-SCF routines are evaluated and
written on read/write file number INFORB.
\item[Link 802]  this Link does the 2-electron integral transformation for
RHF systems.
\item[Link 803]  this Link does the 2-electron integral transformation for
UHF systems.
\end{description}
\subsection{\sf OPTIONS for OVERLAY 8}
\begin{description}
\item[IOP(5)]  RHF or UHF \\
0  read in from ilsw \\
1  RHF \\
2  UHF \\
\item[IOP(6)]  specifies which single-bar integrals are to be
computed. \\
0  (ia|jb) \\
1  (ia|jb), (ij|ab) \\
2  (ia|jb), (ij|ab), (ij|kl) \\
3  (ia|jb), (ij|ab), (ij|kl), (ij|ka) \\
4  (ia|jb), (ij|ab), (ij|kl), (ij|ka), (ia|bc) \\
5  (ia|jb), (ij|ab), (ij|kl), (ij|ka), (ia|bc), (ab|cd) \\
in terms of what can be done with what integrals: \\
0  MP2 \\
2  MP3, CI \\
\item[IOP(7)]  test SCF convergence flag \\
0  YES \\
1  NO \\
\item[IOP(8)]  option to change the core available \\
0  {\tt MDV} is set to default value.(see {\tt DATA} statement) \\
1  read in {\tt MDV} (i6 format) \\
\item[IOP(9)]  use {\tt TRCL80} always.(mainly a debugging option) \\
0  NO \\
1  YES. \\
\item[IOP(10)]  window is selected as follows: \\
0  all molecular orbitals are taken \\
1  the core MOs are frozen \\
2  a record is read in (2i3) indicating the start and the end 
\item[IOP(30)]  the molecular orbitals outside the window are set to zero,
thus simulating the window without changing anything else (for
test purposes). \\
0  NO \\
1  YES \\
\item[IOP(31)]  perform primitive post-SCF operations \\
0  none \\
1  CI \\
\end{description}
\section{\sf OVERLAY 9}
\begin{description}
\item[Link 901]
Conversion of the set of single-bar integrals provided by the
AO-to-MO transformation routine to the packed set of
double-bar integrals and a-coefficients. this program also
calculates the second order Moller-Plesset perturbation
energy {\tt E(MP2)}.
\item[Link 909]
Iterative solution of the CI equations involving all single and
double substitutions and Moller-Plesset perturbation theory at
second and third orders.
partitioned into 5 Links: {\tt CIS1},...,{\tt CIS5}
{\tt CIS1} sets up the information needed for the matrix multiplication
{\tt W=V*A} , {\tt CIS2} to {\tt CIS4} then perform this matrix multiplication,
and {\tt CIS5} finally evaluates all sorts of correlation energies.
\end{description}
\subsection{\sf OPTIONS for OVERLAY 9}
\begin{description}
\item[IOP(5)]  method  \\
1  CID. CI with all double substitutions. \\
2  MP3. third order perturbation theory. \\
\item[IOP(6)]  criteria for termination of the iteration \\
0  default convergence criterion and maxcycle \\
N (N=1,...,6) perform max. N cycles
use default convergence criterion  \\
7  read in maxcycles and convergence criterion (i2,d18.13) \\
\item[IOP(7)]  update the energy in {\tt COMMON/GEN/} \\
0  YES, with the correlation energy, ECI in CI 
and EUMP3 in MP3 calculations  \\
1  YES, with EUMP3. \\
7  NO \\
\item[IOP(18)]  iteration scheme:   \\
Formation of DE in
\[
 A(S) = \frac{W(S)}{(DE - \Delta(S))}
\]
i.e. in the formation of a new wave function.  \\
0  use DE depending on the method used. (IOP(5)). \\
For method = 0 or 1  DE = W(0)/A0  \\
for method greater than 1  DE = 0  \\
Note that for perturbation methods (method=2,3,4,5)
DE is not really needed since the wave function formed is
never  used.  \\
1  W(0)/A0 always. \\
2  0. always. \\
\item[IOP(19)]  inhibit extrapolation  \\
0  NO \\
1  YES \\
\item[IOP(20)]  {\tt CUTOFF} for AO integrals  \\
0  standard {\tt CUTOFF} (10**(-6)) \\
1  {\tt CUTOFF} read in (format(G20.14)) \\
\item[IOP(25)]  print pair contribution and weight to correlation energy  \\
0  NO \\
1  YES, at the end of CI \\
2  YES, at each cycle \\
3  YES, at one cycle given by input (i3) \\
4  YES, at first cycle and at end \\
\item[IOP(26)]  normalization of the wavefunction  \\
0  normalized to {\tt A(0)=1}. \\
1  $\sum_s A(s)^2 = 1$ (all s) \\
Note: perturbation theoretical results are valid with NORM=0 only
\item[IOP(28)]  printing of dominant configurations.  \\
0  do not scan the 'A' vector at the end of CI. \\
1  scan the 'A' vector at the end of CI and print the 
dominant configurations.  \\
\item[IOP(30)]  calculation of the CI density matrix.  \\
0  do not calculate the density matrix. \\
1  calculate the density matrix at the end of CI. \\
\end{description}

