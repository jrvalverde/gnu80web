\chapter{\sf Banana Skins}
\label{bananas}
\section{\sf Non-existent Functions}
It is relatively easy to list the limitations of the {\tt gnu80}
system; maximum number and type of basis functions, maximum
amount of integral storage, type of quantum mechanical model ate. etc.
What are not quite so obvious are the limitations of some of the 
underlying {\em theoretical methods}.

Tecnically, for example, the basic {\bf HF} models are the
equations resulting from using the {\em parametric} variation method
on the Hartree-Fock Energy functional; the full (functional) variation
method in which {\em arbitrary} variations are allowed in the form of the MOs
has been replaced by the variation of a finite set of linear parameters,
the MO coefficients. This is done for entirely practical reasons and
in most cases the errors arising from this limitation of the
variation method are entirely {\em quantitative}; the minimum energy
of the Hartree-Fock limit is not reached. 

However, the parametric variation principle makes the all-important
{\em assumption} of the {\em existence} of the solutions. A closed-shell
calculation of the electronic structure of the hydride anion proceeds
smoothly to a minimum which is, unfortunately, an artifact of
the linear expansion method and the closed-shell model. The energy of
this anion is higher than that of the hydrogen atom and so the HF
solution of lowest energy for the hydride anion should be a hydrogen atom
and a free electron. Using the Different Orbitals for Different Spins
(DODS or UHF) model will generate a solution which is as close to this
latter situation as the basis allows but, {\em unless very diffuse
orbitals have been included in the basis} to simulate a free-electron
function, the true minimum is not accessible to the linear expansion
method; this is a {\em qualitative} error.
\begin{center}
\fbox{
\parbox{3.5in}{
\bf The General rule is: do not attempt to use
the facilities of {\tt gnu80} to calculate
``Approximations to functions which do not exist'' .
}
}
\end{center}
\section{\sf Anions}
The largest class of system for which the Hartree-Fock solutions
do not exist (their energies are higher than that of the neutral
molecule) are anions. Very, very few anions carrying a single negative
charge have bound-state Hartree-Fock solutions and {\em no}
multiply-charged anions have bound solutions. This is in the
Hartree-Fock limit, any approximations due to finite bases will,
of course, make this result stronger.

As in the case of the hydride anion mentioned above, use of the {\bf HF}
model with a standard basis will often generate a bogus approximation
to the non-existent bound state.
\section{\sf Dissociation Limits}
It is well-known that the closed-shell MO method, with its
insistence on sets of {\em doubly-occupied} Molecular Orbitals will
often give the wrong description of a molecule as the bond lengths
increase. Typically, the closed-shell {\bf HF} model will predict
that a bond dissociate to {\em ionic} fragments  rather than
neutral species. Often the use of the {\bf UHF} model in these cases
will solve the problem, provided that the program is flexible enough
to allow the use of the {\bf UHF} model with an even (nominally
paired) number of electrons.
\section{\sf Spin Eigenfunction Constraints}
There is a wealth of practical/computational experience which shows
that the imposition of a spin eigenfunction constraint on a
single-determinant wavefunction often constrains the action of the
variation principle so badly that completely spurious results are
obtained. 
The classic case in the literature is that of the 
optimisation of the structure of the allyl radical which has
a single unpaired electron in the conventional picture.
If a single determinant wave function has this constraint {\em imposed}
throughout the structure optimisation, there is an artificial
cusped maximum in the energy at the observed structure and a pair of
equivalent minima having one long and one short C-C bond. 

If the constraint is removed (i.e. a UHF function is used) the
optimisation generates the correct qualitative structure and the quantitative
result depends, as usual, on the quality of the basis.

The recommendation is:
\begin{center}
\fbox{
\parbox{3.5in}{
\bf Always use the UHF model if there is
any suspicion of qualitative errors in the
structure.
}
}
\end{center}


