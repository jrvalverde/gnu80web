\newpage
\setcounter{page}{1}
\markboth{\bf TREAD}{\bf TREAD}
\begin{description}
\item[NAME] TREAD \\
LCAOMOSCF calculations
\item[SYNOPSIS] \ \\
{\tt
   subroutine TREAD(FILE,X,M,N,MM,NN,K) \\
   double precision X(1) \\
   integer FILE, M, N, MM, NN, K \\
}
\item[DESCRIPTION] \ \\
TREAD reads data from the gnu80 internal file number {\tt FILE}
into the double precision array {\tt X}. In the calling segment
X must be {\tt DIMENSION}ed {\tt MM} by {\tt NN}. The data read
is stored in this array as far as {\tt M} by {\tt N} consistent
with FORTRAN matrix storage rules. {\tt K} indicates whether or
not the matrix has been stored in a compressed mode (for symmetric
matrices) i.e. contains only M(M+1)/2 elements, not M**2.
The routine is called to read actual matrices from files {\em and}
to read COMMON blocks in which case usually {\tt N=NN=1}
and {\tt M=MM=} length of file (in units of double precision reals,
padded out if necessary)
\item[ARGUMENTS:] \ \\
\begin{description}
\item[FILE]  The gnu80 internal file number
\item[X] Array to receive the data read.
\item[M] Actual number of rows in read matrix.
\item[N] Actual number of columns in read matrix.
\item[MM] Number of rows in the DIMENSION statement of calling program.
\item[NN] Number of columns in the DIMENSION statement of calling program.
\item[K] {\tt K = 0} means that all the matrix is read, {\tt K = 1}
means only ``half'' was read.
\end{description}
\item[SEE ALSO] \\ \\\\
TWRITE, TQUERY, NTRAN, FILEIO
\item[DIAGNOSTICS]
None; but NTRAN tracks errors
\end{description}
