\chapter{\sf Standard Nomenclature}
\label{app1a}
\section{\sf The Hamiltonian}
The molecular Hamiltonian used in the energy calculations
of standard quantum chemistry is always the non-relativistic, Born-Oppenheimer
``electrostatic'' Hamiltonian, which is, in
atomic units:
\begin{equation}
\hat{H} ( \vec{r_1}, \vec{r_2}, \ldots , \vec{r_n} ) = 
\sum_{i=1}^{n} { \hat{h} (\vec{r_i}) } + 
\sum_{i=1}^{n} { \sum_{j < i} { \frac{1}{ r_{ij} } } }
\label{H}
\end{equation}
Here, there are assumed to be $n$ electrons in the molecule and their
position vectors are $\vec{r_i}$.
Each electron has a {\em One-electron Hamiltonian} of identical
form:
\begin{equation}
\hat{h} (\vec{r_i}) = - \frac{1}{2} \nabla^2 (\vec{r_i})
- \sum_{A=1}^{N} { \frac{Z_A}{ | \vec{r_i} - \vec{r_A} | } }
\label{h}
\end{equation}
where the position vectors of the nuclei are $\vec{r_A}$ and their
charges are $Z_A$, and it issumed that there are $N$ of them.

The electron repulsion terms are simply the Coulomb repulsions
between unit like charges:
\[
\frac{1}{r_{ij}} = \frac{1}{ | \vec{r_i} - \vec{r_j} | }
\]
the summation is over all {\em distinct} pairs and $i \not= j$ of course.

The associated Schr\"{o}dinger equation:
\begin{equation}
\hat{H} ( \vec{r_1}, \vec{r_2}, \ldots , \vec{r_n} ) 
\Psi ( \vec{x_1}, \vec{x_2}, \ldots , \vec{x_n} ) = E 
\Psi ( \vec{x_1}, \vec{x_2}, \ldots , \vec{x_n} ) 
\label{se}
\end{equation}
in which the many-electron wavefunction 
$\Psi ( \vec{x_1}, \vec{x_2}, \ldots , \vec{x_n} )$ depends on the
{\em spatial} and {\em spin} ``co-ordinates'' of the electrons.
The collection of three spatial co-ordinates ($\vec{r_i}$) and one
spin variable is written as $\vec{x_i}$. 
This equation cannot be solved and {\em ab initio} methods are designed 
to generate {\em approximate
and model} solutions of \ref{se} by a variety of variational and
perturbation techniques.
\section{\sf Many-electron Wavefunctions}
\label{wf}
The many-electron wavefunction is approximated by a linear combination
of {\bf Slater Determinants} 
$\Phi_K ( \vec{x_1}, \vec{x_2}, \ldots , \vec{x_n} )$, 
each of which is the anti-symmetrised
product of $n$ {\bf Spin-Orbitals} $\chi (\vec{x_i})$ depending
on the space and spin variables of just one electron; for example
a determinant constructed from the first $n$ spin-orbitals 
$\chi_1 \ldots \chi_n$ is:
\begin{center}
\[
\Phi (\vec{x_1},\vec{x_2},\ldots , \vec{x_n} ) = 
\frac{1}{\sqrt{n!}}
\left |
\begin{array}{cccc}
\chi_1 (\vec{x_1}) & \chi_1 (\vec{x_2}) & \dots  & \chi_1 (\vec{x_n})  \\
\chi_2 (\vec{x_1}) & \chi_2 (\vec{x_2}) & \dots  & \chi_2 (\vec{x_n})  \\
\chi_3 (\vec{x_1}) & \chi_3 (\vec{x_2}) & \dots  & \chi_3 (\vec{x_n})  \\
\ldots & \ldots & \ldots & \ldots \\
\ldots & \ldots & \ldots & \ldots \\
\ldots & \ldots & \ldots & \ldots \\
\chi_{n-2} (\vec{x_1}) & \chi_{n-2} (\vec{x_2}) & \dots  & 
\chi_{n-2} (\vec{x_n})  \\
\chi_{n-1} (\vec{x_1}) & \chi_{n-1} (\vec{x_2}) & \dots  
& \chi_{n-1} (\vec{x_n})  \\
\chi_n (\vec{x_1}) & \chi_n (\vec{x_2}) & \dots  & \chi_n (\vec{x_n})  
\end{array}
\right |
\]
\end{center}
The factor $1/\sqrt{n!}$ normalises $\Phi$ if the $\chi_i$ are
an orthonormal set.

The number and construction of these determinants defines a {\bf model}
of molecular electronic structure and the accuracy with which
the spin-orbitals may be computed is defined by practical factors in
the system, in general
\begin{equation}
\Psi ( \vec{x_1}, \vec{x_2}, \ldots , \vec{x_n} ) 
\approx \sum_{K=0}^{M} 
{ A_K \Phi_K ( \vec{x_1}, \vec{x_2}, \ldots , \vec{x_n} ) }
\label{wftilde}
\end{equation}
The {\bf Hartree-Fock} model throws all its effort into obtaining the
best possible {\em one term} expansion; $A_0 = 1$, $A_K = 0$ for
$ K > 0 $. The {\bf Configuration Interaction} and
{\bf Moller-Plesset} methods improve on this single-term model by
extending the expansion using the {\em virtual} orbitals generated
by the Hartree-Fock variational procedure as a bye-product.
\section{\sf Spin-Orbitals}
\label{spinorb}
There are just two possible spin ``functions'' conventionally
written $\alpha$ and $\beta$ and usually the
generation of spin-orbitals which are {\em mixtures} of these two functions
is not considered.
Thus the computational problem is the determination of the
{\bf spatial orbitals} which are the spatially dependent factors
of the spin-orbitals $\chi_i$:
\begin{eqnarray*}
\chi_i (\vec{x}) = \psi_{i'} (\vec{r}) \alpha \\
{\mbox \rm or  }\\
\chi_i (\vec{x}) = \psi_{i'} (\vec{r}) \beta
\label{spaceorb}
\end{eqnarray*}
In many applications it is useful to have a more compact notation
for the relationship between the spin-orbitals and the
spatial orbitals since the spin function is just  a label; the
``bar'' and ``no bar'' notation is used:
\begin{eqnarray*}
\psi_i = \psi_i \alpha \\
\overline{\psi}_i = \psi_i \beta
\end{eqnarray*}
That is, the notation $\psi_i$ may mean the spatial orbital
{\em or } the $\alpha$ spin-orbital with spatial factor $\psi_i$
according to context. Sometimes care must be taken to 
distinguish the two cases.
\section{\sf Linear Expansions for the Spatial Orbitals}
Each  spatial molecular orbital $\psi_i$, a function of ordinary 
three-dimensional space,
is expanded as a linear combination of {\bf Basis
Functions} which are fixed for a particular calculation and
are chosen on a variety of theoretical and (mostly) practical
grounds. These basis functions ($\phi_k (\vec{r})$) are key
elements in the success of any calculation of molecular electronic
structure:
\begin{equation}
\psi_i (\vec{r}) = \sum_{k=1}^{m} { \phi_k (\vec{r}) C_{ki} }
\label{lcao}
\end{equation}
Where there are $m$ basis functions with which to expand
the $n$ optimum molecular orbitals. In view of the relationship between
spatial and spin orbitals, not all the $n$ spatial molecular orbitals
need be different; sometimes there will be {\em pairs} which
are the same and the associated spin-orbitals only differ in
spin factor.
The Hartree-Fock variational method optimises the linear
coefficients $C_{ki}$ to ensure the best possible (lowest energy)
description of the molecular system.

For reasons which are entirely practical, the basis functions
must be {\bf Gaussian Functions}, functions which have a factor
\[
exp( -\alpha |\vec{r}|^2)
\]
as part of their functional form. Since the ``natural''
atomic orbitals have a dependence like $exp(-\zeta |\vec{r}| ) $
there has to be a much longer expansion in terms of Gaussians to
ensure an accurate molecular orbital is computed, typically two or three
times the length of a Slater orbital expansion.

The use of Gaussian functions {\em directly} in
\ref{lcao} would therefore make excessive requirements of
storage for the electron-repulsion integrals and so a compromise
is used whereby the length of the explicit expansion in \ref{lcao}
is restricted by taking the basis functions themselves to
be {\em fixed} linear combination of so-called {\bf Primitive}
Gaussians:
\begin{equation}
\phi_k (\vec{r}) = \sum_{j=1}^{n_k} { \eta_j (\vec{r}) d_{jk} }
\label{lcprim}
\end{equation}
Here the length of the expansion may depend on the basis
function in question, so that the {\bf Degree of Contraction}
$n_k$ depends on $k$.
\section{\sf Primitive Gaussians}
\label{prims}
The general form of a primitive Gaussian function is usually
chosen to be the product of a Cartesian Factor and an exponential:
\begin{equation}
\eta(\vec{r}) = N x^{\ell} y^{m} z^{n} exp(- \alpha r^2 )
\end{equation}
where $ r = | \vec{r} |$ and $\ell, m, n$ are integers which characterise
the {\em type} or {\em order} of the Gaussian function. $N$ is a numerical
factor chosen to {\em Normalise} the function to unity, clearly depending
on $\alpha, \; \ell, \; m, \; $ and $n$.

If a Gaussian primitive is expressed in terms of a global co-ordinate
system, the components of the position vector of the centre on which
it is based appear in an obvious way.
\begin{equation}
\eta_j (\vec{r}) = N( \ell_j , m_j , n_j ; \alpha_j ) 
(x-x_A )^{\ell_j} ( y - y_A )^m_j ( z - z_A )^n_j 
exp( \alpha_j | \vec{r} - \vec{r_{A}} |^2 )
\end{equation}
where the explicit dependence of the primitive on the position
of its nucleus is given; $\vec{r_A} =(x_A , y_A , z_A )$ is the position
vector of centre $A$. The subscript $j$ serves to identify this
particular $\eta_j$ among the many.

The type of this primitive is given by 
\[
t = \ell_j + m_j + n_j
\]
and a terminology related to the familiar atomic orbitals is used:
\begin{description}
\item[t = 0] an s-type Gaussian or zeroth-order Gaussian
\item[t = 1] a p-type Gaussian or first-order Gaussian
\item[t = 2] a d-type Gaussian or second-order Gaussian
\item[t = 3] an f-type Gaussian or third-order Gaussian
\end{description}
In the first two cases (s and p) there is a direct correspondence
between the Gaussians and the real Atomic Orbitals. For d,f and
higher Gaussians there are more Cartesian factors of a given type than
real Atomic Orbitals of the corresponding angular momentum 
($t$ is equal to the total angular momentum quantum number usually
written as $\ell$ in atomic theory).


This technique of retaining some of the Gaussian primitives in
{\em fixed} linear combinations is called {\bf Contraction}.
Of course, the calculations using all the primitives still have to be performed,
but the advantage gained by using so-called contracted basis functions
is that {\em storage} is saved. The price to be paid for the
contraction technique is loss of variational flexibility; only
the linear combination coefficients of the {\em basis functions}
are optimised, not the coefficients of each primitive.
\section{\sf Single Determinant Energy Expression}
The mean value of the energy of a single determinant of
orthonormal (orthogonal and normalised) molecular orbitals 
\[
\int { \chi_i (\vec{x})\chi_i (\vec{x}) d \tau } = \delta_{ij}
\]
is
\begin{equation}
E[\Phi ] = \int { \Phi^* (\vec{x_1},\vec{x_2}, \ldots ,\vec{x_n} )
\hat{H}(\vec{r_1},\vec{r_2}, \ldots ,\vec{r_n} )
\Phi (\vec{x_1},\vec{x_2}, \ldots ,\vec{x_n} ) d\tau_1 d\tau_2 \ldots d\tau_n }
\end{equation}
\begin{center}
\fbox{
\parbox{3.5in}{
Integration over spin {\em and} space variables
is denoted by
\[
\int \ldots d \tau \,\,\, or \,\,\, \int \ldots d\tau_1
\]
while integration over space {\em only} is written
\[
\int \ldots dV \,\,\, or \,\,\, \int \ldots dV_1
\]
That is:
\[
\int \ldots d \tau = \int \int \ldots dV ds
\]
where $s$ is the spin ``variable'' .
That is, $dV$ refers to the three spatial variables in $\vec{r}$
and $d \tau$ to the four variables in $\vec{x}$.
}
}
\end{center}

Completion of the integration, by separation and use of
the specific form of $\hat{H}$ (Equation \ref{H}), 
together with the orthonormality
relationships reduces this integral (over $3n$ spatial variables
and $n$ spin ``variables'' ) to:
\begin{equation}
E = \sum_{i=1}^{n} { h_{ii} }
+ \sum_{i=1}^n { \sum_{j \leq i}^{n} {
( J_{i,j} - K_{ij} ) } }
\end{equation}
Where
\[
h_{ii} = \int { \chi_i^* (\vec{x}) \hat{h}(\vec{r}) \chi_i (\vec{x}) d\tau }
\]
$\hat{h}$ is the one-electron hamiltonian \ref{h} and 
\begin{equation}
J_{ij} = < i j | i j > = 
\int { d\tau_1 \int { d\tau_2 \chi_i^* (\vec{x_1})\chi_j^* (\vec{x_2})
\left ( \frac{1}{r_{12}} \right )
\chi_i (\vec{x_1})\chi_j (\vec{x_2}) } }
\end{equation}
\begin{equation}
K_{ij} =  
\int { d\tau_1 \int { d\tau_2 \chi_i^* (\vec{x_1})\chi_j^* (\vec{x_2})
\left ( \frac{1}{r_{12}} \right )
\chi_j (\vec{x_1})\chi_i (\vec{x_2}) } }
\end{equation}

The physical interpretation of the terms is straightforword.
The integrals $h_{ii}$ are the energy of an electron moving
in the attractive field of the nuclei {\em alone} (i.e. in
the absence of the other electrons). The integrals $J_{ij}$
are the mean repulsions between electrons occupying $\chi_i$
and $\chi_j$; they are called ``Coulomb'' terms for this reason.
The integrals $K_{ij}$ arise from the anti-symmetry of the wavefunction
and have no srict classical analogue but their principle function
in the energy expression is to cancel out the ``self-repulsion''
which would be included if they were not present (notice the summation
{\em includes} the term $i=j$). The $K_{ij}$ are called
``Exchange Integrals'' because of the way they arise in
the mathematics of the expansion of the determinant.
\section{\sf Notation for Repulsion integrals}
It is usual to write the electron repulsion terms in the 
single-determinant energy expression as special cases of a
more general repulsion integral e.g:
\begin{equation}
K_{ij} = < i j | j i > = 
\int { d\tau_1 \int { d\tau_2 \chi_i^* (\vec{x_1})\chi_j^* (\vec{x_2})
\left ( \frac{1}{r_{12}} \right )
\chi_j (\vec{x_1})\chi_i (\vec{x_2}) } }
\end{equation}

The notation $< i j | i j >$ and $< i j | j i > $ has been
introduced in anticipation of a more general electron
repulsion integral $ < i j | k \ell > $:
\begin{equation}
 < i j | k \ell > = 
\int { d\tau_1 \int { d\tau_2 \chi_i^* (\vec{x_1})\chi_j^* (\vec{x_2})
\left ( \frac{1}{r_{12}} \right )
\chi_k (\vec{x_1})\chi_{\ell} (\vec{x_2}) } }
\label{ijkl}
\end{equation}
Integrals of this type appear in the energy expression of
multi-determinant wave functions in the ``cross terms''
between different determinants and are included here to 
define the notation.

The complex-conjugate notation has been given explicitly since,
although the primitives and basis functions are obviously all {\em real},
the expansion coefficients $C_{ki}$ may well be complex in some
applications; leading to complex $\chi_i$.

There are a variety of notations for the repulsion integrals
either singly or in ``standard combinations'' , each
has its own logic and there is no overwhelming reason to
choose one or the other. It is sometimes more intuitively
acceptable to use the physical interpretation of these integrals
as the net repulsion between a charge distribution 
$\chi_i (\vec{x_1})\chi_k (\vec{x_1})$ and 
$\chi_j (\vec{x_2})\chi_{\ell} (\vec{x_2})$
as a justification for the ``charge-cloud'' notation:
\begin{equation}
( i k , j \ell ) = < i j | k \ell >
\end{equation}
where {\em round brackets} and {\em no} vertical bar distinguish
between the two notations. The charge-cloud notation is particularly
useful when the integrals are over {\em basis functions} not
molecular orbitals since these are always real and the various possible
permutations of $i, j, k, \ell $ which do not change the
value of $(ik, j \ell )$  are easier to see, in fact, if the 
functions {\em are} real then interchanging $i$ and $k$, or
$j$ and $\ell$, or the {\em pairs} $(i,k)$ and $(j \ell )$
do not change the value of $(ik, j \ell )$ as can be seen from the 
definition
(equation \ref{ijkl}).

Integrals like \ref{ijkl} usually occur in {\em pairs} in the
evaluation of energy integrals from determinantal wavefunctions
and it is often convenient to use a single symbol to
mean a Coulomb integral and its corresponding Exchange integral
although the difference between the two terms is no longer valid
if $ i \not= k$ and $j \not= \ell$:
\begin{equation}
<i j || k \ell > = < i j | k \ell > - < i j | \ell j >
\label{doubar}
\end{equation}
The double bar serving to indicate that the ``exchange''
term has been included.
\section{\sf Spatial Orbital Repulsion Integrals}
Throughout the previous two sections the electron-repulsion
integrals have been expressed in terms of the {\em spin-orbitals}
$\chi_i (\vec{x})$, but, since the ``integration'' over
the spin is trivial, it is always possible to reduce integrals like
\ref{ijkl} to a ``genuine'' integration over space (six
dimensional since there are two particles involved). The electron-repulsion
operator $1/r_{12}$ does not involve spin so that the spin integration can
always be separated into factors which are zero or one depending of the
spin factors involved:
\begin{eqnarray}
\int { \alpha^* \alpha ds } = 1 \\
\int { \beta^* \beta ds } = 1 \\
\int { \alpha^* \beta ds } = 0 \\
\int { \beta^* \alpha ds } = 0 
\label{spinints}
\end{eqnarray}
so that any integral $<i j | k \ell >$ which contains
spin-orbitals $\chi_i$ and $\chi_k$ which have
{\em different} spin factors will be {\em zero} as will
any integral with different spin factors in $\chi_j$ and
$\chi_{\ell}$.

Thus, if the spin factors in the pairs $\chi_i , \chi_k$
and $\chi_j , \chi_{\ell} $ are the {\em same} (and only then)
the integral
\begin{equation}
 < i j | k \ell > = 
\int { d\tau_1 \int { d\tau_2 \chi_i^* (\vec{x_1})\chi_j^* (\vec{x_2})
\left ( \frac{1}{r_{12}} \right )
\chi_k (\vec{x_1})\chi_{\ell} (\vec{x_2}) } }
\end{equation}
becomes
\begin{equation}
 < i' j' | k' \ell ' > = 
\int { dV_1 \int { dV_2 \psi_{i'}^* (\vec{r_1})\psi_{j'}^* (\vec{r_2})
\left ( \frac{1}{r_{12}} \right )
\psi_{k'} (\vec{r_1})\psi_{\ell '} (\vec{r_2}) } }
\end{equation}
where $\psi_{i'}$ is the spatial factor in the spin-orbital
$\chi_i$ etc.

This fact reduces the number of repulsion integrals, particularly
exchange integrals.
\section{\sf Basis Function Repulsion Integrals}
In the calculation of molecular electronic structure by the
basis function expansion method it is necessary to calculate
the {\em molecular orbital} repulsion integrals by calculating the
corresponding replusion integrals involving the {\em basis
functions} and using the linear combination coefficients to
generate the {\em molecular orbital} integrals. That is the integrals
\begin{equation}
 < i j | k \ell > = 
\int { dV_1 \int { dV_2 \phi_i^* (\vec{x_1})\phi_j^* (\vec{x_2})
\left ( \frac{1}{r_{12}} \right )
\phi_k (\vec{x_1})\phi_{\ell} (\vec{x_2}) } }
\end{equation}
Where the same symbol $< i j | k \ell > $ has been used to denote
an electron repulsion integral over the basis functions as was used
in the last section for these integrals over molecular orbitals. This is
standard practice and it is usually clear from the context
which integral is meant. If there is doubt then the more
explicit notation
\begin{equation}
 < \phi_i \phi_j | \phi_k \phi_{\ell} > = 
\int { dV_1 \int { dV_2 \phi_i^* (\vec{r_1})\phi_j^* (\vec{r_2})
\left ( \frac{1}{r_{12}} \right )
\phi_k (\vec{r_1})\phi_{\ell} (\vec{r_2}) } }
\end{equation}
may be used for the basis-function integrals and
\begin{equation}
 < \chi_i \chi_j | \chi_k \chi_{\ell} > = 
\int { d\tau_1 \int { d\tau_2 \chi_i^* (\vec{x_1})\chi_j^* (\vec{x_2})
\left ( \frac{1}{r_{12}} \right )
\chi_k (\vec{x_1})\chi_{\ell} (\vec{x_2}) } }
\label{ijklfull}
\end{equation}
for the molecular orbital integrals. In practice, it is usually
the basis-function integrals which are denoted by the simpler
form $< i j | k \ell > $ and, if necessary, the molecular orbital integrals
by the more explicit notation.

Notice that, since the basis functions are {\em spatial} functions
the basis-function electron-repulsion integrals involve no
spin integration; if one of them is zero it is because of
symmetry or simply because the two charge-clouds are very remote.
