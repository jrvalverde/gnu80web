\documentstyle[12pt]{article}
\begin{document}
\section{\sf NONDEF and IOP; gnu80 and g92}
The use of {\bf NON DEF}ault options in {\tt gnu80} or {\tt g92} is quite
common when performing calculations on systems of complex
and recalcitrant electronic structure.

At the moment (11/92), the method of {\em supplying} these non-default
options is different in {\tt gnu80} and {\tt g92} although attempts are
made to maintain the non-default options in {\tt gnu80} as a proper
subset of the ones available in {\tt g92}.
\subsection{\tt gnu80}
Users of {\tt gnu80} will be familiar with the method:
\begin{itemize}
\item
Use the {\bf NONDEF} Command on the Command Record; like
\begin{verbatim}
      # HF/3-21G,OPT,NONDEF
\end{verbatim}
\item
Then supply the non-default options in a separate input {\em section}
{\em i.e.} 
\begin{enumerate}
\item
one blank record to terminate the Command Record, 
\item
then
line(s) of non-default options, in the usual format
{\small
\begin{verbatim}
    Overlay Number/Option number=Value, [Option Number=Value,...];
\end{verbatim}
}
\item
then a terminating blank record. 
\end{enumerate}
For example;
\begin{verbatim}
      # HF/3-21G,OPT,NONDEF

      5/6=9;

        Very Tight Convergence requested in SCF by use
         of IOPT(6) = 9  (Convergence criterion 10**(-9))

      0 1
      Hg
      .....  etc  .....

\end{verbatim}
sets Option Number 6 equal to 9 in Overlay 5 (the SCF Overlay).
\item
Now continue with the Title and Z-matrix input sections as usual.
\end{itemize}
\subsection{\tt g92}
In {\tt g92} the whole thing is done on the Command Record, using the 
Command {\bf IOP}. The same rules apply for the naming conventions
etc. so that the above example becomes for {\tt g92}:
\begin{verbatim}
      $RunGauss
      # HF/3-21G,OPT,IOP(5/6=9)

        Very Tight Convergence requested in SCF by use
         of IOPT(6) = 9  (Convergence criterion 10**(-9))

      0 1
      Hg
      .....  etc  .....
\end{verbatim}
\subsection{\sf Harmonisation}
{\tt gnu80} will be changed to accept the {\tt g92}-style Command
but the {\bf NONDEF} command will be retained since it has some
advantages and also to provide downward compatibility with existing
job files. When the change has been done it will be announced; it is
not imminent.
\end{document}

