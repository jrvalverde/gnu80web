\chapter{\sf Standard Methods: Input/Output Examples}
\label{app2a}
\section{\sf Introduction}
\label{inex}
By far the best method of becoming familiar with the use
of {\tt gnu80} is by studying examples. In this section there
are a number of ``typical'' runs with some comments
and in the next section there are selected listings of the
output generated by these runs. Most of the examples use
Standard Methods. 

Throughout these examples the records are numbered,
this is simply for convenience of reference, the record numbers
{\em do not} form part of the input to {\tt gnu80}. Again, throughout
the examples a blank record is indicated by ` `(blank record)'' 
in order
to be free of any quirks of line-spacing in the manual; of course
``(blank record)'' must be replaced in the actual input
by a blank record!

Only a {\em few} examples of the output generated by these input
examples are given later as the output quickly becomes unwieldy
in a manual!

\newpage
\subsection{\sf Example 1 HF calculation at fixed Geometry}
{\tt 
\begin{enumerate}
\item \# HF/STO-3G
\item (blank record)
\item  WATER AT EXPERIMENTAL GEOMETRY
\item (blank record)
\item 0 1
\item O
\item H 1 0.956
\item H 1 0.956 2 104.5
\item (blank record)
\end{enumerate}
}
This is a straightforward (single-point) 
closed-shell (RHF) run on the (singlet) ground state
of the water molecule using the STO-3G basis which is also the default.  
No {\bf VARIABLES} are
introduced, so only Input Sections 1, 2 and 3 are needed.
Record 1 is the command record and the blank record
terminates  the job type input section.  Records 3 and 4 are
the title input section.  The remaining records (5, 6, 7, 8, 9) make up the
Molecule Specification Input Section. 
\subsubsection{\sf Alternative, Using the Element names for atom reference}
{\tt 
\begin{enumerate}
\item \# HF/STO-3G
\item (blank record)
\item Water at Experimental Geometry
\item (blank record)
\item 0 1
\item O
\item H O 0.956
\item H O 0.956 2 104.5
\item (blank record)
\end{enumerate}
}
This dataset will generate exactly the same run as the one above,
the only difference is that the hydrogen atoms' positions are
referred to the oxygen atom {\em by its symbol in the Z-matrix}
rather than by its number in the records of the Z-matrix.
Similar replacements may be made in all the datasets in the examples,
if this is a more attractive method of presentation.

The method of ``reference by symbol'' does have the advantage
of ``Z-matrix position invariance\'' ; adding or subtracting atoms to
the Z-matrix does not involve changes to existing records since they
are internally consistent. Notice that this method is best used together
with the precaution of giving all atoms in a given run a different 
symbol; in the above
example the two hydrogens have the same label. This will not
cause trouble in this particular case, of course.
\newpage
\subsection{\sf Example 2. Simple Geometry Optimisation}
{\tt 
\begin{enumerate}
\item \# HF/STO-3G OPT
\item (blank record)
\item WATER STO-3G STRUCTURE
\item (blank record)
\item 0 1
\item O
\item H 1 P
\item H 1 H 2 A
\item (blank record)
\item R=1.0
\item A=105.0
\item (blank record)
\end{enumerate}
}
This is an optimization where the O-H bond length R and the H-O-H bond angle
A are adjusted to minimize the HF/STO-3G energy.  Note that the 
{\bf VARIABLE}
R appears in {\em two} places in the Z-matrix (both O-H bonds).  
This means that the optimization
is {\em implicitly constrained} and that the two bondlengths 
are not allowed to take on separate
(different) values.  This use of {\bf VARIABLES} 
permits symmetry constraints to
be imposed {\em and maintained}
$(C_{2v}$ in this case).  This example has a {\bf VARIABLES} input
section (records 10, 11 and 12).
\newpage
\subsection{\sf Example 3 A General Optimisation}
{\tt 
\begin{enumerate}
\item   \# HF/STO-3G OPT
\item 
\item   WATER STO-3G STRUCTURE UNCONSTRAINED
\item 
\item   0 1
\item   0
\item   H 1 R1
\item   H 1 R2 2 A
\item 
\item   R1=0.9
\item   R2=1.1
\item   A=105.0
\item 
\end{enumerate}
}
This optimization is not constrained 
to retain $C_{2v}$ symmetry during the optimisation
because the two bond lengths are represented
by different {\bf VARIABLES} which are given different initial values.
\newpage
\subsection{\sf Example 4  Use of  CONSTANTS}
{\tt 
\begin{enumerate}
\item  \# HF/STO-3G OPT
\item (blank record)
\item  WATER STO-3G STRUCTURE
\item (blank record)
\item  0 1
\item  C
\item  H1 0 R
\item  H2 0 R H1 A
\item (blank record)
\item  R=1.0
\item (blank record)
\item  A=105.0
\item (blank record)
\end{enumerate}
}
This is similar to Example 2, except that (1) the Hydrogen nuclei are
numbered directly as H1 and H2 and (2) only the bond length is varied,
the angle A being treated as a constant.  The final section 
is an example of a {\bf CONSTANTS} Input Section.
\newpage
\subsection{\sf Example 5 A more General Z-matrix}
{\tt 
\begin{enumerate} 
\item  \# HF/3-21G OPT
\item (blank record)
\item  FLUOROMETHANE C3V 3-21G OPTIMIZATION
\item (blank record)
\item  0 1
\item  C
\item  F 1 CF
\item  H 1 CH 2 HCF
\item  H 1 CH 2 HCF 3  120.
\item  H 1 CH 2 HCF 3 -120.
\item (blank record)
\item  CF=1.38
\item  CH=1.09
\item  HCF=110.6
\item (blank record)
\end{enumerate}
}
This HF/3-21G optimization of Fluoromethane in $C_{3v}$ symmetry illustrates
the use of the {\em dihedral angle} for the second and third hydrogen atoms.
Note that mnemonic choice of names of {\bf VARIABLES}.
\newpage
\subsection{\sf Example 6 Use of a Dummy Centre}
{\tt 
\begin{enumerate}
\item  \# HF/3-21G OPT
\item (blank record)
\item  AMMONIA C3V OPTIMIZATION
\item (blank record)
\item  0 1
\item  N
\item  - 1 1.
\item  H 1 NH 2 HNX
\item  H 1 NH 2 HNX 3  120.
\item  H 1 NH 2 HNX 3 -120.
\item (blank record)
\item  NH=1.0
\item  HNX=70.
\item (blank record)
\end{enumerate}
}
This example illustrates the use of a Dummy centre (number 2) to fix the
three-fold axis in $C_{3v}$ ammonia.  Note that the ``radial''
position of the dummy on
the axis is irrelevant and that the distance 1.0 used could have been
replaced by any other positive number.  HNX is the angle between an NH
bond and the threefold axis.
\newpage
\subsection{\sf Example 7 Direct use of HNH angle}
{\tt 
\begin{enumerate}
\item   \# HF/3-21G OPT
\item 
\item   AMMONIA C3V 3-21G OPTIMIZATION
\item 
\item   0 1
\item   N
\item   H 1 NH
\item   H 1 NH 2 HNH
\item   H 1 NH 2 HNH 3 HNH 1
\item 
\item   N H=1.0
\item   HNH=106.0
\item 
\end{enumerate}
}
This is similar to Example 6 except that the HNH bond angle is used as a
{\bf VARIABLE}.  J=+1 for the third hydrogen permits it to be defined by
two bond angles.
\newpage
\subsection{\sf Example 8 Use of a Dummy atom to maintain ``symmetry''}
{\tt 
\begin{enumerate}
\item  \# HF/3-21G OPT
\item (blank record)
\item  OXIRANE C2V 3-21G OPTIMIZATION
\item (blank record)
\item  0 1
\item  X
\item  C1 X HALFCC
\item  C  X     UX    C1  90.
\item  C2 X HALFCC     C  90. C1 180.
\item  H1 C1    CH     X  HCC 0  HCCO
\item  H2 C1    CH     X  HCC 0 -HCCU
\item  H3 C2    CH     X  HCC 0  HCCO
\item  H4 C2    CH     X  HCC 0 -HCCO
\item (blank record)
\item  HALFCC=0.75
\item  OX=1.0
\item  CH=1.08
\item  HCC=130.
\item  HCCO=130.
\item (blank record)
\end{enumerate}
}
This Example illustrates two points.  First, a dummy centre is placed at the
center of the C-C bond to help {\em constrain} the CCO triangle to be isoceles.
OX is then the perpendicular distance from 0 to the C-C bond and the angles
OXC are held at 90.0  .  Second, note that some of the entries in the
Z-matrix are represented by the {\em negative} of the dihedral angle variable
HCCO.  This is useful in symmetry-constrained situations like this.
\newpage
\subsubsection{\sf Examples 9 and 10 Bond Angle Limitations}
If {\bf FORCE} or {\bf OPT} runs are carried out, {\tt gnu80} is unable to handle
bond angles of 180  which occur in linear molecular fragments.  Examples
are linear acetylene and the C4 chain in butatriene.  Difficulties
may also be encountered in nearly linear situations such as ethynyl groups
in unsymmetrical molecules.  These sutuations can be avoided by introducing
dummy centres along the angle bisector and using the half-angle as the
{\bf VARIABLE} or {\bf CONSTANT}.  This is illustrated in example 9 and 10. \\
\subsection{\sf Example 9 Linear HCN}
{\tt 
\begin{enumerate}
\item  \# HF/6-31G* OPT
\item (blank record)
\item  HYDROGEN CYANIDE LINEAR 6-31G* OPTIMIZATION
\item (blank record)
\item  0 1
\item  N
\item  C 1 CN
\item  - 2 1. 1 90.
\item  H 2 CH 3 90. 1 180.
\item (blank record)
\item  CN=1.20
\item  CH=1.06
\item (blank record)
\end{enumerate}
}
\newpage
\subsection{\sf Example 10 Use of a Half-Angle}
{\tt 
\begin{enumerate}
\item   \# HF/3-21G OPT
\item  (blank record)
\item   NCOH 3-21G OPTIMIZATION
\item  (blank record)
\item   0 1
\item   N
\item   C 1 CN
\item   - 2 1. 1 HALF
\item   0 2 CO 3 HALF 1 180.
\item   H 4 OH 2  COH 3   0.
\item  (blank record)
\item   CN=1.2
\item   CO=1.3
\item   OH=1.0
\item   HALF=80.
\item   COH=105.
\item  (blank record)
\end{enumerate}
}
In this optimization, HALF represents half of the NCO angle which is expected
to be close to linear.  Note that the position of the dummy centre is chosen
so that a value of HALF less than 90  corresponds to a (slightly) Cis
arrangement.
\newpage
\subsection{\sf Example 11 A preparation run for an Open-Shell species}
{\tt 
\begin{enumerate}
\item  \# GUESS=ONLY
\item (blank record)
\item  AMINO RADICAL  TEST OF INITIAL GUESS
\item (blank record)
\item  0 2
\item  N
\item  H 1 NH
\item  H 1 NH 2 HNH
\item (blank record)
\item  NH=1.03
\item  HNH=120.0
\item (blank record)
\end{enumerate}
}
This is a short test of the $NH_2$  radical to find whether any
{\bf ALTER} commands are needed to generate a desired electronic state.
Note that {\bf HF/STO-3G} is used by default since the command record
is not null.
\newpage
\subsection{\sf Example 12 Use of Example 11 in a Production Run}
{\tt 
\begin{enumerate}
\item  \# HF/STO-3G ALTER OPT
\item (blank record)
\item  AMINO RADICAL  STO-3G STRUCTURE OF 2-A1 STATE
\item (blank record)
\item  0 2
\item  N
\item  H 1 NH
\item  H 1 NH 2 HNH
\item (blank record)
\item  NH=1.03
\item  HNH=120.
\item (blank record)
\item (blank record)
\item  4 5
\item (blank record)
\end{enumerate}
}
This example finds the {\bf UHF/STO-3G} structure of the $\ ^2A_1$ 
excited state of the
amino radical.  Examination of the output from the previous example indicates
that a $\beta$-electron has to move from orbital 4 to orbital 5 to obtain the
correct initial configuration.  Note than an extra blank record  is
necessary to indicate an empty $\alpha$ {\bf ALTER} section. 
\newpage
\subsection{\sf Example 13 Moller-Plesset Run}
{\tt 
\begin{enumerate}
\item  \# MP3/6-31G**
\item (blank record)
\item  WATER MP3/6-31G** AT HF/6-31G* GEOMETRY
\item (blank record)
\item  0 1
\item  O
\item  H 1 P
\item  H 1 F 2 A
\item (blank record)
\item  P=0.947
\item  A=105.5
\item (blank record)
\end{enumerate}
}
\newpage
\subsection{\sf Example 14 Configuration Interaction Run}
{\tt 
\begin{enumerate}
\item   \# CID/6-31G** NOPOP
\item 
\item   WATER CID/6-31G**  AT HF/6-31G* GEOMETRY
\item 
\item   0 1
\item   O
\item   H 1 R
\item   H 1 R 2 A
\item 
\item   R=0.947
\item   A=105.5
\item 
\end{enumerate}
}
This example does a single point run at the CID/6-31G*//HF/6-31G*
level of theory.
