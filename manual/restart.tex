\documentstyle[12pt]{article}
\begin{document}
\section{Restarting a gnu80 Run}
\label{restart}
Usually, when a gnu80 job exits (either normally or abnormally), it
leaves potentially useful data on the disk in the form of read-write
files, two-electron integrals,  or post-SCF buckets.  

The {\bf RESTART}
feature allows a subsequent gnu80 job to recover and use these data,
provided they have been placed in non-scratch files. In the normal course
of events, gnu80 uses scratch files for both electron-repulsion
integral storage and general file use. It is possible, by the
use of {\em file control records} to make these files permanent.
There are two files, the ``integral'' file (for repulsion
integrals) and the ``read-write'' file (for general data
storage, inter-link communication etc.etc.).

The files are made permanent by placing one or two file control records
{\bf before} the Job Control Record in a gnu80 job input.
They both have the ` \% ' symbol in the first column of the record
and have the form:
\begin{description}
\item[\%int=filename1] 
\item[\%rwf=filename2] 
\end{description}
where ``filename1'' and ``filename2''
are two (different!) filenames which satisfy the naming conventions of
the local operating system (this is not checked by gnu80, of course).
Note that, as usual in gnu80, case is not significant and both files
will be generated with {\em upper-case} names. This may well cause
some surprises if a lower-case name was intended (in Unix, for example).

Both records have all blanks ignored (even {\em within} filenames, 
and examples are: \\
\ \\
{\bf \%int = repulsion.int} \\
\ \\
{\bf \%RWF=dump.rwf}
\ \\
\ \\
If you need blanks in your filenames, you must change the code
in {\tt SUBROUTINE L0CMND} to omit blank stripping.

If a job is to be restarted
using these saved files, the restart job must
have the same file control records
before the Job Command Record.
Sometimes, if files are to be updated during a sequence
of restarted runs, it is advisable to
copy the saved files and restart with the copies.

The ROUTE is stored on the read-write file, so a job can only be 
{\em re}-started, not submitted as a continuation. 
It may be restarted at any point up to and including
the point at which it finished a complete Link. For example,
if a job has actually run to completion it may be restarted anywhere 
in the original ROUTE.

Thus must be indicated {\em where} the job  is to be restarted.  
This is done
with a record of the form:  \\
\ \\
{\bf \#RESTART Lxxx(N)}  \\
\ \\
The ` \#' must be in column 1, and ` Lxxx(N)' 
indicates that the job is to
restart at the N'th occurrence of link Lxxx. N is optional, and defaults
to 1. In fact, ` Lxxx' is optional also, and if not given, the job
restarts at the link in which the previous job terminated. Since the
original input  file is lost, you must supply any input to the job that
the remaining links will need. 

{\bf Note that the RESTART Command starts immediately after the \#,
there is no blank as in the Job Command Record}

It is also possible to supply  a new
route while  restarting.  In this case, the route information in the
read-write file is updated while the  rest of the data remains intact.
this is accomplished with the {\bf  RESTART} command:  \\
\ \\
\ \\
{\bf \#RESTART L1  \\
\#P new route goes here....}  \\
\ \\
\ \\
This causes execution to resume with Link1, which will read a new
route. One must, of course, supply any data input required by the new
route. This is a method of avoiding the re-computation of the
repulsion integrals in a job related to an earlier one. Of course,
the new ROUTE must explicitly omit the calculation which is being
avoided (this usually means a {\bf NONSTD} command).
\newpage
Thus a job originally submitted as: \\
\ \\
{\bf \%int = repulsion.int} \\
{\bf \%RWF=dump.rwf} \\
{\bf \# HF/6-31g} \\
(blank record) \\
Data Records \\
\ \\
can be restarted at the SCF stage by: \\
\ \\
{\bf \%int = repulsion.int} \\
{\bf \%RWF=dump.rwf} \\
{\bf \#RESTART L501} \\
(blank record) \\
\ \\
\ \\
always provided that the original job got as far as generating the
data for Link 501.
\subsection{Restarting; An Example}
It often happens that it is difficult to get a Geometry Optimisation
to {\em start} in the sense that it is difficult to get a converged set
of MO coefficients for the initial geometry. Usually, since the 
changes in geometry are small during the geometry optimisation, once a set of
converged MO coefficients has been obtained for one point
these coefficients are good enough to ensure that, 
when used as a starting point for other geometries close by, things will
go smoothly. 

This is particularly galling if one has done a huge calculation of 
molecular integrals and the SCF does not converge. It is useful to be able
to try the SCF again with some different convergence aids. Here is a pair
of jobs which illustrate this practice.
\newpage
The first job is an SCF calculation on the water molecule 
with a standard basis:
\begin{verbatim}
     %INT=example.int
     %RWF=example.rwf
     # HF/3-21G,VSHIFT=500

       Test of the use of RESTART

     0 1
     O
     H 1 0.956
     H 1 0.956 2 104.5

\end{verbatim}
In point of fact this run does go to completion with the convergence
aid of ``Level Shifter" set to $0.5$ ({\tt VSHIFT = 500};
actual value $500/1000$) but such a job may have failed to converge.

Notice that the two files {\bf must} be saved to use for a {\tt RESTART}.
\newpage
Now we may restart the job from where it finished with a different
Level Shifter:

\begin{verbatim}
      %INT=example.int
      %RWF=example.rwf
      %SAVE=H2O.SAVE
      #RESTART L1
      # NONSTD
      5/9=3500,32=1/1,2;
      99//99;

\end{verbatim}

This time the Route has to be given explicitly by use of the {\tt NONSTD}
command; it was constructed by setting up a test Command Record anticipating
success and therefore saving the converged MO coefficients:

\begin{verbatim}
      # HF/3-21G,VSHIFT=3500,SAVE=MO
\end{verbatim}

and editing the output Route from gnu80.

The meaning of the items are:
\begin{itemize}
\item
The Leading {\tt 5} is the Overlay to be run ( Overlay 5 is the
SCF overlay)
\item
{\tt 9=3500} is just {\tt VSHIFT=3500} in Overlay 5
\item
{\tt 32=1} is just {\tt SAVE=MO} in Overlay 5
\item
{\tt 1,2} are the Links to be run in Overlay 5 ({\tt L501, L502}
\end{itemize}
This job, when successfully run will generate a converged set 
of MO coefficients and save them for future use in an optimisation, say,
on file {\tt H2O.SAVE} suitable for use in a job like:
\newpage
\begin{verbatim}
     %GUESS=H2O.SAVE
     # HF/3-21G,VSHIFT=500,GUESS=READ,OPT

         Re-using the Converged MO coefficients

     0 1
     O
     H1 O ROH
     H2 O ROH H1 104.5

     ROH=0.956

\end{verbatim}
      
None of the working files are saved, again anticipating success.
\subsubsection{Error Messages}
If a Link is specified on the {\bf RESTART} record which is
not in the original ROUTE, an error message is given and the job terminates.
\end{document}
