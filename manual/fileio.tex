\documentstyle{book}
\begin{document}
\newpage
\setcounter{page}{1}
\markboth{\bf FILEIO}{\bf FILEIO}
\begin{description}
\item[NAME] FILEIO \\
Random access Input/Output
\item[SYNOPSIS] \ \\
{\tt
   subroutine FILEIO(IOPER, IFILNO, LEN, Q, IPOS) \\
   double precision Q(1) \\
   integer IOPER, IFILNO, LEN, IPOS \\
}
\item[DESCRIPTION] \ \\
{\bf FILEIO} performs all the  functions  dealing  with the logically
{\em random-access} I/O of gnu80.
 
Although {\bf FILEIO} may appear to maintain a large  number  of
files  (up  to 200), all of this data is actually stored into a
{\em single}  disk file in the usual FORTRAN sense.  
This (FORTRAN random access) file is divided  up by {\bf FILEIO}
and  portions of it are allocated to each of the logical files
(buckets and read-write files). Thus ``files '' in
{\bf FILEIO} terminology are not the same as ``files'' 
in standard FORTRAN terminology. It is therefore necessary to
use one or the other consistently in a program. gnu80 uses
``files'' always to mean {\bf FILEIO}-compatible
files. Thus data {\em must} always be read or written by {\bf FILEIO}
and {\em not} by standard FORTRAN READ or WRITE statements.
Of course, {\bf FILEIO} itself uses standard FORTRAN statements
to do the actual I/O!

The two types of file ``buckets'' and ``read-write
files'' only differ conceptually not essentially; roughly
speaking, read-write files are inter-link communication
and buckets are scratch space (particularly during integral
transformations).

For each file (bucket or read-write  file),  {\bf FILEIO}  
maintains  both  read  and  write pointers.  A write operation on a
file, for instance, starts at the position of the write pointer
for  that  file.   For each value written, the write pointer is
incremented.  The read pointer behaves the same way on read operations.
\item[ARGUMENTS:] \ \\
\begin{description}
\item[IOPER]   type of I/O operation to perform:
\begin{description} 
\item[0] define file   {\tt IFILNO}  to have length  {\tt LEN}   
\item[+1]  synchronous write operation. Control   returns  to  the
calling routine when the write is completed.
(this operation defines a file with length {\tt LEN} if the file
was not previously defined). Note that in gnu80 {\em portable}
FORTRAN I/O is used so that {\em all } I/O is synchronous notwithstanding
information to the contrary.
\item[+2]  synchronous read operation.
\item[4]  define subfile.
\item[5]  delete the file indicated by {\tt IFILNO}.
\item[6]  delete all routine-volatile files (those  with  numbers
larger that 2999).
\item[7]  delete all the link-volatile files.  This should  not
be done by a user directly, but rather let {\tt NEXTOV} do this
for you when overlaying.
\item[8]  delete all overlay-volatile files.  Once again, don't do
this unless you're sure you know what you're doing.
\item[9]  {\bf FILEIO} open.  This is done in chain at the beginning of
the program, and should not be used elsewhere.
\item[10]  {\bf FILEIO} close.  This is done automatically by {\tt NEXTOV} on
its way out, and should not be used elsewhere.
\item[11]  this returns the length of the specified file (in {\tt LEN}).
if the file is undefined, then 0 is returned.  Thus,
you can check for the existence of a file before trying
to read from it.
\item[12]  {\bf FILEIO} initialization. This is done by {\tt MAIN} at the
beginning of the run, and should not be done elsewhere.
\end{description}
\item[IFILNO] The absolute value of {\tt IFILNO} is the number  of  the  file
which is to be read, written, defined, or deleted.  Any
non-zero value in permitted, but the following  
conventions are to be observed in gnu80:
\begin{description} 
\item[1-499]  permanent buckets.
\item[500-999] permanent read-write files.
\item[1000-1499]  overlay volatile buckets.  These are  
deleted automatically before each new overlay.
\item[1500-1999]  overlay volatile read-write files.
\item[2000-2499]  link volatile buckets.  These  are  
automatically deleted before each new link.
\item[2500-2999]  link volatile read-write files.
\item[3000-3499]  routine volatile buckets.  These are  
deleted when {\bf FILEIO} is called with {\tt IOPER=6},
and also before each new link.
\item[3500-3999]  routine volatile read-write files.
\end{description} 
For read or write operations, supplying the {\em negative}
of  the  file number causes the read or write 
pointer to be rewound (reset to the base of the file) 
before  the operation. Note that this is done before the
argument {\tt IPOS} is processed.  For file deletion or file
definition operations, the sign of the file number is
ignored.
\item[LEN]   This quantity is the number of double precision values
(quadruples of bytes, usually)  to be transferred in a read or a write 
operation, or the number of these values to be allocated
in a define file operation.
\item[Q] a double precision array for the data read or written.
\item[IPOS] Used to help specify the position in the   file
at  which  an I/O operation will commence. The value of
the read or write pointer is incremented by {\tt IPOS} before
the read or write operation is performed. The possible
rewind of the pointer (as specified by  the  {\em sign}  of
{\tt IFILNO}) is done before {\tt IPOS} is considered.  Thus, a 
{\em positive} file number means that {\tt IPOS} specifies  the  new
position relative to the current position; a {\em negative}
{\tt IFILNO} means {\tt IPOS} specifies the new  position relative
to the begining of the file.
\end{description}
\item[SEE ALSO] \ \\
{\bf NTRAN, TREAD, TWRITE}
\item[DIAGNOSTICS] \ \\
None directly, but {\bf NTRAN} provides some.
\end{description}
\end{document}
