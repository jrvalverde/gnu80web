\chapter{\sf General Use of {\tt gnu80}}
\label{chap3}
\section{\sf Beyond ` Standard Methods'}
The reasons why one might want to use the more fundamental
structures of {\tt gnu80} are, in general, one or more of the following:
\begin{enumerate}
\item The basis sets provided internally by {\tt gnu80} may not
be adequate or appropriate for a particular problem. In section
\ref{basis_lims}, the limitations of the basis sets are given; acceptable
quality internal basis sets are only available for atoms
up to Argon. Although {\tt gnu80} recognises the chemical symbols
of the atoms of the third row of the periodic table (first transition
series), there are {\em no} basis sets provided for these atoms.
\item The calculations may be required in a sequence
which is not generated by the command record parser.
\item
Much the most common requirement is to recover from some
kind of disaster in a Standard {\tt gnu80} run; a convergence
failure or a machine failure during a long run.
\end{enumerate}
In point of fact, many of these requirement can be met by using
a ``standard method'' command record including
the {\bf NONDEF} command and a specific set of non-default options.
\section{\sf Program Structure}
\subsection{\sf Introduction}
In order to be able to use the full flexibility of {\tt gnu80}
it is necessary to have some knowledge of the way the system
is structured and the way in which the Commands and Options
of the last Chapter relate to the more basic Overlays, Links
and Options. Basically, General Use consists of replacing the
work of the module  which parses the command record and generates
a ROUTE through {\tt gnu80}. If the output of one of the
sample data sets is examined, the ROUTE is the set of records
which consists of (small) integers separated by slashes, commas
and equals signs at the start of the job. It is this sequence
or some replacement for it which must be generated ``by hand''
if {\tt gnu80} is to be of General Use.

However, except for the simplest ROUTEs with the most common options
(which could, in any case, be more easily generated by Standard Methods!)
it is almost impossible to generate a ROUTE from first principles.
What is done in practice, if the Standard Methods command record will
not generate the ROUTE required, is to use the Standard Methods
parser to generate the nearest ROUTE  and then {\em modify} this ROUTE
manually by adding links and options.

In fact, this can be done in a two-stage process by starting
with a Standard Method command record, adding the {\bf NONDEF}
command and suppliying some non-default options then, finally,
changing the ROUTE and any remaining options. The beginner
is strongly advised to get some familiarity with the {\bf NONDEF}
command and its consequences before experimenting with General Use.

In fact, it is just about as feasible
to write a useful, non-standard, ROUTE from first principles 
as it is to learn to ride a bicycle or
to swim by  reading the manual.
\subsection{\sf Overview of {\tt gnu80}}
{\tt gnu80} consists of a series of programs which may be called in
arbitrary sequence and which communicate with each other through disc
files. Each such program is called a {\bf LINK}; the links are grouped into
{\bf OVERLAYS}. The numbering of the links is such that each overlay may
contain up to 99 links. Thus overlay 1 may have links 101, 102, ...
199, overlay 2 may have links 201,202, ... 299 and so forth. Links with
closely related functions 
(from the quantum chemistry point of view)
are grouped together in the same overlay.
Note that this terminology is essentially historical and
is only used to designate the different
{\em parts} of the program, and has little coincidence with the physical
overlay structure of the load module if, indeed, the load module
is explicitly overlaid; it may be all loaded or the ``overlays''
managed by the virtual memory manager.
\begin{description}
\item[Overlay 0] reads the command record, forms the detailed
route, and sets up files and some general tables.
\item[Overlay 1] reads Input Sections 2---5 and control optimization
procedures.
\item[Overlay 2] computes Cartesian coordinates of the nuclei and derives
symmetry related information.
\item[Overlay 3] sets up the basis set and evaluates one- and two-electron
integrals.
\item[Overlay 4] forms the initial guess needed for Hartree-Fock
wavefunctions.
\item[Overlay 5] determines the Hartree-Fock wavefunction (SCF procedures).
\item[Overlay 6] performs analysis and output of Hartree-Fock wavefunctions
and energies.
\item[Overlay 7] evaluates integral derivatives (with respect to nuclear
coordinates) and uses them to evaluate Hartree-Fock forces on the
nuclei.
\item[Overlay 8] transforms the two-electron integrals to the basis of
molecular orbitals.
\item[Overlay 9] calculates the correlation energy either by Pertubation
Theory or by Configuration Interaction.
\item[Overlay 99] is the termination Overlay.
\end{description}

{\bf The complete set of links, together with a short functional
description, is listed in Appendix \ref{app3a} at the end of this chapter.}
\section{\sf Program Options}
The execution of each Overlay is controlled by a number of OPTIONS
(numbered from 1 to 50). Each OPTION may be assigned positive integer
values, 0 being the default. 

Note that the value of an OPTION is held
unchanged throughout execution of {\em all the links in one Overlay.} 
Thus
the significance of a particular OPTION applies to all the component
links in one pass through the Overlay. An exception is Overlay 5, where
the OPTIONs in Link 503 differ from those appropriate to 501 or 502.
However, these are independent SCF programs and execution of 503
immediately after 501 or 502 is never required. 

{\bf The full list of
OPTIONs in {\tt gnu80} is given in Appendix \ref{app3b} which follows
Appendix \ref{app3a} at the end of the chapter.}
\section{\sf Non-Default OPTIONs}
\label{nondef}
OPTIONs set by the route generator can be reset to any particular
value without having to specify a complete non-standard route (see 
\ref{nonstd} for non-standard routes).
This is accomplished by using the {\bf NONDEF} command 
in the command record.
A separate input section (1b) should be inserted {\em after the
blank record following the command record} which specifies 
the non-default OPTIONs.
For most jobs, all of the non-default OPTIONs can be placed on one
record, but overflow to other records is permissable. The section must be
terminated by a blank record.

Non-default OPTIONs can be set for any record generated by the route
generator. They are set by specifying the Overlay for which the OPTIONs
are to be changed, followed by a separator ( space , : / ) followed
by a list of the OPTIONs and their new values. 
Thus: \\
3/34=1,35=4 \\
will set OPTIONs 34 and 35 in Overlay 3 to 1 and 4 respectively.
If OPTIONs have to be reset in different Overlays the
specifications must be separated by a semi-colon (;). 
Thus:  \\
2/34=1;3/34=5  \\
will set OPTION 34 to 1 in Overlay 2 and to 5 in Overlay 3.
It is sometimes necessary to set OPTIONs in one or more occurences
of an Overlay. This is done by specifying the occurence number of the
link:  \\
3(2)/34=5;3(3)/34=5   \\
will set OPTION 34 to 5 in the second and the third
occurence of link 3  and  \\
3(*)/34=1  \\
will set OPTION 34 to 1 in all occurences of Overlay 3.
\section{\sf Non-Standard Routes}
\label{nonstd}
Non-Standard routes may be specified by requesting a particular
sequence of Overlays and links together with associated OPTIONs. 

The
command record is now replaced by a series of
job-descriptions records beginning with the record:  \\ 
\vspace{0.25cm}
{\bf \# NONSTD}  \\
\vspace{0.25cm}
This is
followed by one record for each Overlay in the sequence. This record gives
the Overlay number, a slash (/) symbol, the OPTIONs for that Overlay, 
another slash (/) symbol, the
list of links within that Overlay to be executed 
and finally a semi-colon (;) symbol.  \\
Thus:  \\
\vspace{0.25cm}
{\bf 7/5=3,7=4/2,3,16;}  \\
\vspace{0.25cm}
commands a pass through links 702, 703, 716 (in this order) with OPTION 5
set equal to 3 and OPTION 7 equal to 4. If all OPTIONs have their
default (0) value, the record would be:  \\
\vspace{0.25cm}
{\bf 7//2,3,16;}  \\
\vspace{0.25cm}
One further feature in the route specification is the {\bf JUMP} number.
This is given in parentheses at the end of the link list, before the
semicolon. It indicates where the execution jumps to after completion
of the current Overlay. If jump is omitted, the default value is +1,
indicating that the program will proceed to the next record in the list.
if JUMP = -4, on the other hand, as in:  \\
\vspace{0.25cm}
{\bf 7//2,3,16(-4);}  \\
\vspace{0.25cm}
then execution will next take place four route records back. The {\bf JUMP}
feature permits loops to be built into the route and is useful for
optimization runs. JUMP is altered to +1 in Overlay 1 when optimization
are complete by an appropriate argument in the call to the chaining
routine.
\section{\sf General Basis Sets: non-STO}
\label{genbasis}
A set of ``standard'' Basis sets (STO-3G, 3-21G, etc.) 
is stored internally in
{\tt gnu80}.  When one of these bases is selected (either with a basis
command in the command record or by setting the appropriate Overlay 3
OPTIONs), no basis set input is required.  

A ``non-standard''
basis can be used in a {\tt gnu80} calculation, either by specifying
{\bf GEN} (for general basis) in the command record, or by setting the
appropriate overlay 3 OPTIONs.  In this case, the basis set description
must be provided as {\em input} in a separate basis input section.  

In this context, any basis which is not {\em internal} to {\tt gnu80}
(see Section \ref{basis}) is called a general basis. This includes
both what might be called ``genuine'' general bases {\em and}
``standard'' bases {\em which are not provided by } {\tt gnu80}.
For example, the STO-1G basisis a general basis,
not because it is in any way unusual but simply because STO-1G bases 
provided internally
by {\tt gnu80}. In practice, this is the most common reason for the
use of the {\bf GEN} command; to enable a standard 
(i.e. non-Pseudopotential) calculation to be performed
on a molecule containing an atom of atomic number greater than Argon.

Note that the internally-stored ``minimal'' bases
(STO-3G) are used in the calculation of the default GUESS of SCF
starting point; the Huckel or Projected Huckel GUESS. Thus, if
an atom of atomic number greater than that that of the heaviest atom for
which this basis is available is to be included in a molecular
calculation, the default GUESS cannot be provided. The only sensible
option in this case is to use the eigenvectors of the one-electron
Hamiltonian i.e.
GUESS=CORE. This is done and a message printed.
\subsection{\sf Shells}
Before describing
the format of the basis section, a discussion of {\em shells} is useful.

A {\em shell} is a set of basis functions 
$\{\phi_\mu\}$ with common shared exponents. {\tt gnu80}
supports {\em four} kinds of shells: 
\begin{itemize}
\item s shells; An s-shell contains a single s-type basis function. 
\item p shells; A p-shell contains
the three basis functions $p_x ,\; p_y , \;$ and $p_z$. 
\item d shells; A d-shell may be defined to
contain either the six ``second-order'' basis functions 
$d_{x^2}, \; d_{y^2}, \; d_{z^2}, \; d_{xy}, \; d_{yz}, \; d_{zx} $
or the five ``pure d'' (Cubic Harmonic) basis functions 
$d_{z^2-r^2}, \; d_{x^2-y^2}, \; d_{xy}, \; d_{yz}, \; d_{zx}$
\item sp shells; An sp-shell contains four basis functions with common
gaussian exponents: one s-type function and the three p-functions 
comprising a p-shell.
\end{itemize}
Usually, a single basis function is a linear combination of more than one
primitive gaussian function.  Thus, for an s-type function, one may
have for the basis function $\phi_\mu (r)$:
\[
\phi_\mu (\vec{r}) = \sum_{i=1}^{N} { d_{i\mu} exp(- \alpha_i
f^2 r^2 ) }
\]
Here N is the number of primitive functions composing the basis
function and is called the {\em degree of contraction} of the basis function.
The coefficients $d_{i\mu}$ are called contraction coefficients.  
The quantities
$\alpha_i$ are the {\em exponents} and $f$ is the {\em scale factor} 
for the basis function.
\subsection{\sf Exponents and Scale Factors}
The STO-NG Gaussian expansions of Slater Type Orbitals are provided
for the first 36 atoms of the periodic table (H---Ar) and 
assume the following STO orbital exponents which are either
the optimum isolated-atom values for inner STOs or
molecule-optimised values for the valence shells. The exponents
are constrained by the condition that they are for {\em shells}
rather than for individual STOs i.e. 
\[
\zeta_{2s} = \zeta_{2p} \\
\zeta_{3s} = \zeta_{3p}
\]
There is some confusing nomenclature in the literature arising from the
way in which the Gaussian expansion of STOs is carried out.
A given expansion of a STO of a given type (e.g. 2p) is a set of
linear expansion coefficients and Gaussian exponents as above.

The expansion coefficients ($d_{i\mu}$) are {\em independent} of the STO
exponent ($\zeta , $ say) and the $\alpha_i$ depend on the {\em square}
of $\zeta$. Thus the expansion coefficients and exponents ($\alpha_i$
are usually quoted for an STO exponent $\zeta = 1$ and the corresponding
scaled vales for all other STOs
can be easily obtained.

This ``scaling'' of the Gaussian exponents has led to
the use of the term ``scale factors'' being used for the
STO {\em exponents}. However, it is an empirical fact that, in
a molecular environment, the STO optimum STO exponents
are not quite the same as the ones for the separate atoms and so
they are often ``scaled'' to allow for this molecular
electron re-arrangment. The numbers which do this scaling
(usually numbers around unity, but typically greater than 1)
are also call ``scale factors'' ! Thses are the $f_i$ in
the above expansion.
In this manual the STO orbital exponents will be called 
exponents $(\zeta 's)$ and any factors which multiply
the orbital exponents to accommodate a molecular environment
will be called scale factors.

With this in mind the standard STO {\em exponents} for the STO-NG
expansions are: \\
\begin{center}
\begin{tabular}{|r|l|l|l|}
\hline
Atom &  1s &  2sp & 3sp \\
\hline
H   &  1.24 & - & - \\
He  &  1.69 & - & - \\
Li  &  2.69 & 0.80 & - \\
Be  &  3.68 & 1.15 & - \\
B   &  4.68 & 1.50 & - \\
C   &  5.67 & 1.72 & - \\
N   &  6.67 & 1.95 & - \\
O    & 7.66 & 2.25 & - \\
F & 8.65 & 2.55 & - \\
Ne & 9.64 & 2.88 & - \\
Na & 10.61 & 3.48 & 1.75  \\
Mg & 11.59 & 3.90 & 1.70  \\
Al & 12.56 & 4.36 & 1.70  \\
Si & 13.53 & 4.83 & 1.75  \\
P & 14.50 & 5.31 & 1.90 \\
S & 15.47 & 5.79 & 2.05  \\
Cl & 16.43 & 6.26 & 2.10  \\
Ar & 17.40 & 6.74 & 2.33  \\
\hline
\end{tabular}
\end{center}
\ \\
The N-31G, N-31G* and N-31G** have the same inner-shell
STO orbital exponent (the one expanded in terms of N Gaussians)
and a ``split-valence'' set of two STOs in the valence shell
(one expanded in terms of 3 Gaussians and one expanded as a single term).
The relevant valence STO exponents are obtained from the above
{\em single } exponents by multiplying each single valence-shell
STO exponent by each of the two {\em scale factors} given below in turn.
\ \\
\begin{center}
\begin{tabular}{|l|r|r|}
\hline
Hydrogen & 1s & 1s'  \\
\hline
H & 1.20 & 1.15  \\
\hline
\end{tabular}
\end{center}
\ \\
\begin{center}
\begin{tabular}{|l|r|r|r|r|r|}
\hline
Atom & 1s & 2sp & 2sp' & 3sp & 3sp'  \\
\hline
B & 1.00 & 1.03 & 1.12 & - & - \\
C & 1.00 & 1.00 & 1.04 & - & -\\
N & 1.00 & 0.99 & 0.98 & - & -\\
O & 1.00 & 0.99 & 0.98 & - & -\\
F & 1.00 & 1.00 & 1.00 & - & -\\
P & 1.00 & 1.00 & - & 0.98 & 1.02 \\
S & 1.00 & 1.00 & - & 0.98 & 1.01 \\
Cl & 1.00 & 1.00 & - & 1.00 & 1.01 \\
\hline
\end{tabular}
\end{center}
\ \\
In the case of the LP-N1G (Local potential optimised) bases
no scale factors are used i.e. the scale factor is 1 for both
inner and outer orbitals.

The N-31G* and N-31G** basis sets
have polarisation functions added with the following
standard polarization {\em exponents}. \\
\begin{center}
\begin{tabular}{|l|r|}
\hline
Atom &   Value \\
\hline
H    &  1.1  \\
Li   &  0.2 \\
Be   &  0.4 \\
B    &  0.6 \\
C-Ne &  0.8 \\
\hline
\end{tabular}
\end{center} 
\ \\  
The standard polarization exponents for STO-NG* basis are: 
\ \\
\begin{center}
\begin{tabular}{|r|l|}
\hline
Atom  &  Value \\
\hline
Na, Mg &  0.09 \\
Al-Cl &  0.39 \\
\hline
\end{tabular}
\end{center}
\ \\
The maximum degree-of-contraction ($N$) permitted in {\tt gnu80} is eight.

Just as the basis function here is formed from a linear combination
of primitive functions; a shell, in general, is composed of primitive
shells.  A p-shell, for instance, is a set of three p-type basis
functions with common exponents, contraction coefficients
and scale factors.  In a sp-shell there are four basis functions (s,
and a p-shell) and four primitive functions in each primitive shell.
In each primitive shell, the exponents and scale factors are the same
in each of the four functions.  However, the Contraction coefficients
for the s-type function are usually be different than for the three p-type
functions.

To illustrate further the features described above, consider the series
of basis sets STO-3G, 4-31G and 6-31G* for the carbon atom.  The
STO-3G basis on any first row atom consists of only 2 shells.  One
shell is an s-shell consisting of a set of 3 primitive gaussian
functions least-squares fitted to a Slater Type Orbital (STO)
1s orbital with an
appropriate scale-factor; for the carbon atom, the ls orbital scale-factor
is 5.67.  The other sp-shell is a least-squares fit of 3 gaussians to
Slater 2s and 2p orbitals with the constraint that the s and p functions
have equal exponents.

For carbon atom the 2sp-shell has a scale factor of 1.72.  The 4-31G
basis on a first row atom has three shells.  One shell is a contraction of
four primitive s-type gaussians with a scale factor of unity.  The second
shell is a combination of three primitive sp-shells, again with a scale
factor of unity.  The third shell consists of a single sp-function with a
scale factor of 1.04.  The 6-31G* basis has four shells and differs
qualitatively from the 4-31G basis in only two respects.  First, the
innermost shell is a contraction of six gaussian instead of four; and
secondly the last shell is a d-shell.
 
External basis sets may be read in to {\tt gnu80} by setting IOP(5)=7
and ICP(6)=0 or by specifying {\bf GEN} (for general basis) in the 
command record.
Further specification of the desired external basis
can be achieved by using IOP(8) (refer to Appendix \ref{app3b} 
for a listing of
the overlay 3 OPTIONs).  All external basis input is handled by routine
GBASIS.  A schematic illustration of this input is given below.

\begin{itemize}
\item Number of Gaussian functions  (degree of contraction) in each shell
(80I1).
\item Center assignments (zero-center terminates reading of input).  35I2
format.
\item Shell descriptor record (A4,A6,A4,I2,F12.6)  \\
Field 1 (A4)  ' STO' use STO routines.  \\
'    ' read in records defining functions.  \\
'****' step to next set of centers.  \\
Field 2 (A6)  name used in printing and if field 1 = ' STO', this
field defines the routine from which the STO-NG
functions are taken.  \\
Field 3 (A4)  type of shell.  \\
{\tt '   S}'  s-shell.  \\
{\tt '   P}'  p-shell.  \\
{\tt '   D}'  d-shell.  \\
{\tt '   F}'  f-shell.  \\
{\tt '  SP}'  sp-shell.  \\
{\tt ' SPD}'  spd-shell.  \\
Field 4 (I2)  number of Gaussians (degree of contraction) for the
current shell.  \\
Field 5 (F12.6)  scale-factor for current shell. \\
Primitive Gaussian record (4E20.10)  \\
Field 1  exponent.  \\
Field 2  s-coefficient.  \\
Field 3  p-coefficient.  \\
Field 4  d-coefficient.  \\
In the case of an f-shell, the f-coefficient is taken from field one.
the number of primitive function records read in is determined by the
degree of contraction specified on the preceeding shell descriptor
record.
\end{itemize}


The basic unit of information that routine GBASIS deals with is the
{\em shell definition block}.  A shell definition block, together with IOP(8),
contains all necessary information to define a shell of functions.
It consists of a {\em shell descriptor record} and from one to eight 
{\em primitive
Gaussian records}.  The shell descriptor record has 5 fields with 
\begin{center}
{\tt FORMAT(4X, A6, A4, I2, F10.4) } \\
or \\
{\tt FORMAT(A4, A6, A4, I2, F10.4) } 
\end{center}
If the first four characters of the record are ` \ STO' ,
then the data may be abbreviated and the routines provided will
generate the exponents and coefficients for 1s 2sp 3sp 3d and 4sp
shells, this option will be described later.

If the first four characters in the record are blank, the rest of
the record 
contains IORB, ITYPE, NGAUSS and SC. 
\begin{description}
\item[IORB] is simply a four-character label used to identify this
shell, e.g. ` 2SPI' for the larger-exponent (``inner'' )
shell of a split-valence pair of shells.
\item[ITYPE]
defines the shell type and shell constraint and may take on the values
` bbbS' ` bbbP'  and ` bbSP' ( b denotes a blank space) 
denoting, respectively,
an s-shell, p-shell, or an sp-shell. 
\item[NGAUSS] specifies the number of primitive
Gaussian shells in the contraction for the shell being defined and must
be between 1 and 8.  
\item[SC] is the shell scale-factor  
\end{description}
Subsequent records define the exponents $\alpha_k$ and contraction
coefficients, $d_{k\mu}$ for the NGAUSS primitives composing the shell.

The FORMAT of these records is (4E20,10). 
\begin{description}
\item[field 1] contains the exponent $\alpha_k$ 
\item[fields 2] contains the coefficient of the s-type function
in this shell,
\item[field 3] contains the coefficient of the p-type function
if this shell requires one (i.e. is an sp shell)
\item[field 4] contains the coefficient of the d-type function
in this shell if one is required.
\end{description}
NGAUSS such records are required.

One customarily places at least one, and quite often several shells on any
given nuclear centre.  
A centre definition block consists of a centre identifier
record, and one shell definition block for each shell desired on the centre(s)
specified, and is terminated by a record with  **** in columns 1-4. The centre
identifier record has 35I2 FORMAT and simply gives the number of the centres
on which the basis functions are to be placed.

The decription of input to GBASIS is now complete except for the first record
and the last record.  The first record (80I1) contains the degree of
contraction for each shell in the calculation.  Centres are delimited
by a contraction of 0 and the list is terminated by a 9.  Overall input
to GBASIS is terminated by a blank record.
\newpage
\subsubsection{\sf Example}
In this example, 
which is Example 16 of the input examples,
a 6-31G** SCF calculation is performed on the water
molecule.  However, instead of using the internally stored basis, the
basis set specification is provided as input (in the Basis Section).
The Basis Section is called for by the presence of the command 
{\bf GEN} in the
command record.  The command {\bf 6D}
indicates that if d-type functions are included in the basis, then the six
Cartesian Gaussian functions should be used.  The Title and Geometry
Sections are as described earlier, and the remaining records  constitute the
Basis Section.
%
{\small
{\tt
\begin{enumerate}
\item \#N RHF/GEN, 6D
\item (blank record)
\item gnu80 EXAMPLE JOB  16
\item WATER MOLECULE: RHF/6-31G** READ IN, EXPERIMENTAL GEOMETRY
\item (blank record) 
\item 0 1
\item O
\item H 1 0.957
\item H 1 0.957 2 104.5
\item  (blank record)
\item 6311031103119
\item 01
\item         1S   S 6 1.00
\item 5484.67166           0.00183107443
\item  825.234946           0.0139501722
\item  188.046958           0.0684450781
\item 52.9645              0.232714336
\item  16.8975704           0.4701928980
\item 5.79963534           0.358520853
\item       2SPI  SP 3 0.99
\item  15.8551334          -0.110777549         0.0708742682
\item 3.67302682          -0.148026262         0.339752839
\item 1.03434522           1.13076701          0.727158577
\item       2SPO  SP 1 0.98
\item 0.281138924          1.0                 1.0
\item          D   D 1 1.
\item 0.8                                                         1.
\item ****
\item 0203
\item         1S   S 3 1.20
\item 13.007734            0.03349460434
\item  1.96207942           0.2347269535
\item 0.444528953          0.813757326
\item         1S   S 1 1.15
\item 0.121949156          1.0
\item          P   P 1 1.0
\item 1.1                  0.0                 1.0
\item ****
\item (blank record)
\end{enumerate}
}
}
\newpage
Record 11 specifies the number of shells on each of the three centres, and
the number of primitives in each shell.  Column one contains a 6, indicating
that the first shell on centre 1 (Oxygen) is a contraction of six primitive
Gaussians.  The 3 in column two indicates that the second shell on centre 1
is a contraction of three primitives.  The next two columns contain 1's; the
next two shells on this centre each contain only one primitive shell.  
The 0
in column five terminates the list of shells for this centre.  Thus, the
Oxygen atom (centre 1) has four shells centred on it.  These shells are
contracted shells with degrees of contraction of 6, 3, 1 and 1, respectively.
The specification for centre two is provided in column 5-7.  This centre has
three shells, with degrees of contraction of 3, 1 and 1, respectively.
Centre three has the same specification as centre two. Finally, the
9 in column 13 terminates the list.

Records 12---28 and 29---39 constitute two centre definition blocks.
Consider the second of these two blocks, records 29---39.  
The first line in the
block provides a list of the centres to which the shells are to be
attached.  In this case centres two and three are specified.  Thus, the two
hydrogen atoms will each have the same types of basis functions. records
30---33 provide the first of three shell definition blocks for the
hydrogens.  The shell descriptor record, record 30, contains four fields.
The first field is just a string used for output.  The second field is
` bbbS', indicating that the shell being defined is an s-shell.  The
next field specifies the degree of contraction for the shell, in the
case three.  The final field contains the scale factor for the shell,
1.20.  Since the degree of contraction here is three, the next three
lines must specify the exponents and contraction coefficients for
the three primitives.  In each of records 31---33, the first field
contains the gaussian exponent, and the second field provides the
contraction coefficient.

The next shell definition block, records 34 and 35, illustrates nothing
new.  It merely specifies an s-shell with scale factor 1.15 and degree
of concentration 1.
Records 26-37 specify a p-shell with degree of contraction 1 and a scale
factor of 1.0.  The gaussian exponent is 1.1, the s-coefficient field
is zero (for a p-shell) and the p-coefficient is 1.0.  record 38
terminates the centre definition block, and record 39 (a blank line)
terminates the Basis Input Section.

The specification of an sp-shell is illustrated by records 24 and 25.  The
second of the shell descriptor record (record 24) is ` bbSP' , 
specifying an sp-shell.
The degree of contraction here is one, and record 25 provides the exponent,
s-coefficient, and p-coefficient for the single primitive shell.  Records
20---23 describe an sp-shell with degree of contraction three.
Finally, the specification of a d-shell is shown in records 26 and 27.  
The
second field of record 26 is ` bbbD' , specifying a d-shell.  
On the following
line, the exponent is provided, the s- and p- coefficients are zero, and the
d-coefficient is 1.0
\newpage
\subsubsection{\sf Example 16, Output summary}
{\small
\begin{verbatim}
  25 BASIS FUNCTIONS       42 PRIMITIVE GAUSSIANS
   5 ALPHA ELECTRONS        5 BETA ELECTRONS
     NUCLEAR REPULSION ENERGY    9.1969310957 HARTREES
 RAFFENETTI 1 INTEGRAL FORMAT.
 TWO-ELECTRON INTEGRAL SYMMETRY IS TURNED ON.
     5235 INTEGRALS PRODUCED FOR A TOTAL OF      5235
     9148 INTEGRALS PRODUCED FOR A TOTAL OF     14383
 PROJECTED HUCKEL GUESS.
 INITIAL GUESS ORBITAL SYMMETRIES.
       OCCUPIED: (A1) (A1) (B2) (A1) (B1)
       VIRTUAL:  (B2) (A1)
 RHF CLOSED SHELL SCF.
 REQUESTED CONVERGENCE ON DENSITY MATRIX=  0.5000d-04 WITHIN  32 CYCLES.

SCF DONE:  E(RHF) =  -76.0231735680     A.U. AFTER    7 CYCLES
            CONVG  =    0.2557d-04             -V/T =   2.002801
 ORBITAL SYMMETRIES.
       OCCUPIED: (A1) (A1) (B2) (A1) (B1)
       VIRTUAL:  (A1) (B2) (B2) (A1) (A1) (B1) (B2) (A1) (A2) (A1)
                 (B1) (A1) (B2) (B2) (A2) (B1) (A1) (A1) (B2) (A1)
  THE ELECTRONIC STATE IS 1-A1.
                           1         2         3         4         5
                         (A1)      (A1)      (B2)      (A1)      (B1)
     EIGENVALUES --   -20.56041  -1.34040  -0.70352  -0.56872  -0.49704
          CONDENSED TO ATOMS (ALL ELECTRONS)
              1          2          3
  1   O   8.056815   0.308390   0.308390
  2   H   0.308390   0.379646  -0.024833
  3   H   0.308390  -0.024833   0.379646
          TOTAL ATOMIC CHARGES.
              1
  1   O   8.673594
  2   H   0.663203
  3   H   0.663203
 DIPOLE MOMENT (DEBYE): X= 0.0000   Y= 0.0000   Z= 2.1846   TOTAL= 2.1846
\end{verbatim}
}
\newpage
\section{\sf General Basis Sets: STO type}
The explicit input of an STO-NG basis is only necessary
in order to be able to carry out a calculation on a molecule
using STO-NG basis functions which are
not provided internally or non-minimal
STO-NG bases.

As outlined above the shell descriptor record has 5 fields with 
\begin{center}
{\tt FORMAT(A4, A6, A4, I2, F10.4) } 
\end{center}
If the first item supplied as data is ` bSTO' (b is a blank)
then (for shells up to 4sp) the input of non-standard basis data is
assumed to be that of the Gaussian expansion of an STO and does not
require the exponents and coefficients to be supplied explicitly,
although they may be, if required.

Thus in this case only the string ` bSTO' is supplied and the rest of this
record contains {\em for each shell}:
contains IORB, ITYPE, NGAUSS and SC. 
\begin{description}
\item[IORB] is now interpreted and must contain one or other of
` bbnS' , ` bbnP' , ` bnSP' , ` bbnD' where N is the
principle quantum number of the shell. 
\item[ITYPE]
defines the shell type and shell constraint and may take on the values
` bbbS' ` bbbP'  ` bbbD' and ` bbSP' ( b denotes a blank space) 
denoting, respectively,
an s-shell, p-shell, d-shell or an sp-shell. 
\item[NGAUSS] specifies the number of primitive
Gaussian shells in the contraction for the shell being defined and must
be between 1 and 8.  Here it is the N of STO-NG
\item[SC] is the shell scale-factor; in this case the STO shell exponent.  
\end{description}
\newpage
\subsubsection{\sf An Example: Zinc Dimethyl}
The rules in this section are best understood by an example, here is
the  STO data for the Zinc dimethyl molecule:
\begin{verbatim}
# HF/GEN 5D

Zinc dimethyl HF/STO-3G (input Basis)

0 1
ZN
C1 ZN RZNC
H1 C1 RCH ZN AN
H2 C1 RCH ZN AN H1 120.0
H3 C1 RCH ZN AN H1 -120.0
X  ZN 1.0 C1 90.0 H1 0.0
C2 ZN RZNC X 90.0 C1 180.0
H4 C2 RCH ZN AN H1 180.0
H5 C2 RCH ZN AN H2 180.0
H6 C2 RCH ZN AN H3 180.0

RZNC=1.682
RCH=1.09
AN=108.0

333330330303030330303039
 1
 STO    1S   S 3   29.43
 STO   2SP  SP 3   12.52
 STO   3SP  SP 3   5.19
 STO    3D   D 3   4.90
 STO   4SP  SP 3   1.90
****
 2 6
 STO    1S   S 3   5.67
 STO   2SP  SP 3   1.72
****
 3 4 5 7 8 9
 STO    1S   S 3   1.24
****
(blank record)
\end{verbatim}
\newpage
There are a few points to be made about this dataset:
\begin{enumerate}
\item Even though the STO bases are provided internally for the 
Zinc, Carbon
and Hydrogen atoms, the use of {\bf GEN} over-rides their use
and they must be provided explicitly for this example.
\item The data is slightly redundant in the sense that the contraction 
lengths are given {\em twice}; once in the initial record of
degrees of contraction (since the system cannot know that some STO
records are to follow) and once on each STO record.
\item The N of STO-NG is taken from the second input item (1S, 4SP etc.)
\item The assumed numbering of the atoms, used both in the degree-of-
contraction record and on the record preceeding each set of shells,
{\em omits the dummy atoms}, thus the atoms in Zinc dimethyl are
numbered 1 to 9, omitting the dummy X, used to avoid an 180 bond angle.
\end{enumerate}
\section{\sf Non-Standard Optimization Input}
In general, for a ``Berny''  optimization or a 
Murtaugh-Sargent
optimization, the more accurate the initial guess of the second
derivative matrix, the fewer steps that will be needed to reach the
stationary point. Because of this feature,  provision 
has been made for users
to {\em provide} diagonal second derivative or (for Berny optimization) to
request numerical computation (by finite difference) of elements of the
force constant matrix.

A ` 1' following a variable value in program input section 4
indicates that the number following on the same record should be used as
the initial diagonal second derivative for that variable. The number
must be in atomic units (Hartree-Bohr-radian).  To convert from
spectroscopic units (mdynes-Angstroms-radian) multiply by 0.064229 for
stretches, 0.22937 for bends and 0.121376 for strech-bond interactions.

A ` 2' following a variable value instruct the program to compute
that diagonal second derivative before proceding with the optimization.

A ` 3' following a variable value causes the diagonal and off-
diagonal second derivatives for that variable to be estimated by finite
differences.
\section{\sf RESTARTing a {\tt gnu80} Run}
\label{restart}
Usually, when a {\tt gnu80} job exits (either normally or abnormally), it
leaves potentially useful data on the disk in the form of read-write
files, two-electron integrals,  or post-SCF buckets. 

The {\bf RESTART}
feature allows a subsequent {\tt gnu80} job to recover and use these data,
provided they have been placed in non-scratch files. In the normal course
of events, {\tt gnu80} uses scratch files for both electron-repulsion
integral storage and general file use. It is possible, by the
use of {\em file control records} to make these files permanent.
There are two files, the ``integral'' file (for repulsion
integrals) and the ``read-write'' file (for general data
storage, inter-link communication etc.etc.).

The files are made permanent by placing one or two file control records
{\bf before} the Job Control Record in a {\tt gnu80} job input.
They both have the ` \% ' symbol in the first column of the record
and have the form:
\begin{description}
\item[\%int=filename1] 
\item[\%rwf=filename2] 
\end{description}
where ``filename1'' and ``filename2''
are two (different!) filenames which satisfy the naming conventions of
the local operating system (this is not checked by {\tt gnu80}, of course).
Note that,  unusually in {\tt gnu80}, case {\bf is}  significant and both files
will be generated with  {\bf names as entered} if this is significant
to the operating system. This may well cause
some surprises if a lower-case name was intended (in Unix, for example).

Both records have all blanks ignored (even {\em within} filenames, 
and examples are: \\
\ \\
{\bf \%int = repulsion.int} \\
\ \\
{\bf \%RWF=dump.rwf}
\ \\
\ \\
If you need blanks in your filenames, you must change the code
in {\tt SUBROUTINE L0CMND} to omit blank stripping.

If a job is to be restarted
using these saved files, the restart job must
have the same file control records
before the Job Command Record.
Sometimes, if files are to be updated during a sequence
of restarted runs, {\bf it is advisable to
copy the saved files and restart with the copies.}

The ROUTE is stored on the read-write file, so a job can most easily be 
{\em re}-started, rather than submitted as a continuation. 
It may be restarted at any point up to and including
the point at which it finished a complete Link. For example,
if a job has actually run to completion it may be restarted anywhere 
in the original ROUTE.

Thus it must be indicated {\bf where} the job  is to be restarted.  
This is done
with a record of the form:  \\
\ \\
{\bf \#RESTART Lxxx(N)}  \\
\ \\
The ` \#' must be in column 1, and ` Lxxx(N)' 
indicates that the job is to
restart at the N'th occurrence of link Lxxx. N is optional, and defaults
to 1. In fact, ` Lxxx' is optional also, and if not given, the job
restarts at the link in which the previous job terminated. Since the
original input  file is lost, you must supply any input to the job that
the remaining links will need. 

{\bf Note that the RESTART Command starts immediately after the \#,
there is no blank as in the Job Command Record}

It is also possible to supply  a new
route while  restarting.  In this case, the route information in the
read-write file is updated while the  rest of the data remains intact.
this is accomplished with the {\bf  RESTART} command:  \\
\ \\
\ \\
{\bf \#RESTART L1  \\
\# new route goes here....}  \\
\ \\
\ \\
This causes execution to resume with Link1, which will read a new
route. One must, of course, supply any data input required by the new
route. This is a method of avoiding the re-computation of the
repulsion integrals in a job related to an earlier one. Of course,
the new ROUTE must explicitly omit the calculation which is being
avoided (this usually means a {\bf NONSTD} command).

Thus a job originally submitted as: \\
\ \\
{\bf \%int = repulsion.int} \\
{\bf \%RWF=dump.rwf} \\
{\bf \# HF/6-31g} \\
(blank record) \\
Data Records \\
\ \\
can be restarted at the SCF stage by: \\
\ \\
{\bf \%int = repulsion.int} \\
{\bf \%RWF=dump.rwf} \\
{\bf \#RESTART L501} \\
(blank record) \\
\ \\
\ \\
always provided that the original job got as far as generating the
data for Link 501.
\subsection{\sf Restarting; An Example}
It often happens that it is difficult to get a Geometry Optimisation
to {\em start} in the sense that it is difficult to get a converged set
of MO coefficients for the initial geometry. Usually, since the 
changes in geometry are small during the geometry optimisation, once a set of
converged MO coefficients has been obtained for one point
these coefficients are good enough to ensure that, 
when used as a starting point for other geometries close by, things will
go smoothly. 

This is particularly galling if one has done a huge calculation of 
molecular integrals and the SCF does not converge. It is useful to be able
to try the SCF again with some different convergence aids. Here is a pair
of jobs which illustrate this practice.
\newpage
The first job is an SCF calculation on the water molecule 
with a standard basis:
\begin{verbatim}
     %INT=example.int
     %RWF=example.rwf
     # HF/3-21G,VSHIFT=500

       Test of the use of RESTART

     0 1
     O
     H 1 0.956
     H 1 0.956 2 104.5

\end{verbatim}
In point of fact this run does go to completion with the convergence
aid of ``Level Shifter" set to $0.5$ ({\tt VSHIFT = 500};
actual value $500/1000$) but such a job may have failed to converge.

Notice that the two files {\bf must} be saved to use for a {\tt RESTART}.
\newpage
Now we may restart the job from where it finished with a different
Level Shifter:

\begin{verbatim}
      %INT=example.int
      %RWF=example.rwf
      %SAVE=H2O.SAVE
      #RESTART L1
      # NONSTD
      5/9=3500,32=1/1,2;
      99//99;

\end{verbatim}

This time the Route has to be given explicitly by use of the {\tt NONSTD}
command; it was constructed by setting up a test Command Record anticipating
success and therefore saving the converged MO coefficients:

\begin{verbatim}
      # HF/3-21G,VSHIFT=3500,SAVE=MO
\end{verbatim}

and editing the output Route from {\tt gnu80}.

The meaning of the items are:
\begin{itemize}
\item
The Leading {\tt 5} is the Overlay to be run ( Overlay 5 is the
SCF overlay)
\item
{\tt 9=3500} is just {\tt VSHIFT=3500} in Overlay 5
\item
{\tt 32=1} is just {\tt SAVE=MO} in Overlay 5
\item
{\tt 1,2} are the Links to be run in Overlay 5 ({\tt L501, L502}
\end{itemize}
This job, when successfully run will generate a converged set 
of MO coefficients and save them for future use in an optimisation, say,
on file {\tt H2O.SAVE} suitable for use in a job like:
\newpage
\begin{verbatim}
     %GUESS=H2O.SAVE
     # HF/3-21G,VSHIFT=500,GUESS=READ,OPT

         Re-using the Converged MO coefficients

     0 1
     O
     H1 O ROH
     H2 O ROH H1 104.5

     ROH=0.956

\end{verbatim}
      
None of the working files are saved, again anticipating success.
\subsection{\sf Portability of RESTART Files}
For obvious reasons of speed of data transfer the two main
{\tt gnu80} files (RWF and INT) are binary files and therefore
not portable between machines of different types.
However the SAVE and GUESS files are written and read as
FORMATTED files and so can be moved from machine to
machine at will.
\subsection{\sf Disaster Recovery}
There is a utility program which will read the standard (``printed") output
of {\tt gnu80} and generate a SAVE file from the printed MO coefficients (assuming
that these occur in the standard output). If you have output from
a job which ran but generated no SAVE file and you wish to RESTART a similar
job, this can be done.

\subsubsection{\sf Error Messages}
If a Link is specified on the {\bf RESTART} record which is
not in the original ROUTE, an error message is given and the job terminates.
