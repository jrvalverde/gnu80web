\documentclass{article}
\begin{document}
\section{\sf Installation of {\tt gnu80}}
\begin{center}
  \fbox{
    \parbox{4.5in}{\bf Insofar as the source code of {\tt gnu80} is
      designed to be portable FORTRAN 77, there are no installation
      instructions except ``Generate the Fortran source from the FWEB
      files, Compile the FORTRAN and link it to form an executable
      module'' .

      However, if you need a simpler route, go to the directory {\tt
        gnu80web} and change the first line to read: \\
\ \\
        {\tt PREFIX="full path to where you are"} \\

       and then say: \\
\ \\
      {\tt ./build\_gnu80 > \& build\_gnu80.log \&} \\
\ \\
    After some time you should find the file {\tt a.out} has been 
    created in that
    directory together with a (large) {\tt build\_gnu80.log} file with
    all the gory details of what has happened during the build.
    
    If you do not have (or do not like) FWEB then you will have to go
    into the {\tt src} and {\tt NBO} directories and edit off all the
    FWEB details from the Fortran source which will be \emph{very}
    tedious.  } }
\end{center}

However, there are one or two points where the FORTRAN 77 standard is
not precise enough to ensure complete portability and it is wise to be
aware of these:
\begin{itemize}
\item The {\em units} of the {\tt RECL} (record length) parameter in a
  direct access {\tt OPEN} statement
\begin{verbatim}
      OPEN(UNIT=file,ACCESS='DIRECT',RECL=record_length, ...)
\end{verbatim}
  is not defined by the standard;
  typically it may be bytes or quadruples of bytes. \\
  In {\tt gnu80} as supplied a unit of {\em bytes} is assumed and {\tt
    RECL}
  is set to 16380 (=4*4095). \\
  To use the existing file I/O routines (via {\tt NTRAN}) the record
  length must be enough to store 4095 default-length {\tt INTEGERS}.
  If a different buffer is required, then {\tt SUBROUTINE NTRAN} must
  have the occurences of 4095 changed and the {\tt RECL} parameter
  changed accordingly.
\item The mode of storage of (particularly) characters in variables of
  type {\tt INTEGER} is not defined by the standard (see {\tt IORD}
  \ref{iord}).
\item The coding used to store characters is not defined although the
  ASCII code is almost universally implemented. {\tt gnu80} asumes
  that the ASCII code is used but it is hoped that the use of EBCDIC
  will not cause difficulties!
\end{itemize}
In addition to these points {\em within} the standard, there is an
important and systematic departure from the standard in this release
of {\tt gnu80}. Characters are stored in variables of non-{\tt
  CHARACTER} type either by reading in A-format or by {\tt DATA}
initialisation statements. This is endemic in the code and cannot be
eradicated from this release!
\begin{center}
  \fbox{
    \parbox{3.5in}{\bf If the storage of {\tt CHARACTER} constants in
      variables of non-{\tt CHARACTER} type is a compilation {\em
        error} on the target machine/compiler, not simply a {\em
        warning}, then this release of {\tt gnu80} cannot be installed
      on that system.  }
  }
\end{center}
\end{document}

