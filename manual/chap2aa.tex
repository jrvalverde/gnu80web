\section{\sf Output Examples}
\label{outex}
It is, for obvious reasons, undesirable  to give complete
output listing for all the input examples listed in Section \ref{inex}.
However, it {\em is} useful to have, for reference purposes, a record
of what the results of each example are. So, for each example
there is, in this Section, {\em one page} of severely
edited output which gives the essence of the calculation.
The exception to this type of display is Example 2 which is, perhaps
the most common type of {\tt gnu80} run; geometry optimisation
at the HF/STO-3G level. In this case the full output is listed together
with some commentary.
\newpage
\subsection{\sf Example 1 and 3---14}
\subsubsection{\sf Example 1}
{\small
\begin{verbatim}
SCF DONE:  E(RHF) =  -74.9627570354     A.U. AFTER    6 CYCLES
            CONVG  =    0.4234d-04             -V/T =   2.004974
 COMPONENT                       A.U.                   KCAL/MOL
 ---------------------------------------------------------------
 TOTAL                      -74.962757035408          -47040.629
 ELECTRONIC                 -84.169308351904          -52817.924
 NUCLEAR REPULSION            9.206551316496            5777.295
 KINETIC                     74.591723452750           46807.798
 POTENTIAL                 -149.554480488158          -93848.428
 ELECTRONIC POTENTIAL      -158.761031804654          -99625.723
 ONE-ELECTRON POTENTIAL    -196.983005149312         -123610.775
 TWO-ELECTRON POTENTIAL      38.221973344659           23985.053
 ORBITAL SYMMETRIES.
       OCCUPIED: (A1) (A1) (B2) (A1) (B1)
       VIRTUAL:  (A1) (B2)
  THE ELECTRONIC STATE IS 1-A1.
      MOLECULAR ORBITAL COEFFICIENTS
                           1         2         3         4         5
                         (A1)      (A1)      (B2)      (A1)      (B1)
     EIGENVALUES --   -20.24160  -1.26904  -0.61852  -0.45319  -0.39132
   1 1   O  1S         -0.99413   0.23274   0.00000  -0.10325   0.00000
   2        2S         -0.02659  -0.83304   0.00000   0.53728   0.00000
   3        2PX         0.00000   0.00000   0.00000   0.00000   1.00000
   4        2PY         0.00000   0.00000   0.60650   0.00000   0.00000
   5        2PZ        -0.00436  -0.13005   0.00000  -0.77698   0.00000
   6 2   H  1S          0.00599  -0.15880   0.44495  -0.27733   0.00000
   7 3   H  1S          0.00599  -0.15880  -0.44495  -0.27733   0.00000
          TOTAL ATOMIC CHARGES.
              1
  1   O   8.367459
  2   H   0.816270
  3   H   0.816270
 DIPOLE MOMENT (DEBYE): X= 0.0000   Y= 0.0000   Z= 1.7276   TOTAL= 1.7276
\end{verbatim}
}
\newpage
\subsubsection{\sf Example 3}
{\small
\begin{verbatim}

                      ---------------------------
                       !  OPTIMIZED PARAMETERS   !
                       ! (ANGSTROMS AND DEGREES) !
 ----------------------                           -----------------------
 !      NAME          VALUE   DERIVATIVE INFORMATION (ATOMIC UNITS)     !
 ------------------------------------------------------------------------
 !       R1          0.9893   -DE/DX =    0.000052                      !
 !       R2          0.9894   -DE/DX =   -0.000009                      !
 !        A        100.0135   -DE/DX =    0.00007                       !
 ------------------------------------------------------------------------
 GRADGRADGRADGRADGRADGRADGRADGRADGRADGRADGRADGRADGRADGRADGRADGRADGRADGRAD
 STANDARD BASIS: STO-3G     (S, S=P, 5D, 7F)
   7 BASIS FUNCTIONS       21 PRIMITIVE GAUSSIANS
   5 ALPHA ELECTRONS        5 BETA ELECTRONS
     NUCLEAR REPULSION ENERGY    8.9065689700 HARTREES
 RAFFENETTI 1 INTEGRAL FORMAT.
 TWO-ELECTRON INTEGRAL SYMMETRY IS TURNED OFF.
 ORBITAL SYMMETRIES.
       OCCUPIED: (A') (A') (A') (A') (A")
       VIRTUAL:  (A') (A')
  THE ELECTRONIC STATE IS 1-A'.
      MOLECULAR ORBITAL COEFFICIENTS
                           1         2         3         4         5
                         (A')      (A')      (A')      (A')      (A")
     EIGENVALUES --   -20.25159  -1.25757  -0.59383  -0.45976  -0.39263
   1 1   O  1S         -0.99422  -0.23376   0.00001  -0.10404   0.00000
   2        2S         -0.02585   0.84444  -0.00005   0.53821   0.00000
   3        2PX        -0.00319   0.09411  -0.39377  -0.57908   0.00000
   4        2PY         0.00000   0.00000   0.00000   0.00000   1.00000
   5        2PZ        -0.00268   0.07897   0.46944  -0.48576   0.00000
   6 2   H  1S          0.00558   0.15560   0.44923  -0.29509   0.00000
   7 3   H  1S          0.00558   0.15558  -0.44922  -0.29513   0.00000
\end{verbatim}
}
\newpage
\subsubsection{\sf Example 4}
{\small
\begin{verbatim}
                       ---------------------------
                       !  OPTIMIZED PARAMETERS   !
                       ! (ANGSTROMS AND DEGREES) !
 ----------------------                           -----------------------
 !      NAME          VALUE   DERIVATIVE INFORMATION (ATOMIC UNITS)     !
 ------------------------------------------------------------------------
 !        R          0.9867   -DE/DX =    0.000072                      !
 ------------------------------------------------------------------------
 GRADGRADGRADGRADGRADGRADGRADGRADGRADGRADGRADGRADGRADGRADGRADGRADGRADGRAD
 STANDARD BASIS: STO-3G     (S, S=P, 5D, 7F)
   7 BASIS FUNCTIONS       21 PRIMITIVE GAUSSIANS
   5 ALPHA ELECTRONS        5 BETA ELECTRONS
     NUCLEAR REPULSION ENERGY    8.9188268795 HARTREES
 RAFFENETTI 1 INTEGRAL FORMAT.
 TWO-ELECTRON INTEGRAL SYMMETRY IS TURNED ON.
 ORBITAL SYMMETRIES.
       OCCUPIED: (A1) (A1) (B2) (A1) (B1)
       VIRTUAL:  (A1) (B2)
  THE ELECTRONIC STATE IS 1-A1.
      MOLECULAR ORBITAL COEFFICIENTS
                           1         2         3         4         5
                         (A1)      (A1)      (B2)      (A1)      (B1)
     EIGENVALUES --   -20.24477  -1.25333  -0.60334  -0.44828  -0.38917
   1 1   O  1S          0.99420   0.23402   0.00000   0.10125   0.00000
   2        2S          0.02597  -0.84450   0.00000  -0.52539   0.00000
   3        2PX         0.00000   0.00000   0.00000   0.00000   1.00000
   4        2PY         0.00000   0.00000   0.60484   0.00000   0.00000
   5        2PZ         0.00405  -0.11908   0.00000   0.77187   0.00000
   6 2   H  1S         -0.00565  -0.15655   0.44665   0.28889   0.00000
   7 3   H  1S         -0.00565  -0.15655  -0.44665   0.28889   0.00000
\end{verbatim}
}
\newpage
\subsubsection{\sf Example 5}
{\small
\begin{verbatim}
                       ---------------------------
                       !  OPTIMIZED PARAMETERS   !
                       ! (ANGSTROMS AND DEGREES) !
 ----------------------                           -----------------------
 !      NAME          VALUE   DERIVATIVE INFORMATION (ATOMIC UNITS)     !
 ------------------------------------------------------------------------
 !       CF          1.404    -DE/DX =    0.000001                      !
 !       CH          1.0794   -DE/DX =    0.000021                      !
 !       HCF       109.3882   -DE/DX =    0.000004                      !
 ------------------------------------------------------------------------
 GRADGRADGRADGRADGRADGRADGRADGRADGRADGRADGRADGRADGRADGRADGRADGRADGRADGRAD
 STANDARD BASIS: 3-21G     (S, S=P, 5D, 7F)
  24 BASIS FUNCTIONS       39 PRIMITIVE GAUSSIANS
   9 ALPHA ELECTRONS        9 BETA ELECTRONS
     NUCLEAR REPULSION ENERGY   37.0948856107 HARTREES
 RAFFENETTI 1 INTEGRAL FORMAT.
 TWO-ELECTRON INTEGRAL SYMMETRY IS TURNED ON.
 ORBITAL SYMMETRIES.
       OCCUPIED: (A1) (A1) (A1) (A1) (E) (E) (A1) (E) (E)
       VIRTUAL:  (A1) (E) (E) (A1) (E) (E) (A1) (A1) (E) (E) (A1)
                 (E) (E) (A1) (A1)
 UNABLE TO DETERMINE ELECTRONIC STATE.
                           1         2         3         4         5
                         (A1)      (A1)      (A1)      (A1)       (E)
     EIGENVALUES --   -26.11975 -11.24559  -1.54808  -0.95383  -0.67357
                           6         7         8         9        10
                          (E)      (A1)       (E)       (E)      (A1)
     EIGENVALUES --    -0.67357  -0.63411  -0.51726  -0.51726   0.27861
\end{verbatim}
}
\newpage
\subsubsection{\sf Example 6}
{\small
\begin{verbatim}
                       ---------------------------
                       !  OPTIMIZED PARAMETERS   !
                       ! (ANGSTROMS AND DEGREES) !
 ----------------------                           -----------------------
 !      NAME          VALUE   DERIVATIVE INFORMATION (ATOMIC UNITS)     !
 ------------------------------------------------------------------------
 !       NH          1.0026   -DE/DX =   -0.000036                      !
 !       HNX        73.6317   -DE/DX =    0.000042                      !
 ------------------------------------------------------------------------
 GRADGRADGRADGRADGRADGRADGRADGRADGRADGRADGRADGRADGRADGRADGRADGRADGRADGRAD
 STANDARD BASIS: 3-21G     (S, S=P, 5D, 7F)
  15 BASIS FUNCTIONS       24 PRIMITIVE GAUSSIANS
   5 ALPHA ELECTRONS        5 BETA ELECTRONS
     NUCLEAR REPULSION ENERGY   12.0363428366 HARTREES
 RAFFENETTI 1 INTEGRAL FORMAT.
 TWO-ELECTRON INTEGRAL SYMMETRY IS TURNED ON.
 ORBITAL SYMMETRIES.
       OCCUPIED: (A1) (A1) (E) (E) (A1)
       VIRTUAL:  (A1) (E) (E) (E) (E) (A1) (A1) (E) (E) (A1)
 UNABLE TO DETERMINE ELECTRONIC STATE.
                           1         2         3         4         5
                         (A1)      (A1)       (E)       (E)      (A1)
     EIGENVALUES --   -15.43152  -1.12003  -0.62031  -0.62031  -0.38894
                           6         7         8         9        10
                         (A1)       (E)       (E)       (E)       (E)
     EIGENVALUES --     0.27525   0.37737   0.37737   1.18654   1.18654
  15        1S  (O)     0.24962
          CONDENSED TO ATOMS (ALL ELECTRONS)
              1          2          3          4
  1   N   6.867228   0.336337   0.336337   0.336337
  2   H   0.336337   0.436131  -0.032274  -0.032274
  3   H   0.336337  -0.032274   0.436131  -0.032274
  4   H   0.336337  -0.032274  -0.032274   0.436131
          TOTAL ATOMIC CHARGES.
              1
  1   N   7.876238
  2   H   0.707921
  3   H   0.707921
  4   H   0.707921
 DIPOLE MOMENT (DEBYE): X= 0.0000   Y= 0.0000   Z= 1.7521   TOTAL= 1.7521
\end{verbatim}
}
\newpage
\subsubsection{\sf Example 7}
{\small
\begin{verbatim}
                       ---------------------------
                       !  OPTIMIZED PARAMETERS   !
                       ! (ANGSTROMS AND DEGREES) !
 ----------------------                           -----------------------
 !      NAME          VALUE   DERIVATIVE INFORMATION (ATOMIC UNITS)     !
 ------------------------------------------------------------------------
 !       NH          1.0026   -DE/DX =   -0.00005                       !
 !       HNH       112.3966   -DE/DX =   -0.000004                      !
 ------------------------------------------------------------------------
 GRADGRADGRADGRADGRADGRADGRADGRADGRADGRADGRADGRADGRADGRADGRADGRADGRADGRAD
 STANDARD BASIS: 3-21G     (S, S=P, 5D, 7F)
  15 BASIS FUNCTIONS       24 PRIMITIVE GAUSSIANS
   5 ALPHA ELECTRONS        5 BETA ELECTRONS
     NUCLEAR REPULSION ENERGY   12.0364013291 HARTREES
 RAFFENETTI 1 INTEGRAL FORMAT.
 TWO-ELECTRON INTEGRAL SYMMETRY IS TURNED ON.
 ORBITAL SYMMETRIES.
       OCCUPIED: (A1) (A1) (E) (E) (A1)
       VIRTUAL:  (A1) (E) (E) (E) (E) (A1) (A1) (E) (E) (A1)
 UNABLE TO DETERMINE ELECTRONIC STATE.
                           1         2         3         4         5
                         (A1)      (A1)       (E)       (E)      (A1)
     EIGENVALUES --   -15.43150  -1.12001  -0.62032  -0.62032  -0.38892
                           6         7         8         9        10
                         (A1)       (E)       (E)       (E)       (E)
     EIGENVALUES --     0.27526   0.37738   0.37738   1.18660   1.18660
          CONDENSED TO ATOMS (ALL ELECTRONS)
              1          2          3          4
  1   N   6.867255   0.336349   0.336349   0.336349
  2   H   0.336349   0.436084  -0.032267  -0.032267
  3   H   0.336349  -0.032267   0.436084  -0.032267
  4   H   0.336349  -0.032267  -0.032267   0.436084
          TOTAL ATOMIC CHARGES.
              1
  1   N   7.876301
  2   H   0.707900
  3   H   0.707900
  4   H   0.707900
 DIPOLE MOMENT (DEBYE): X= 0.0000   Y= 0.0000   Z= 1.7513   TOTAL= 1.7513
\end{verbatim}
}
\newpage
\subsubsection{\sf Example 8}
{\small
\begin{verbatim}
                       ---------------------------
                       !  OPTIMIZED PARAMETERS   !
                       ! (ANGSTROMS AND DEGREES) !
 ----------------------                           -----------------------
 !      NAME          VALUE   DERIVATIVE INFORMATION (ATOMIC UNITS)     !
 ------------------------------------------------------------------------
 !     HALFCC        0.7371   -DE/DX =   -0.00006                       !
 !       OX          1.2713   -DE/DX =   -0.000051                      !
 !       CH          1.0707   -DE/DX =    0.000014                      !
 !       HCC       119.2413   -DE/DX =    0.000031                      !
 !      HCCO       103.2904   -DE/DX =    0.000055                      !
 ------------------------------------------------------------------------
 GRADGRADGRADGRADGRADGRADGRADGRADGRADGRADGRADGRADGRADGRADGRADGRADGRADGRAD
 STANDARD BASIS: 3-21G     (S, S=P, 5D, 7F)
  35 BASIS FUNCTIONS       57 PRIMITIVE GAUSSIANS
  12 ALPHA ELECTRONS       12 BETA ELECTRONS
     NUCLEAR REPULSION ENERGY   74.3281278409 HARTREES
 RAFFENETTI 1 INTEGRAL FORMAT.
 TWO-ELECTRON INTEGRAL SYMMETRY IS TURNED ON.
 ORBITAL SYMMETRIES.
       OCCUPIED: (A1) (B2) (A1) (A1) (A1) (B2) (B1) (A1) (A2) (B2)
                 (A1) (B1)
       VIRTUAL:  (A1) (B2) (B1) (A1) (B2) (A2) (B2) (A1) (B1) (B2)
  THE ELECTRONIC STATE IS 1-A1.
                           1         2         3         4         5
                         (A1)      (B2)      (A1)      (A1)      (A1)
     EIGENVALUES --   -20.46808 -11.23219 -11.23184  -1.39571  -0.94125
                           6         7         8         9        10
                         (B2)      (B1)      (A1)      (A2)      (B2)
     EIGENVALUES --    -0.86460  -0.70946  -0.64316  -0.55335  -0.52035
                          11        12        13        14        15
                         (A1)      (B1)      (A1)      (B2)      (B1)
     EIGENVALUES --    -0.44599  -0.44550   0.27678   0.27965   0.31400
          TOTAL ATOMIC CHARGES.
              1
  1   C   6.191493
  2   O   8.551892
  3   C   6.191493
  4   H   0.766280
  5   H   0.766280
  6   H   0.766280
  7   H   0.766280
 DIPOLE MOMENT (DEBYE): X= 0.0000   Y= 0.0000   Z=-2.7845   TOTAL= 2.7845
\end{verbatim}
}
\newpage
\subsubsection{\sf Example 9}
{\small
\begin{verbatim}
                       ---------------------------
                       !  OPTIMIZED PARAMETERS   !
                       ! (ANGSTROMS AND DEGREES) !
 ----------------------                           -----------------------
 !      NAME          VALUE   DERIVATIVE INFORMATION (ATOMIC UNITS)     !
 ------------------------------------------------------------------------
 !       CN          1.1325   -DE/DX =   -0.000135                      !
 !       CH          1.0588   -DE/DX =    0.000142                      !
 ------------------------------------------------------------------------
 GRADGRADGRADGRADGRADGRADGRADGRADGRADGRADGRADGRADGRADGRADGRADGRADGRADGRAD
 STANDARD BASIS: 6-31G*     (S, S=P, 6D, 7F)
  32 BASIS FUNCTIONS       60 PRIMITIVE GAUSSIANS
   7 ALPHA ELECTRONS        7 BETA ELECTRONS
     NUCLEAR REPULSION ENERGY   24.3124728053 HARTREES
 RAFFENETTI 1 INTEGRAL FORMAT.
 TWO-ELECTRON INTEGRAL SYMMETRY IS TURNED OFF.
 ORBITAL SYMMETRIES.
       OCCUPIED: (SG) (SG) (SG) (SG) (SG) (PI) (PI)
       VIRTUAL:  (PI) (PI) (SG) (SG) (PI) (PI) (SG) (SG) (PI) (PI)
                 (SG) (SG) (SG) (DLTA) (DLTA) (PI) (PI) (DLTA)
                 (DLTA) (PI) (PI) (SG) (SG) (SG) (SG)
 UNABLE TO DETERMINE ELECTRONIC STATE.
                           1         2         3         4         5
                         (SG)      (SG)      (SG)      (SG)      (SG)
     EIGENVALUES --   -15.59707 -11.28843  -1.24284  -0.80838  -0.57381
                           6         7         8         9        10
                         (PI)      (PI)      (PI)      (PI)      (SG)
     EIGENVALUES --    -0.49705  -0.49705   0.20433   0.20433   0.23318
          CONDENSED TO ATOMS (ALL ELECTRONS)
              1          2          3
  1   N   6.475039   0.928252  -0.024403
  2   C   0.928252   4.654130   0.351305
  3   H  -0.024403   0.351305   0.360523
          TOTAL ATOMIC CHARGES.
              1
  1   N   7.378888
  2   C   5.933687
  3   H   0.687425
 DIPOLE MOMENT (DEBYE): X= 0.0000   Y= 0.0000   Z= 3.2086   TOTAL= 3.2086
\end{verbatim}
}
\newpage
\subsubsection{\sf Example 10}
{\small
\begin{verbatim}
                       ---------------------------
                       !  OPTIMIZED PARAMETERS   !
                       ! (ANGSTROMS AND DEGREES) !
 ----------------------                           -----------------------
 !      NAME          VALUE   DERIVATIVE INFORMATION (ATOMIC UNITS)     !
 ------------------------------------------------------------------------
 !       CN          1.1398   -DE/DX =   -0.000449                      !
 !       CO          1.3084   -DE/DX =   -0.000121                      !
 !       OH          0.9703   -DE/DX =    0.000022                      !
 !      HALF        90.6316   -DE/DX =    0.000051                      !
 !       COH       114.1532   -DE/DX =   -0.00011                       !
 ------------------------------------------------------------------------
 GRADGRADGRADGRADGRADGRADGRADGRADGRADGRADGRADGRADGRADGRADGRADGRADGRADGRAD
 STANDARD BASIS: 3-21G     (S, S=P, 5D, 7F)
  29 BASIS FUNCTIONS       48 PRIMITIVE GAUSSIANS
  11 ALPHA ELECTRONS       11 BETA ELECTRONS
     NUCLEAR REPULSION ENERGY   58.2705799512 HARTREES
 RAFFENETTI 1 INTEGRAL FORMAT.
 TWO-ELECTRON INTEGRAL SYMMETRY IS TURNED OFF.
 ORBITAL SYMMETRIES.
       OCCUPIED: (A') (A') (A') (A') (A') (A') (A') (A") (A') (A')
       VIRTUAL:  (A') (A') (A") (A') (A') (A") (A') (A') (A') (A')
  THE ELECTRONIC STATE IS 1-A'.
                           1         2         3         4         5
                         (A')      (A')      (A')      (A')      (A')
     EIGENVALUES --   -20.57074 -15.50999 -11.29438  -1.48307  -1.26122
                           6         7         8         9        10
                         (A')      (A')      (A")      (A')      (A')
     EIGENVALUES --    -0.85979  -0.68335  -0.65765  -0.55829  -0.47341
                          11        12        13        14        15
                         (A")      (A')      (A')      (A")      (A')
     EIGENVALUES --    -0.45424   0.18647   0.24987   0.26428   0.36665
          TOTAL ATOMIC CHARGES.
              1
  1   N   7.428592
  2   C   5.317444
  3   O   8.703003
  4   H   0.550962
 DIPOLE MOMENT (DEBYE): X= 2.1123   Y= 0.0000   Z= 3.2717   TOTAL= 3.8943
\end{verbatim}
}
\newpage
\subsubsection{\sf Example 11}
{\small
\begin{verbatim}
 NUMBER OF SYMMETRY OPERATIONS:  2
 STANDARD BASIS: STO-3G     (S, S=P, 5D, 7F)
   7 BASIS FUNCTIONS       21 PRIMITIVE GAUSSIANS
   5 ALPHA ELECTRONS        4 BETA ELECTRONS
     NUCLEAR REPULSION ENERGY    7.4893197651 HARTREES
 RAFFENETTI 2 INTEGRAL FORMAT.
 TWO-ELECTRON INTEGRAL SYMMETRY IS TURNED ON.
   INITIAL GUESS WAVE FUNCTION.
 REAL ALPHA MO COEFFICIENTS.
                           1         2         3         4         5
   1 1   N  1S         -1.00079  -0.19876   0.00000  -0.02648   0.00000
   2        2S          0.00128   0.76494   0.00000   0.14715   0.00000
   3        2PX         0.00000   0.00000   0.00000   0.00000   1.00000
   4        2PY         0.00000   0.00000  -0.52797   0.00000   0.00000
   5        2PZ        -0.00195  -0.04939   0.00000  -0.96265   0.00000
   6 2   H  1S          0.00462   0.23018  -0.46197  -0.07274   0.00000
   7 3   H  1S          0.00462   0.23018   0.46197  -0.07274   0.00000
 REAL BETA MO COEFFICIENTS.
                           1         2         3         4         5
   1 1   N  1S         -1.00079  -0.19876   0.00000  -0.02648   0.00000
   2        2S          0.00128   0.76494   0.00000   0.14715   0.00000
   3        2PX         0.00000   0.00000   0.00000   0.00000   1.00000
   4        2PY         0.00000   0.00000  -0.52797   0.00000   0.00000
   5        2PZ        -0.00195  -0.04939   0.00000  -0.96265   0.00000
   6 2   H  1S          0.00462   0.23018  -0.46197  -0.07274   0.00000
   7 3   H  1S          0.00462   0.23018   0.46197  -0.07274   0.00000
 INITIAL GUESS ORBITAL SYMMETRIES.
   ALPHA ORBITALS
       OCCUPIED: (A1) (A1) (B2) (A1) (B1)
       VIRTUAL:  (B2) (A1)
   BETA ORBITALS
       OCCUPIED: (A1) (A1) (B2) (A1)
       VIRTUAL:  (B1) (B2) (A1)
\end{verbatim}
}
\newpage
\subsubsection{\sf Example 12}
{\small
\begin{verbatim}
                       ---------------------------
                       !  OPTIMIZED PARAMETERS   !
                       ! (ANGSTROMS AND DEGREES) !
 ----------------------                           -----------------------
 !      NAME          VALUE   DERIVATIVE INFORMATION (ATOMIC UNITS)     !
 ------------------------------------------------------------------------
 !       NH          1.015    -DE/DX =   -0.00006                       !
 !       HNH       131.292    -DE/DX =    0.000012                      !
 ------------------------------------------------------------------------
 GRADGRADGRADGRADGRADGRADGRADGRADGRADGRADGRADGRADGRADGRADGRADGRADGRADGRAD
 STANDARD BASIS: STO-3G     (S, S=P, 5D, 7F)
   7 BASIS FUNCTIONS       21 PRIMITIVE GAUSSIANS
   5 ALPHA ELECTRONS        4 BETA ELECTRONS
     NUCLEAR REPULSION ENERGY    7.5846617923 HARTREES
 RAFFENETTI 2 INTEGRAL FORMAT.
 TWO-ELECTRON INTEGRAL SYMMETRY IS TURNED ON.
 UHF OPEN SHELL SCF.
 ORBITAL SYMMETRIES.
   ALPHA ORBITALS
       OCCUPIED: (A1) (A1) (B2) (A1) (B1)
       VIRTUAL:  (A1) (B2)
   BETA ORBITALS
       OCCUPIED: (A1) (A1) (B2) (B1)
       VIRTUAL:  (A1) (A1) (B2)
  THE ELECTRONIC STATE IS 2-A1.
                           1         2         3         4         5
                         (A1)      (A1)      (B2)      (A1)      (B1)
     EIGENVALUES --   -15.26364  -1.05077  -0.61200  -0.45160  -0.33075
                           1         2         3         4         5
                         (A1)      (A1)      (B2)      (B1)      (A1)
     EIGENVALUES --   -15.23574  -0.93643  -0.57761  -0.27520   0.28110
          ATOMIC SPIN DENSITIES.
              1          2          3
  1   N   1.030351  -0.020328  -0.020328
  2   H  -0.020328   0.017559   0.007920
  3   H  -0.020328   0.007920   0.017559
 DIPOLE MOMENT (DEBYE): X= 0.0000   Y= 0.0000   Z= 0.7624   TOTAL= 0.7624
  FERMI CONTACT ANALYSIS (ATOMIC UNITS).
              1
  1   N   0.515487
  2   H   0.009898
  3   H   0.009898
\end{verbatim}
}
\newpage
\subsubsection{\sf Example 13}
{\small
\begin{verbatim}
SCF DONE:  E(RHF) =  -76.0235780622     A.U. AFTER   14 CYCLES
            CONVG  =    0.1016d-07             -V/T =   2.002297
 ORBITAL SYMMETRIES.
       OCCUPIED: (A1) (A1) (B2) (A1) (B1)
       VIRTUAL:  (A1) (B2) (B2) (A1) (A1) (B1) (B2) (A1) (A2) (A1)
                 (B1) (B2) (A1) (B2) (B1) (A2) (A1) (A1) (B2) (A1)
  THE ELECTRONIC STATE IS 1-A1.
      MOLECULAR ORBITAL COEFFICIENTS
                           1         2         3         4         5
                         (A1)      (A1)      (B2)      (A1)      (B1)
     EIGENVALUES --   -20.55824  -1.34479  -0.71084  -0.56854  -0.49750
 DIPOLE MOMENT (DEBYE): X= 0.0000   Y= 0.0000   Z= 2.1591   TOTAL= 2.1591
 RANGE OF M.O.'S USED FOR CORRELATION:   2  25
 RHF INTEGRAL TRANSFORMATION:
                                 5395 M.O.-INTEGRALS CREATED
 NORM(A1)=  0.10246d+01
 E2=       -0.19576446d+00        EUMP2=      -0.76219342518d+02
 MOLLER-PLESSET THIRD ORDER PERTURBATION THEORY
 **********************************************
 E3=       -0.64696469d-02        EUMP3=      -0.76225812165d+02
 VARIATIONAL ENERGIES WITH THE FIRST-ORDER WAVEFUNCTION:
 E(VAR1)=  -0.76216222061d+02     E(CID,4)=   -0.76219291969d+02
\end{verbatim}
}
\newpage
\subsubsection{\sf Example 14}
{\small
\begin{verbatim}
SCF DONE:  E(RHF) =  -76.0235780622     A.U. AFTER   14 CYCLES
            CONVG  =    0.1016d-07             -V/T =   2.002297
 RANGE OF M.O.'S USED FOR CORRELATION:   2  25
 RHF INTEGRAL TRANSFORMATION:
                                 5395 M.O.-INTEGRALS CREATED
 NORM(A1)=  0.10246d+01
 E2=       -0.19576446d+00        EUMP2=      -0.76219342518d+02
 CONFIGURATION INTERACTION WITH DOUBLE SUBSTITUTIONS
 ***************************************************
 ITERATIONS= 30   CONVERGENCE= 0.200d-05
 ITERATION STEP#  1
 ******************
 E3=       -0.64696469d-02        EUMP3=      -0.76225812165d+02
 DE(CI)=   -0.19264400d+00        E(CI)=      -0.76216222061d+02
 NORM(A)=   0.10237356d+01
 SIZE-CONSISTENCY CORRECTION: 
 S.C.C.=   -0.71077439d-02        E(CI,SIZE)= -0.76223329804d+02
 ITERATION STEP#  2
 ******************
 DE(CI)=   -0.19599910d+00        E(CI)=      -0.76219577165d+02
 NORM(A)=   0.10246551d+01
 SIZE-CONSISTENCY CORRECTION: 
 S.C.C.=   -0.75220564d-02        E(CI,SIZE)= -0.76227099221d+02
 ITERATION STEP#  3
 ******************
  ....
  ....
 ITERATION STEP#  6
 ******************
 EXTRAPOLATION PERFORMED
 DE(CI)=   -0.19629698d+00        E(CI)=      -0.76219875043d+02
 NORM(A)=   0.10250212d+01
 SIZE-CONSISTENCY CORRECTION: 
 S.C.C.=   -0.76495816d-02        E(CI,SIZE)= -0.76227524625d+02
 ITERATION STEP#  7
 ******************
 DE(CI)=   -0.19629722d+00        E(CI)=      -0.76219875282d+02
 NORM(A)=   0.10250232d+01
 SIZE-CONSISTENCY CORRECTION: 
 S.C.C.=   -0.76502237d-02        E(CI,SIZE)= -0.76227525506d+02
 ***************************************************************
\end{verbatim}
}
\newpage
\subsection{\sf Example 2}
Here is an annotated copy of the output
generated by Example 2 of   Section \ref{inex};
perhaps the most common type of {\tt gnu80} run.
The geometry of the water molecule is optimised using
the (closed-shell) HF model and a simple STO-3G
orbital basis; the geometry is constrained to remain at
$C_{2v}$ throughout the optimisation
by the form of the Molecule input section.

This is typical output from a job which requests 
{\bf OPT}imization of a Hartree-Fock
structure.  The following points are of general interest:
\begin{enumerate}
\item  The command record is reproduced.
\item  The detailed route through the links in the program is given.  This
is a standard route generated from the command record.  For detailed
interpretation of the route information, see \ref{app3b}.
\item  The title input section is reproduced.
\item  The charge, multiplicity and input Z-matrix are reproduced together
with initial values of the parameters.  Any inconsistencies (e.g.
incompatible multiplicity and electron number, parameter names not
defined in Z-matrix, etc.) would be noted here and the run aborted.
\item  The string GRADGRAD... delimit output from the Berny optimization
program, link 103.  On the initialization pass through this link, a
table giving the initial values of the variables to be optimized is
printed.  Diagonal second derivatives of the energy with respect to
the variables are needed by the optimization program initial values
are estimated via an empirical procedure.
\item  The initial Z-matrix is listed with the numerical values of all
parameters substituted.  This is followed by a set of Cartesian
coordinates generated from this information and a matrix of
internuclear distance.  The last two of these are useful for debugging
Z-matrix input.  Note that, by default, the run is aborted at this
point if an interatomic distance of less than 0.5A is computed or
if an invalid Z-matrix parameter (such as a negative length) is
encountered.
\item  The next part of the output refers to symmetry aspects of the molecule.
The stoichiometry, framework group    and the corresponding number of
degrees of freedom    are listed.  (Some fairly evident shorthand
symbols such as SGV for $\sigma_v$ are used in the framework notation.) If
the number of variables is equal to the number of degrees of freedom,
the optimization is {\em full} within the symmetry constraints imposed by
the framework group (implicit in the Z-matrix).  If the number of
variables is less than the number of degrees of freedom, the optimization
is {\em partial}.  After the framework group has been determined, axes are
rotated to a standard orientation in which symmetry elements coincide
with Cartesian elements.  (Full explanation of the standard orientation
is in {\tt SUBROUTINE ORDOC} in link 202.)  This does not take place if the
symmetry is so low that if leads to no useful gain in efficiency.  The
new Cartesian coordinates are listed and then the number of symmetry
operations actually used by the program is given. 
({\tt gnu80} only makes use of twofold operations, so this number
may be less than total number of symmetry operations.)
\item  Quantum mechanical data such as basis sets, number of electrons and
number and nature of two-electron integrals are printed next.  (See
Section \ref{stor2e} for a description of integral formats.)  Note that the
information following the basis set in parentheses is in error but it is
only the {\em printing} which is in error!
\item  The initial guess program prints the 
symmetries of the orbitals occupied
in the initial wave function.  If the orbital symmetry assignment fails,
this part of the output will contain some question marks, but the job
will proceed.  If the GUESS=PRINT command had been given, the full
initial-guess wave function would have been printed here.
\item The result of the SCF calculation (RHF in this example) is given
next.
A convergance failure would have been noted 
had it occurred and would have been followed
by abortion of the run.  If this occurs, the initial guess should be
checked and the job rebsummitted with the SCFDM keyword command.  CONVG
is the RMS difference between the density matrix elements in consecutive
cycles of the classical SCF procedure.  -V/T is the virial ratio.
\item The orbital symmetries of the wavefunction are then given and this is
followed by the full function and the one-electron eigenvalues.  The
latter give the Koopmans' approximation to the corresponding ionization
potentials (in Hartrees).  In general only the first five virtual
molecular orbitals are printed.  Note that the printed symmetry of the
wave function is wrong if there are any degenerate occupied molecular
orbitals.
\item A Mulliken population analysis is next carried out; the dipole moment
is also given.  All of this output uses the standard orientation.  If
the optimization (OPT) command had not been used, the output would have
ended here.  As the population analysis matrix is symmetric it is
necessary to double the values of the off-diagonal elements to get
an overlap population.
\item The axes are next restored to the original set (as noted in the output)
This is followed by a calculation of the Cartesian forces (in Hartrees/
Bohr) and the corresponding derivatives with respect to the internal
coordinates (lengths and angles used in the Z-matrix).  If the FORCE
command had been used instead of OPT, the output would have ended here.
\item The program then proceeds to the next step in the optimization.
Note that all of the output from the optimization program is in atomic
units.  Using the current second-derivative (Hessian) matrix and the
calculated first derivative vector, projection is made to a new set of
values for the variables. (NEW X).  In general, each step is composed
of a linear search along the line connecting the current and previous
points and a Newton-Raphson (quadratic) step.  Under certain conditions
one of these searches may be omitted in which case an appropriate
explanation is printed.
\item There are four conditions which must be met before it is deemed that
a stationary point has been located.  A table is printed showing the
current status of each of the conditions.
\item A new Hartree-Fock calculation is now initiated and carried through
in the same way as before (except that the previous wave function is
used for the new initial guess).  This leads to further optimization
steps.
\item The optimization is terminated when all four conditions are
simultaneously met, as indicated by message OPTIMIZATION COMPLETED.
(If the optimization procedure fails then resubmission with OPT=MS is
recommended.)  A table of the values of the variables at the stationary
point is then printed along with the final forces.  Note that the
variable values are printed an Angstroms and degrees but that the
derivatives are given in Hartree/Bohr or Hartree/radian.
\item The final wave function and Mulliken analysis is printed.
\end{enumerate}
