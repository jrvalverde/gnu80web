\chapter{\sf Stubs, Ghosts and Error Messages}
\section{\sf Stubs and Ghosts}
Stubs and Ghosts are either the remains of earlier stages in the
development of a program or unfulfilled promises for the future
development of the system.
\begin{description}
\item[Stub] A Stub is a program segment ({\tt SUBROUTINE} or {\tt FUNCTION}
which has a correct interface to be called but actually does nothing
(except possibly output a message that it is a Stub) and returns control
to the caller immediately.
Its existence may be evidence of the intention to provide a new facility
or that a redundant facility has been removed.
\item[Ghost] A Ghost is a piece of code which is {\em in practice}
unreachable. Again, it may be evidence of good intentions or of
debugging. Most compilers will object to the presence of code which is
{\em in principle} unreachable (un-numbered statements following a
transfer of control, for example) but they cannot detect situations like
\begin{verbatim}
            I = 0
            IF(I .EQ. 0) GO TO 999
      C     Unreachable
            A = B
\end{verbatim}
\end{description}
{\tt gnu80} has many Stubs and Ghosts which exist for both types of
reason. In particular, there are Stubs which are generated by the
removal of machine-specific (Assembler) segments. They could be removed
by systematically deleting the (useless) calls.
\section{\sf Error Messages}
There is no explicit list of error messages generated by {\tt gnu80}
because they are just that; error {\em messages} not error {\em codes}.
They are mostly completely self-explanatory.
