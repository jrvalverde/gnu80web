\chapter*{\sf How to use this Manual}
\thispagestyle{plain}
\begin{itemize}
\item If you come to a copy of {\tt gnu80} already installed on a
particular system, you may only need to browse through
Chapter \ref{chap1} and its Appendix and use the information
in Chapter \ref{chap2} and its Appendices in order to get useful
work from the system.
\item If you wish to install the program read  Appendix \ref{install} before
proceeding (it is only one or two pages!).
\item If you intend to install, use and modify the system, you
will need to browse through most of the manual and ultimately through
the code.
\end{itemize}
All users will benefit from browsing through the first two Chapters and
their Appendices.
\chapter*{\sf History}
\thispagestyle{plain}
{\tt gnu80} is a connected system of programs capable of
performing {\em ab initio} Molecular Orbital (MO) calculations within the
Linear Combination of Atomic Orbitals (LCAO) framework 
plus a restricted but extremely useful
ability to calculate ``post Hartree-Fock'' (correlated) electron
distributions and energies. 

{\tt gnu80} is a
further development of the Gaussian 70, 76 and 80 systems already
published. Gaussian 80 was originally implemented by J.A. Pople {\em et al.}
and distributed (at cost via the Quantum Chemistry Program Exchange)
for a DEC
Vax-11/780.  Gaussian 80 was subsequently implemented 
on the Amdhal V7b with
IBM's operating system MVS-3.8 at the State University of Leiden, The
Netherlands by P.N. van Kampen, G.F. Smits, F.A.A.M. de Leeuw and C.
Altona. 

Since the release of GAUSSIAN 80 all subsequent enhancements of
the GAUSSIAN series of programs (82, 86, 88) have been sold under
strict licence which, in particular, excludes their distribution
to third parties. Since it is the aim of the work done here to make
the software as freely available as possible,
it is from the last 
{\em 
public domain} 
version of the GAUSSIAN
series that {\tt gnu80} has been developed.

This decision, of course, means that the enhancements to the
system 
by the Carnegie-Mellon Team 
since GAUSSIAN 80 are not included, but the principle
``workhorse'' parts of the system are present 
(HF geometry optimisations,
MP2 calculations etc.) and the method of interfacing user modules
with {\tt gnu80} is explicitly given.

\thispagestyle{plain}
{\bf In fact, this software is distributed on the strict understanding
that it  must be improved or passed on to a third party
who will improve it.
}

%This manual has been developed from the comments included in the
%FORTRAN source and, although no charge is made for the {\tt gnu80}
%{\em program}, the costs involved in production and maintenance
%of this document have to be recouped by means of a charge for
%the documentation.

